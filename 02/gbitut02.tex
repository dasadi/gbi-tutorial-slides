%beamer

% Comment/uncomment this line to toggle handout mode
\newcommand{\handout}{}

\input{../framework/PraeambelTut.tex}

\morescalingdelimiters

\begin{document}
\starttut{2}

\section{Organisatorisches}

\begin{frame}[t]{Änderung Übungsblattabgabe}
	So sollte eure Abgabe aussehen:
	\begin{itemize}
		\item Namen und Matrikelnummer \textbf{beider} Abgabepartner aufs Deckblatt!
		\item Handschriftliche Bearbeitung (Papier oder Tablet mit Ausdruck)
		\item Lesbar schreiben, Rand freilassen!
		\item Übungsblatt \textbf{pünktlich} abgeben!
		\item Abgabe nun in Holzkiste \textbf{vor Raum 016}!
	\end{itemize}
\end{frame}

\mycomment{ % From year 2020, omitted because tutorial dropped because of holiday
	\begin{frame}{Zu Blatt \#1}
		
		Schnitt: \quad \thassedaniel{14.6}{13.7} / 17~P
		
		\begin{itemize}[<+->]
			\item Erstmal: Volle Punktzahl ist utopisch. Fehler macht man, um draus zu lernen – dazu sind ÜBs da
			\item \textbf{Einfachstes} Beispiel für $(A \setminus (B \setminus C)) \setminus D = (A \setminus B) \setminus (C \setminus D)$? \only<beamer:0>{\\ \impl $A = B = C = D = \emptyset$.}
			\item Für eine Menge $A$ ist $\setsize{A + x}$ nicht definiert! \\
			Passt mit euren \textbf{Operatoren} auf: \impl $\setsize{A \cup \set{x}}$
			\item Wenn keine Begründung/Beweis gefordert ist, müsst ihr keinen angeben. („Geben Sie an...“ / „Was ist...“)
			\item Nicht mehrere Alternativen angeben, unter denen ich mir die richtige heraussuchen soll. (That's \emph{your} job!)
			\mycomment{\item \textbf{TACKERN} und \textbf{DECKBLÄTTER}! Sonst \alert{\textbf{null Punkte}}!}
			\daniel{\item Bitte die Aufgabenblätter \textbf{nicht} mitabgeben (Deckblatt reicht), sonst wird mein Stapel so fett. \smiley}
		\end{itemize}
		\thasse{\FalseQuestionE{Ein Beispiel langt als Beweis.}{...bis ein anderes Beispiel herkommt, bei dem es kaputt geht... \\
				\impl eine Behauptung \textbf{widerlegen} geht mit nem Beispiel sehr gut!}}
	\end{frame}
}

\mycomment{  % Keine Zeit... Auch nicht sehr nötig diesmal...
	\begin{frame}[t]{Übungsblätter: Häufige Fehler \daniel{und anderer Kram}}
		
		\TrueQuestionE{Beweise fängt man mit der Behauptung an.}{}
		\FalseQuestionE{Wenn wir $A$ schreiben, ist das immer eine Menge, \\ und $f, g$ sind immer Funktionen.}{Das muss man \textbf{explizit} angeben.}
		\FalseQuestionE{Ein Beispiel langt als Beweis.}{...bis ein anderes Beispiel herkommt, bei dem es kaputt geht... \\
			\impl eine Behauptung \textbf{widerlegen} geht mit nem Beispiel sehr gut!}
		%Mit einem Beispiel kann man eine Behauptung widerlegen, es zeigt aber nicht die Allgemeingültigkeit}
		\FalseQuestionE{$\setsize{M} = M$ für $M \subseteq \N$.}{$\setsize{M}$ ist die \textbf{Anzahl} der Elemente in $M$. Nix mit „Betrag“.}
		
		\medskip
		
		\uncover<9->{Passt auf, was eure Variablen sind und welche Operationen ihr darauf anwenden könnt. Mengen kann man nicht mit $\bund$ und $\boder$ verknüpfen und auch keine \impl dazwischen tun.}
		
	\end{frame}
}


%\begin{frame}{Wahr oder falsch?}
%	\Socrative
%	\TrueQuestionE{$ \setsize{\emptyset} = 0$.}{}
%	\FalseQuestionE{$ \setsize{ \set{\set{}} } = 0$.}{$ \setsize{\set{\set{}}} = \setsize{\set{\emptyset}} = 1$.}
%	\TrueQuestionE{$ \setsize{\set{ 1, \set{2, 3}, 4, \set{5,6,7}}} = 4$.}{$\set{2, 3}$ und $\set{5,6,7}$ sind zwei einzelne Objekte!}
%	\TrueQuestionE{$ \{1, 2\} = \{2, 1\}$.}{}
%	\FalseQuestion{$ (1, 2) = (2, 1)$.}
%	\FalseQuestionE{$ \left(M\cup A\right)\setminus M = A $.}{für z.~B. $ M = A = \{1\}$}
%	%TODO Relationen-Fragen noch? Wenn's im ersten Tut schon vorkam!
%	\visible<12->{
%	\begin{block}{Bemerkungen}
%		Menge / Tupel: Klammern beachten! Einfache Beispiele helfen!
%	\end{block}
%	}
%\end{frame}

\begin{frame}[t]{Mengengleichheit: Aufgabe}
	\begin{itemize}
		\item<1-> Sei $ A, B, C $ beliebige Mengen. Zeigt, dass gilt 
		\begin{align*}
		A \cup (B \cap C) &=  (A \cup B) \cap (A \cup C)
		\end{align*}		 
		\only<2-5>{
			\item<2-5> \textbf{Richtung}: $ A \cup (B \cap C) \subseteq  (A \cup B) \cap (A \cup C) $ 
			\item<3-5> Sei $ x \in A \cup (B \cap C) $.
			\item<4-5> Fall 1: Ist $ x \in A$, so folgt $ x \in A \cup B $ und $ x\in A \cup C$ und damit $  x\in (A \cup B) \cap (A \cup C) $
			\item<5-5> Fall 2 : Ist $x\notin A$, so gilt $ x \in B \cap C$, also auch $ x \in B$ und $ x \in C$. Dann ist auch $ x \in A \cup B $ und $ x \in A \cup C$ und es folgt $  x\in (A \cup B) \cap (A \cup C) $
		}
		\only<6->{
			\item<6-> \textbf{Richtung}: $ A \cup (B \cap C) \supseteq  (A \cup B) \cap (A \cup C) $
			\item<7-> Wähle $ x \in (A \cup B) \cap (A \cup C) $. Dies bedeutet $x\in A \cup B $ und $ x \in A \cup C $. 
			\item<8-> Fall 1: $ x \in A $. Dann folgt $ x \in A \cup (B \cap C) $
			\item<9-> Fall 2: $ x \notin A$. Dann muss $ x \in B $ und $ x \in C $ gelten, denn sonst wäre $ x \notin (A \cup B) \cap (A \cup C) $. Somit $ x\in B \cap C$ und $ x \in A \cup (B \cap C) $. $\qed$
		}
	\end{itemize}
\end{frame}

\begin{frame}{Potenzmengen: Aufgabe}
	Gebt eine Abbildung $\phi \colon 2^{M} \functionto 2^{M}$ so an,
	dass für jedes $L \in 2^{M}$ und für jedes $w \in M$ gilt:
	\begin{equation*}
		w \in L \text{ genau dann, wenn } w \notin \phi(L).
	\end{equation*}
	
	\pause
	\begin{threealign}
		\phi \colon 2^{M} &\functionto& 2^{M},\\
		L &\mapsto& M \setminus L.
	\end{threealign}
\end{frame}

\input{../Bloecke/Funktionen.tex}

\begin{frame}[t]{Injektivität und Surjektivität}
	Was bedeutet nochmal injektiv und surjektiv?
	\begin{itemize}
		\item Eine Funktion $ f \from M \functionto N $ heißt \textit{injektiv} wenn gilt: \\
			\pause
			für alle $m_1,m_2 \in M$ gilt: Aus $ f(m_1)=f(m_2) $ folgt $ m_1 = m_2 $
		\pause
		\item Eine Funktion $ f \from M \functionto N $ heißt \textit{surjektiv} wenn gilt: \\
			\pause
			für alle $n \in N$ existiert ein $m \in M$ sodass $f(m)=n$ gilt
	\end{itemize}
	\pause
	\begin{block}{Wahr oder Falsch?}
		Sei $M$ eine beliebige endliche Menge und $f$ eine Abbildung $f \from M \functionto M$. \\
			\TrueQuestion{Wenn $f$ injektiv ist, dann ist $f$ auch surjektiv.}
			\TrueQuestion{Wenn $f$ surjektiv ist, dann ist $f$ auch injektiv.}
			\FalseQuestionE{Die beiden Aussagen gelten auch, wenn $M$ nicht endlich ist.}{Gegenbeispiele: \\ 
				$f \from \N_0 \functionto \N_0, n \mapsto 2 \* n $ \\ 
				$g \from \N_0 \functionto \N_0, n \mapsto \floor{\dfract n/2 }$}
	\end{block}
\end{frame}

\input{../Bloecke/Woerter.tex}


\begin{frame}	
	\begin{block}{Was ihr nun wissen solltet}
		\begin{itemize}
			\item Wie und wo ihr euer Übungsblatt \textbf{ab sofort} abgebt
			\item Welche Eigenschaften Relationen und Funktionen haben können
			\item Was Alphabete und Wörter sind
			\item Wie man mit Wörtern rechnet
		\end{itemize}
	\end{block}
	
	\begin{block}{Was nächstes Mal kommt}
		\begin{itemize}
			\item Sinnvollere Gebilde als \word{\thassedaniel{egnarts\sp si\sp efiL}{retsinnaL\sp nosrO}} mit \emph{formalen Sprachen}
			\item Aus Sage wird Logik: \emph{Aussagenlogik}
		\end{itemize}
	\end{block}
\end{frame}

\xkcdframe{1121}{Danke für eure Aufmerksamkeit! \smiley}{1.5}
\slideThanks

\end{document}