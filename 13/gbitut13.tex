%beamer

% Comment/uncomment this line to toggle handout mode
\newcommand{\handout}{}

\input{../framework/PraeambelTut.tex}

\morescalingdelimiters

\begin{document}
\starttut{13}

\section{Rückblick}

\begin{frame}{Zu Übungsblatt \#10}
	Bisheriger Schnitt: \quad 9.3 / 21~P

	\begin{itemize}[<+->]
		\item 12 von 23 TutandInnen haben etwas abgegeben
		\item Die Musterlösung findet ihr im \ILIAS unter Übungsblätter
		\item Korrekturen gibt es jetzt!
		\item Ihr habt alle pünktlich abgegeben :)
	\end{itemize}
\end{frame}

\begin{frame}{Zu Übungsblatt \#10}
	Die häufigsten Fehler:
	\begin{itemize}[<+->]
		\item[4a)] ``Geben Sie ein Schema an''
		\implitem Ein Schema muss \textbf{formal} korrekt angegeben werden
		\item[5a)] Beim Abarbeiten einer Zeichenfolge darauf achten, in welchen Zustand, wenn notwendiges Zeichen nicht als nächstes kommt
		\item[5b)] Ein Alphabet ist nach Definition eine nichtleere, \textbf{endliche} Menge
		\implitem Argumentation mit unendlich vielen Übergängen/Zuständen \textbf{falsch}
	\end{itemize}
\end{frame}

\begin{frame}{Zu Übungsblatt \#10 - Aufgabe 2}
	\includegraphics[width=\textwidth,height=\textheight,keepaspectratio]{UB10_2.png}
	\begin{itemize}[<+->]
		\item[b)] Wie kann die oben vorgestellte Konstruktion erweitert werden, sodass Produktionen, bei denen mehrere Terminalsymbole erzeugt werden, in A ebenfalls korrekt abgebildet werden? Beschreiben Sie die Ergänzung formal korrekt und zeichnen Sie als Beispiel das Resultat (ausschließlich) für die Produktion $(S \to \word{abab}A)$. Gehen Sie dabei von $T = \{a,b,c\}$ aus.
	\end{itemize}
\end{frame}


\framePrevEpisode

\begin{frame}{Algorithmen}
	\begin{block}{Definition}
		Ein Algorithmus ist...
		\begin{itemize}
			\item Eine endliche Beschreibung
			\item aus elementaren Anweisungen, 
			\item die deterministisch ($=$ ohne Zufall!) ausgeführt werden.\\
				{\small (Manchmal auch gemischt mit (Pseudo-)Zufallselementen)}
			\item Eine endliche Eingabe gibt endliche Ausgabe...
			\item in endlich vielen Schritten.
			\item Das funktioniert für beliebig große Eingaben und
			\item ist nachvollziehbar bzw. verständlich.
		\end{itemize}
	\end{block}
	\pause
	Woher wissen wir, ob ein Algorithmus korrekt ist? \pause \impl Hoare-Kalkül
\end{frame}

%\begin{frame}[t]{Wahr oder Falsch?}
%	\FalseQuestionE{Das (komplizierte) Master-Theorem kann man immer anwenden.}{ Nur bei rekursiven Algorithmen, bei denen das Problem in gleich große Teilprobleme aufgeteilt wird.}
%	\FalseQuestionE{Jeder Moore-Automat kann in einen Mealy-Automaten umgewandelt werden, der für jedes Wort die gleiche Ausgabe produziert.}{ Für das leere Wort kann ein Mealy-Automat niemals eine Ausgabe produzieren.}
%	\TrueQuestionE{Endliche Akzeptoren sind Moore-Automaten mit dem Ausgabealphabet $\{\word 0,\word 1\}$.}{}
%	\FalseQuestionE{Mit endlichen Automaten kann jede beliebige Sprache erkannt \\ werden.}{Tatsächlich ist die Menge der akzeptierbaren Sprachen sogar sehr eingeschränkt.}	
%\end{frame}

\input{../Bloecke/Hoare.tex}

\input{../Bloecke/GrossO.tex}

\begin{frame}	
	\begin{block}{Was ihr nun wissen solltet}
		\begin{itemize}
			\item Korrektheit von Algorithmen zeigen
			\item Asymptotisches Wachstum
		\end{itemize}
	\end{block}

	
	\begin{block}{Was nächstes Mal kommt}
		\begin{itemize}
			\item O-Kalkül - Rechnen
			\item Endlich: Graphen
			\item verschiedene Darstellungsformen von Graphen
		\end{itemize}
	\end{block}
\end{frame}

% TODO ?
%input{../Bloecke/StrukturelleInduktion}

%TODO replacement?
{\xkcdframe{1724}{Danke für eure Aufmerksamkeit! \smiley}{2.5}}
%\thasse{\lastframe{0.5}{0}{xkcd/proofs_1724.png}{https://www.xkcd.com/1724/}}

\slideThanks

\end{document}