%beamer

% Comment/uncomment this line to toggle handout mode
\newcommand{\handout}{}

\input{../framework/PraeambelTut.tex}

\morescalingdelimiters

\begin{document}
\starttut{4}

\section{Rückblick}

\begin{frame}{Zu Übungsblatt \#2}
	Schnitt: \quad 10,7 / 20~P

	\begin{itemize}[<+->]
		\item 19 von 24 TutandInnen haben etwas abgegeben. Weiter so!
		\item Die Musterlösung findet ihr im \ILIAS unter Übungsblätter
		\item Korrekturen gibt es jetzt!
		\item Ihr habt alle pünktlich abgegeben :)
		\item Das zweite Übungsblatt war die letzten Jahre immer schwierig
	\end{itemize}
\end{frame}

\begin{frame}{Zu Übungsblatt \#2}
	Die häufigsten Fehler:
	\begin{itemize}[<+->]
		\item Definiert alle Variablen, die ihr benutzt!
		\item Benutzt Variablen nicht doppelt \\
				z.B. bei: $n \mapsto \{n \elem \N_0 \mid n > 5\}$ ist unklar, was mit n gemeint ist
		\implitem \begin{enumerate}
			\item $n \mapsto \{x \elem \N_0 \mid x > 5\}$
			\item $n \mapsto \{x \elem \N_0 \mid x > 5, x > n\}$
		\end{enumerate}
		\item Einfach Pfeile ($\leftarrow, \rightarrow, \leftrightarrow$) benutzt man nur bei der \textit{formalen} Betrachtung von Aussagenlogik!
		\implitem Für Folgerungen und Äquivalenzen im Beweis benutzt man Doppelpfeile ($\Rightarrow, \Leftarrow, \Leftrightarrow$).
		\item Behauptungen müsst ihr als solche kennzeichnen
	\end{itemize}
\end{frame}

\begin{frame}{Zu Übungsblatt \#2 - Aufgabe 2.1}
	\begin{itemize}[<+->]
		\item M (von Übungsblatt 1) vs. $\mathcal{M}$ (von Übungsblatt 2)
		\implitem $\mathcal{M}$ enthällt \textbf{Mengen} und keine einzelnen Tupel
		\item Bitte aktualisierte Version des ÜBs nochmal anschauen, da Relation jetzt korrekt formuliert!
		\item e): Die Vereinigung ist \textbf{nicht} linkstotal
		\item Warum?
		\implitem Weil aus 2 korrekten Tupelmengen eine nicht korrekte Tupelmenge mit $\cup$ gebildet werden kann\\
			z.B. $\{(a,1)\}$ und $\{(a,2)\}$ würden mit der Vereinigung zu $\{(a,1), (a,2)\}$ führen
	\end{itemize}
\end{frame}

\begin{frame}{Zu Übungsblatt \#2 - Aufgabe 2.3}
	\begin{itemize}[<+->]
		\item b): evil(e) war eine Abbildung zu einer unendlichen \textbf{Menge von unendlichen Mengen}
		\implitem $evil(x) := \{y\elem\N_+\mid y > x\}$ ist \textbf{keine} korrekte Antwort
		\implitem Sei $x\elem \N_+$ , $G(x) := \{y\elem\N_+\mid y > x\}$ und \\
			$evil(x) := \{T \subset \N \mid $ T enthält genau ein Element aus $G(x)$ nicht$\}$
		\item c): Definiton der Funktion durch Potenzmenge gesucht, da nur so mehrere/keine Optionen abgedeckt werden können
	\end{itemize}
\end{frame}

\begin{frame}{Zu Übungsblatt \#2 - Wdh. Set-Comprehensions}
	\begin{itemize}[<+->]
		\item Grundsätzliche Form: $\set{... \vphantom{(} \Mid ......}$
		\item Linke Seite: irgendeine Variable, z.B. $n\elem\N$
		\item Rechte Seite: Eine Bedingung für die Variable, z.B. $n>2$
		\implitem Zusammen: $\set{n\elem\N \vphantom{(} \Mid n>2}$
		\item Mehrere Bedingungen durch \textbf{,} separiert
	\end{itemize}
\end{frame}

\framePrevEpisode

\begin{frame}{Kahoot!}
	\begin{itemize}[<+->]
		\item Kahoot! ist ein anonymes Online-Quiz
		\item Ihr bekommt Punkte für schnelles und richtiges raten
		\item Ich schalte das Quiz frei und ihr könnt über \url{https://kahoot.it} beitreten
		\item Das Kahoot! könnt ihr euch später nochmal unter diesem Link angucken: \\
			\url{https://create.kahoot.it/share/gbi-woche-4-einstieg/869ad0a5-9772-43a9-8a85-b03d44a99625}
	\end{itemize}
\end{frame}

%\begin{frame}{Rückblick}
%	\begin{itemize}
%		\item \textbf{Aussagen} sind Sätze, die wahr oder falsch sind
%		\item Wir können Aussagen mit \textbf{Konnektiven} zusammenbauen: \\
%		$\bund, \boder, \bnot, \bimp$
%		\item \textbf{Aussagevariablen} helfen dabei, konkrete Inhalte zu ignorieren 
%		\item \textbf{Interpretationen} liefern Wahrheitswerte zu Variablen
%		\item $val_I(\*)$ liefert Wahrheitswert für ganze Formel (rekursiv)
%	\end{itemize}
%\end{frame}

%\begin{frame}[t]{Wahr oder Falsch?}
% 	% Socrative: https://b.socrative.com/teacher/#import-quiz/31489145
% 	\Socrative
% 	
% 	\TrueQuestion{Dieser Satz ist eine Aussage.}
% 	\FalseQuestionE{$\alA \boder \alB = \alB \boder \alA$}{Das sind syntaktisch verschiedene AL-Formeln!}
% 	\TrueQuestion{$\alA \boder \alB \equiv \alB \boder \alA$}
% 	% \TrueQuestion{Der AL-Kalkül ist vollständig und korrekt.}  % Wurde noch gar nicht erklärt...!?
% 	\TrueQuestion{Es gibt unendlich viele Axiome im Aussagenkalkül.}
% 	\FalseQuestion{$\bleftBr \word G \bimp \word H \brightBr \; \vdash \;  \bleftBr \word H \bimp \word G \brightBr$}
% 	\FalseQuestion{Induktion kann man nur auf Zahlen anwenden.}
%\end{frame}


\input{../Bloecke/Induktion.tex}

\input{../Bloecke/FormaleSprachen.tex}

\mycomment{
	\section{Sprachen: Aufwärmen}
	
	
	\begin{frame}{Rückblick}
		\begin{itemize}
			\item \textbf{Alphabet} $A$ mit Zeichen, aus denen wir Wörter zusammenbauen
			\item Nicht immer haben all diese Wörter einen Sinn
			\item Wir definieren selbst, welche Wörter wir als korrekt ansehen und akzeptieren wollen.
			\item Eine solche Teilmenge aller möglichen Wörter nennen wir \textbf{formale Sprache}
		\end{itemize}
	\end{frame}
}

% TODO: Im letzten Jahr hat die Zeit nicht gereicht,
% daher diesen Inhalt hier eher streichen.
%\input{Sprache_gesucht_beamer.tex}

%\input{../Bloecke/FormaleSprachenP2.tex}

\begin{frame}{Ausblick: Klammerausdrücke}
	
	Was ist mit der Sprache aller gültigen Klammerausdrücke? Können wir die auch mit $\set{}$, $\*$, ${}^*$ und ${}^+$ angeben? \only<beamer:0>{\\ \emph{Spoiler: Nein, das geht nicht!}}\\[1em]
	\pause
	
	\begin{block}{}
		\Large
		\centering
		COMING SOON... \\[1em]
	\end{block}

	\begin{figure}[H]
		\centering
		\includegraphics[scale=0.7]{xkcd/(.png}
		\vspace{-7pt}
		\caption{ \texttt{\url{https://xkcd.com/859/}} }
	\end{figure}
\end{frame}

%\input{../Bloecke/Darstellung.tex}

\begin{frame}	
	\begin{block}{Was ihr nun wissen solltet}
		\begin{itemize}
			\item Wie man einen Beweis mittels vollständiger Induktion führt
			\item Was man nicht per vollständiger Induktion beweisen kann
			\item Was eine formale Sprache ist und warum das Konzept wichtig ist
			\item Wie man formale Sprachen formal angeben kann
			\item Wie man formale Sprachen konkateniert
		\end{itemize}
	\end{block}
	
	\begin{block}{Was nächstes Mal kommt}
		\begin{itemize}
			\item Wie man von Zahlendarstellungen zu Zahlen kommt...
			\item[] ... und wieder zurück
			\item Nicht immer so positiv: Negative Zahlen
			%\item Komprimierung: Huffmann-Codierungen
		\end{itemize}
	\end{block}
\end{frame}

% TODO 
\thassedaniel{
	\xkcdframevert{953}{Danke für eure Aufmerksamkeit! \smiley}{2.5}
}{
	\xkcdframe{1516}{Win by Induction}{2}
}
\slideThanks

\end{document}
