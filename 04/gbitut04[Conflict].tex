%beamer

% Comment/uncomment this line to toggle handout mode
%\newcommand{\handout}{}

\input{../framework/PraeambelTut.tex}

\morescalingdelimiters

\begin{document}
\starttut{4}

\section{Organisatorisches}

\begin{frame}{Zu Übungsblatt \#2}
	Schnitt: \quad 9,9 / 19~P

	\begin{itemize}[<+->]
		\item 19 von 23 Tutanden haben etwas abgegeben. Weiter so!
		\item Die Korrektur und die Musterlösung findet ihr im ILIAS-Aufgaben-Objekt
		\item Nur die Person, die das Übungsblatt abgegeben hat, bekommt die Rückmeldung \impl tauscht euch aus!
		\item Ihr habt alle pünktlich abgegeben :)
		\item Notation lief um einiges besser, aber immer noch nicht perfekt
	\end{itemize}
\end{frame}

\begin{frame}{Zu Übungsblatt \#2}
	Die häufigsten Fehler:
	\begin{itemize}[<+->]
		\item Definiert alle Variablen, die ihr benutzt! \\
			z.B. wenn ihr schreiben wollt: \textit{aus $w(i)=0$ folgt $w(i) \not = 1$}, dann müsst ihr vorher irgendwo $w$ und $i$ definiert haben.
		\item Eine \textbf{Annahme} beschreibt etwas, das \textit{eigentlich} nicht unbedingt gelten muss, bei dem ihr aber \textit{annehmt}, dass es gilt.
		\implitem Benutzen bei Widerspruchsbeweisen und Fallunterscheidungen (\textit{Fall 1: $x=0$: ..., Fall 2: $x \not = 0$: ...})
		\implitem Nicht als Überschrift für Definitionen, Voraussetzungen und Beweise benutzen!
		\item Einfach Pfeile ($\leftarrow, \rightarrow, \leftrightarrow$) benutzt man nur bei der \textit{formalen} Betrachtung von Aussagenlogik!
		\implitem Für Folgerungen und Äquivalenzen im Beweis benutzt man Doppelpfeile ($\Rightarrow, \Leftarrow, \Leftrightarrow$).
		\item Behauptungen müsst ihr als solche kennzeichnen
	\end{itemize}
\end{frame}

\begin{frame}{Zu Übungsblatt \#2}
	Die häufigsten Fehler:
	\begin{itemize}[<+->]
		\item Kommutativität und Assoziativität zeigen
		\item Kommutativität und Assoziativität widerlegen
		\item Wir machen das jetzt nochmal zusammen
	\end{itemize}
\end{frame}

\framePrevEpisode

\begin{frame}{Kahoot!}
	\begin{itemize}[<+->]
		\item Kahoot! ist ein anonymes Online-Quiz
		\item Ihr bekommt Punkte für schnelles und richtiges raten
		\item Ich schalte das Quiz frei und ihr könnt über \url{https://kahoot.it} beitreten
		\item Das Kahoot! könnt ihr euch später nochmal unter diesem Link angucken: \\
			\url{https://create.kahoot.it/share/gbi-woche-3-einstieg/c0638af4-aaa2-4892-8916-a2eab442312d}
	\end{itemize}
\end{frame}

%\begin{frame}{Rückblick}
%	\begin{itemize}
%		\item \textbf{Aussagen} sind Sätze, die wahr oder falsch sind
%		\item Wir können Aussagen mit \textbf{Konnektiven} zusammenbauen: \\
%		$\bund, \boder, \bnot, \bimp$
%		\item \textbf{Aussagevariablen} helfen dabei, konkrete Inhalte zu ignorieren 
%		\item \textbf{Interpretationen} liefern Wahrheitswerte zu Variablen
%		\item $val_I(\*)$ liefert Wahrheitswert für ganze Formel (rekursiv)
%	\end{itemize}
%\end{frame}

%\begin{frame}[t]{Wahr oder Falsch?}
 %	% Socrative: https://b.socrative.com/teacher/#import-quiz/31489145
 %	\Socrative
 %	
 %	\TrueQuestion{Dieser Satz ist eine Aussage.}
 %	\FalseQuestionE{$\alA \boder \alB = \alB \boder \alA$}{Das sind syntaktisch verschiedene AL-Formeln!}
 %	\TrueQuestion{$\alA \boder \alB \equiv \alB \boder \alA$}
 %	% \TrueQuestion{Der AL-Kalkül ist vollständig und korrekt.}  % Wurde noch gar nicht erklärt...!?
 %	\TrueQuestion{Es gibt unendlich viele Axiome im Aussagenkalkül.}
 %	\FalseQuestion{$\bleftBr \word G \bimp \word H \brightBr \; \vdash \;  \bleftBr \word H \bimp \word G \brightBr$}
 %	\FalseQuestion{Induktion kann man nur auf Zahlen anwenden.}
%\end{frame}


\input{../Bloecke/Aussagenlogik.tex}

\input{../Bloecke/Aussagenlogik2}

\input{../Bloecke/FormaleSprachen.tex}

\mycomment{
	\section{Sprachen: Aufwärmen}
	
	
	\begin{frame}{Rückblick}
		\begin{itemize}
			\item \textbf{Alphabet} $A$ mit Zeichen, aus denen wir Wörter zusammenbauen
			\item Nicht immer haben all diese Wörter einen Sinn
			\item Wir definieren selbst, welche Wörter wir als korrekt ansehen und akzeptieren wollen.
			\item Eine solche Teilmenge aller möglichen Wörter nennen wir \textbf{formale Sprache}
		\end{itemize}
	\end{frame}
}




% TODO: Im letzten Jahr hat die Zeit nicht gereicht,
% daher diesen Inhalt hier eher streichen.
%\input{Sprache_gesucht_beamer.tex}

\input{../Bloecke/FormaleSprachenP2.tex}

\begin{frame}{Ausblick: Klammerausdrücke}
	
	Was ist mit der Sprache aller gültigen Klammerausdrücke? Können wir die auch mit $\set{}$, $\*$, ${}^*$ und ${}^+$ angeben? \only<beamer:0>{\\ \emph{Spoiler: Nein, das geht nicht!}}\\[1em]
	\pause
	
	\begin{block}{}
		\Large
		\centering
		COMING SOON... \\[1em]
	\end{block}

	\begin{figure}[H]
		\centering
		\includegraphics[scale=0.7]{xkcd/(.png}
		\vspace{-7pt}
		\caption{ \texttt{\url{https://xkcd.com/859/}} }
	\end{figure}
\end{frame}

%\input{../Bloecke/Darstellung.tex}

\begin{frame}	
	\begin{block}{Was ihr nun wissen solltet}
		\begin{itemize}
			\item Was eine formale Sprache ist und warum das Konzept wichtig ist
			\item Wie man einfache formale Sprachen formal angeben kann
			\item Einfache Operationen auf formalen Sprachen
			\item Wie man formale Sprachen angeben kann
			\item Wie man Beweise mit formalen Sprachen führt
		\end{itemize}
	\end{block}
	
	\begin{block}{Was nächstes Mal kommt}
		\begin{itemize}
			\item Wie man von Zahlendarstellungen zu Zahlen kommt...
			\item[] ... und wieder zurück
			\item Nicht immer so positiv: Negative Zahlen
			%\item Komprimierung: Huffmann-Codierungen
		\end{itemize}
	\end{block}
\end{frame}

% TODO 
\thassedaniel{
	\xkcdframevert{953}{Danke für eure Aufmerksamkeit! \smiley}{2.5}
}{
	\xkcdframe{1516}{Win by Induction}{2}
}
\slideThanks

\end{document}
