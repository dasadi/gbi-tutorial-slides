%beamer

% TODO:
% - VL-Folien einarbeiten
% - Weniger f_* etc. Theorie, mehr Aufgaben zwischendrin und zu Akzeptoren
% - Bessere Hinleitung zu Akzeptoren

% Comment/uncomment this line to toggle handout mode
\newcommand{\handout}{}

\input{../framework/PraeambelTut.tex}

\morescalingdelimiters

\begin{document}
\starttut{12}

\section{Rückblick}

\begin{frame}{Zu Übungsblatt \#9}
	Schnitt: \quad 14.5 / 25~P

	\begin{itemize}[<+->]
		\item 14 von 23 TutandInnen haben etwas abgegeben
		\item Die Musterlösung findet ihr im \ILIAS unter Übungsblätter
		\item Korrekturen gibt es jetzt!
		\item Ihr habt alle pünktlich abgegeben :)
	\end{itemize}
\end{frame}

\begin{frame}{Zu Übungsblatt \#9}
	Die häufigsten Fehler:
	\begin{itemize}[<+->]
		\item Allgemein: Wenn in der Aufgabe \textit{Definieren Sie} steht, ist eine \textbf{korrekte mathematische Definition} gefordert
		\item[1a)] Viele haben z.B. $\{ S \to S \word{aab}\}$ vergessen 
		\implitem $\word{aaaabbbbaaaa}$ sonst nicht ableitbar
		\item[1b)] Tippfehler in der Aufgabe \impl Lösung ist trivial. Jedoch trotzdem volle Punkte
		\item Tippfehler ist korrigiert zur Übung für die Klausur
	\end{itemize}
\end{frame}

\begin{frame}{Zu Übungsblatt \#9}
	Die häufigsten Fehler:
	\begin{itemize}[<+->]
		\item[5c)] Bei PL-Formeln: Komplett ausschreiben, keine Abkürzungen wie: $H = \bleftBr \plexist \plz \plall \ply G \brightBr \alimpl F$ 
		\item[6c)] Sichergehen, dass 2 verschiedene Schurken/Drohnen ($\plx , \ply$) nicht die gleichen sind: $\alnot \bleftBr \plx \pleq \ply \brightBr$ 
		\item Jeder Schurke kennt einen anderen Schurken, der (unwissentlich) von einer
		B.I.R.D.-Drohne ausgespäht wird.
	\end{itemize}
\end{frame}

\begin{frame}{Klausurvorbereitung}
	1. Materialien anschauen
	\begin{itemize}
		\item Folien der Vorlesung
		\item Skript der Vorlesung
		\item Folien des Tutoriums
	\end{itemize}

	\pause
	2. Zusammenfassung oder Karteikarten schreiben \\

	\pause
	3. Gaaanz viele Aufgaben rechnen
	\begin{itemize}
		\item Übungsblätter nochmal bearbeiten
		\item Übungsblätter aus vergangenen Jahren bearbeiten
		\item \textbf{$\rightarrow$ Altklausuren $\leftarrow$} rechnen
		\item Aufgabenarchiv: \url{http://gbi.ira.uka.de/archiv/}
	\end{itemize}
\end{frame}

\mycomment{
	\begin{frame}{Schwarzes Brett}
		\begin{itemize}
			\item Es wird insgesamt \textbf{6~Blätter} geben \impl Gesamtpunkte $=$ 205~P
			\implitem Wer \textbf{$\geq$ 102.5~P} hat, \textbf{hat den Schein garantiert}. \\
			{\small (Wer etwas weniger hat: vielleicht auch, keine Ahnung. Stay tuned.)}
			\item Es wird ein \textbf{Bonusübungsblatt} geben, auf dem ihr zusätzl. Punkte sammeln könnt (aber nicht müsst). \\
			Das wird dann (nach Ende der VL-Zeit) beim Übungsleiter Zenkel hinterlegt, sobald korrigiert.
		\end{itemize}
	\end{frame}
}

\framePrevEpisode

\begin{frame}{Rückblick: Turingmaschine}
	\begin{Definition}
		Eine Turingmaschine $T$ ist definiert als $$ T = (Z, z_0 , X, f,g, m)$$
		\begin{itemize}[<+->]
			\item $Z \quad$ Zustandsmenge 
			\item $z_0\in Z \quad$ Startzustand
			\item $X \quad$ Bandalphabet mit $\square \in X$
			\item $f:Z\times X \dashrightarrow Z \quad$ Übergangsfunktion
			\item $g:Z\times X\dashrightarrow X \quad$ Ausgabefunktion  \quad (\textbf{genau ein Zeichen} als Ausgabe!)
			\item $m:Z\times X \dashrightarrow \{\text{L},\text{0},\text{R}\} \quad$ Bewegungsfunktion
		\end{itemize}
		\pause
		Alle Funktionen können auch nur partiell definiert ($\dashrightarrow$) sein. \\
	\end{Definition}
\end{frame}

\begin{frame}{Aufgabe}
	Entwerft eine Turingmaschine, die die Sprache $ \{\word 0^k\word 1^k \mid k\in \N_0 \} $ akzeptiert.
	% TODO Lösung
\end{frame}

\input{../Bloecke/TMKomplexitaet.tex}

\input{../Bloecke/Hoare.tex}


% next time \input{../Bloecke/Akzeptoren}

% ---------

\mycomment{
% WiSe 10/11 Aufgabe 6 c 
\begin{frame}{Übung: Akzeptoren}
	Die Sprache $L\subseteq \{\word a,\word b\}^\ast $ sei definiert als die Menge aller Wörter $w$, die folgende Bedingungen erfüllen:
	\begin{align*}
	N_{\word b}(w) &> N_{\word a}(w)\\ 
	\forall v_1,v_2 \in \{\word a,\word b\}^\ast : \qquad w &\neq v_1 \word{bb} v_2 
	\end{align*}
	
	Gebt einen endlichen Akzeptor an, der $L$ erkennt. \\
	
	\bigskip
	\pause
	\begin{block}{Tipps}
		\begin{itemize}[<+->]
			\item Die zweite Bedingung bedeutet: Das Wort darf nirgends zwei \word b hintereinander enthalten.
			\item Was passiert, wenn das Wort mit einem \word a beginnt?\\
			Kann das Wort noch akzeptiert werden?
		\end{itemize}
	\end{block}
\end{frame}

\begin{frame}{Übung: Akzeptoren: Lösung}
	\begin{figure}
		\centering
		\includegraphics[width=0.7\linewidth]{automaten/Loesung2.pdf}
	\end{figure}
\end{frame}
}

\begin{frame}	
	\begin{block}{Was ihr nun wissen solltet}
		\begin{itemize}
			\item Wie man die Laufzeit von TMs beurteilt
			\item Was \textbf{P} und \textbf{PSPACE} sind
		\end{itemize}
	\end{block}
	
	\begin{block}{Was nächstes Mal kommt}
		\begin{itemize}
			\item Wie man die Korrektheit von Algorithmen beweist
			\item Wie man die Effizienz von Algorithmen beurteilt
			\item Endlich: Graphen
			%\item Ein Regelwerk für einen Ausdruck -- Reguläre Ausdrücke
			%\item Rechtslineare Grammatiken
			%\item Turingmaschinen -- mächtiger wird es nicht mehr!
		\end{itemize}
	\end{block}
\end{frame}


\xkcdframe{287}{Danke für eure Aufmerksamkeit! \smiley}{2.5}
%\lastframe{0.6}{30}{xkcd/automation_1319.png}{http://www.xkcd.com/1319}
%\xkcdframe{0.5}{30}{xkcd/houston_1438.png}{http://www.xkcd.com/1438}{Oh, hi mom. No, nothing important, just work.}



\slideThanks

\end{document}