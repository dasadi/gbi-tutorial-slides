%beamer

% Start with video if you like it and can get the sound working well:
% https://www.youtube.com/watch?v=Dy0hJWltsyE

% Comment/uncomment this line to toggle handout mode
%\newcommand{\handout}{}

\input{../framework/PraeambelTut.tex}

\begin{document}
\starttut{1}

\section{\thassedaniel{Organisatorisches}{Orga-Kram}}

\aboutMeFrame

\begin{frame}[t]{Tutorium (Def.)}
	\begin{itemize}[<+->]
		\item Inhalte der VL verstehen und anwenden
		\implitem Den Stoff etwas weniger formal behandeln
		\item \textbf{Eure Fragen} klären!
		\item Beispiele, Aufgaben (da kommt \textbf{ihr} ins Spiel! \smiley)
		\item Besprechung häufiger Fehler auf den Übungsblättern
	\end{itemize}
	% No time. Kurz mündlich.
	% \NoQuestionE{Dürft ihr mich siezen?}{Ihr werdet sonst erschossen. \smiley \hspace{.5\linewidth}\mbox{} \\ \impl Sagt einfach \textbf{Du}.} 
\end{frame}

\begin{frame}[t]{Sicherheitshinweise}
	\begin{itemize}[<+->]
		\item Das Tutorium ersetzt die Vorlesung \textbf{nicht!} \\
			Wir werden manche Formalien auslassen, diese können aber dennoch für die Klausur wichtig sein!
		\item This is a test - Ihr seid die Versuchskaninchen für mein erstes Tutorium 
		\item Wenn ich / meine Folien Blödsinn reden: \\
			Schreien! / Mail an mich! \\
			Verbindlich \textbf{nur} Inhalt aus der VL/Übung! 
		\item \textbf{Feedback erwünscht}: \\
			Zu schnell/langsam/leicht/schwer/viel/wenig/... ? \impl \textbf{Sagt es mir!}\\
			Jederzeit per Mail oder direkt im Tut.
	\end{itemize}
\end{frame}

\thasse{   %  Timing issues. Showing that XKCD takes time (esp. reading everything!)
	\only<beamer:0>{
	\begin{frame}{Feedback}
		 Feedback bitte so konkret wie möglich, mit \enquote{Ich gebe dem Tut 4 Sterne.} kann ich nicht viel anfangen.
		
		\begin{figure}
			\centering 
			\includegraphics[scale=0.5]{xkcd/tornadoguard_937}
		\end{figure}
		
	\end{frame}
	}
}

\begin{frame}[t]{Präsenztutorium}
	\begin{itemize}
		\item \textbf{3G-Pflicht} in jedem Präsenz-Tutorium \\
			Gültigkeit: \quad Schnelltest: 24~h \quad PCR-Test: 48~h
		\item Bei jedem Tutoriumstermin Kontrolle 
		\implitem Bitte ein paar Minuten vorher da sein
		\item Vor dem Betreten des Raunms KONKIT-QR-Code scannen
	\end{itemize}
\end{frame}


\begin{frame}[t]{GBI bestehen}
	GBI habt ihr bestanden, wenn ihr Übungsschein* und Klausur bestanden habt:
	\begin{itemize}
		\item \textbf{Übungsschein}: im Laufe des Semesters \\
			  bestanden, wenn ihr min. je 50~\% der Gesamtpunkte in der ersten und zweiten Hälfte der Übungsblätter erreicht \\

		\item \textbf{Klausur}: \Klausurtermin \\
			  schriftlich, keine Hilfsmittel, sicher bestanden ab 50~\% der Punkte
	\end{itemize}
	\pause
	\DependsQuestionE{* Muss ich den Schein machen?}{
		Infos, InWis, Info-Lehramt: \quad  Ja \quad - Alle Anderen: \quad Vielleicht \\
		{\small Das alles \textbf{ohne Gewähr}! Eure Fakultäten haben am Ende Recht.} \\
		\smallskip
		\Impl Trotzdem: ÜBs machen lohnt sich. \smiley \\
	}
	\pause
	GBI ist Teil der \textbf{Orientierungsprüfungen} für Informatiker und Info-Lehramt!
	\begin{itemize}
		\implitem Übungsschein im ersten Semester \textbf{versuchen} (da im Zweiten nicht angeboten), im Dritten bestehen! 
		\implitem Die Klausur könnt ihr auch ohne Schein schreiben und müsst ihr \textbf{spätestens} im zweiten Semester versuchen und im Dritten bestehen!
	\end{itemize}
\end{frame}

\begin{frame}{Übungsblätter}
	Im laufenden Semester werden wöchentlich insgesamt 12 Übungsblätter ausgegeben:
	\begin{itemize}
		\item \hstretchto{\textbf{Rückgabe}:}{\textbf{Ausgabe}:} Freitags, üblicherweise um 12h im \ILIAS \\
		\item \hstretchto{\textbf{Rückgabe}:}{\textbf{Abgabe}:} Spätestens Freitags (7 Tage später), 12:30 Uhr in den GBI-Briefkasten (1.UG Infobau)*
		\item \textbf{Rückgabe}: Montags zum Tutorium
	\end{itemize}
	\pause	
	* Was, wenn der Briefkasten kaputt ist? \\
	\impl Alternativ bei Marko Kleine Büning (Raum 016) abgeben \\
\end{frame}

\begin{frame}{Übungsblätter - Bearbeitungshinweise}
	\begin{itemize}
		\item Bearbeitungen müssen \textbf{handschriftlich} und inklusive \textbf{ausgefülltem Deckblatt} auf Papier abgegeben werden \\
		\pause
		\item Bei verspäteten Abgaben wird das Blatt \textbf{nicht gewertet} und eine Korrektur kann nicht garantiert werden
		\pause
		\item Die ersten sechs Blätter dürft ihr zu zweit bearbeiten, den Rest nur alleine
		\implitem \textbf{Keine} Plagiate, \textbf{keine} Gruppenabgabe (sonst \textbf{Übungsschein weg}!)
		\pause
		\item Aufgabenblätter bitte \textbf{nicht} mit abgeben (spart Papier)
		\pause
		\item Schreiben in Schwarz oder Blau (oder andere dunkle Farbe $\notin \{\text{Rot}, \text{Grün}, \text{Weiß}\}$ )	
		\item Links und rechts bitte einen Rand frei lassen. \\
			  Dann kann ich's auch korrigieren. \smiley
		\item Ordnung behalten: \\
			  \textbf{Welche Aufgabe} ist das gerade?\quad \impl Deutlich machen!
		\item Was ich nicht lesen kann, gibt keine Punkte! \\
			  \small Was ich schwer lesen kann, u.~U. auch... 
	\end{itemize}
\end{frame}



\begin{frame}{Wo bekomme ich Hilfe?}	
	\begin{block}{Fragen}
		\begin{itemize}
			\item \textbf{Hier} im Tut!
			\item Ins Forum im \ILIAS. (Dann haben alle was davon. \smiley) \\
			\pause
			\item Organisatorischer Spezialkram? \\
				  \impl an Kleine Büning/Sinz \\
						(\mailto{marko.kleinebuening@kit.edu} / \mailto{carsten.sinz@kit.edu}) \\
				  Bei Orga-Problemen aber \textbf{Name} und \textbf{Matrikelnummer}  mitangeben! \\
				  Fachliches bitte nicht per Mail, lieber \impl Sprechstunde, \ILIAS...!
			\item Tut-spezifisches an \textbf{mich} (\mailto{\mymail})
		\end{itemize}
	\end{block}
	\pause
	\begin{block}{Content}
		\begin{itemize}
			\item Vorlesungs-, Übungs- und Tutoriums-Folien, Skript, Übungsblätter, Lösungen, ...: im \ILIAS.
			\item Alte Klausuren und Übungsblätter gibt's im Archiv: \url{http://gbi.ira.uka.de}
		\end{itemize}
	\end{block}
\end{frame}

\begin{frame}{Wo bekomme ich Hilfe?}	
	\begin{block}{Und wenn's um persönliche Themen geht?}
		\begin{itemize}
			\item Allgemeine Studienberatung: \url{https://www.sle.kit.edu/imstudium/zib.php}
			\item Informatik Studiengangservice: \url{https://www.informatik.kit.edu/iss.php}
			\item Studium und Behinderung: \url{https://www.studiumundbehinderung.kit.edu/}
			\item Psychotherapeutische Beratung: \url{https://www.sw-ka.de/en/beratung/}
			\item Telefonseelsorge: \url{https://www.telefonseelsorge-karlsruhe.de/}
			\item Anonymes Zuhörtelefon: \url{https://nightline-karlsruhe.de/}
		\end{itemize}
	\end{block}
\end{frame}

% No time. Seriously, no.
%
% Pretty irrelevant. Don't spend a lot of time here or leave it out completely.
%

\section{Signale, Mitteilungen, Informationen, Daten}

\morescalingdelimiters

\begin{frame}{Signale, Mitteilungen, Informationen, Daten}
	\begin{block}{Signal, Mitteilung und Medium}
		Ein Signal überträgt eine Mitteilung. Mitteilungen können in einem Medium gespeichert werden.
	\end{block}
	\pause
	\begin{block}{Nachricht}
		Mitteilung, bei der vom Medium und den Einzelheiten der Signale abstrahiert wird.
	\end{block}
	\pause
	\begin{block}{Information}
		Durch \textit{Interpretation} in einem \textit{Bezugssystem} kann man einer Nachricht eine Information (Bedeutung) zuweisen. \\
		\impl kontextabhängig!
	\end{block}

% Not relevent as of WS 20/21
%	\pause
%	\begin{block}{Informationsgehalt einer Nachricht}
%		Bei gleicher Wahrscheinlichkeit möglicher Nachrichten: \\ \quad In etwa \quad $\log(\text{Anzahl möglicher Nachrichten})$ \\
%		Versch. Einheiten: \\ \vspace{-.2\baselineskip}
%		\begin{itemize}
%			\item Naturalis: $\log_e$ 
%			\item Hartley: $\log_{10}$ 
%			\item Shannon: $\log_2$
%		\end{itemize}
%		\impl Mehr dazu in TGI
%	\end{block}	
\end{frame}


\input{../Bloecke/Mengen.tex}

\input{../Bloecke/Relationen1a.tex}

\section{Eigenschaften von Relationen}

\begin{frame}{Eigenschaften von Relationen}
	\begin{block}{Totalität}
		Eine Relation $R \subseteq A \times B$ heißt \textbf{linkstotal}, wenn es für \emph{jedes} $a \in A$ ein zugehöriges $b \in B$ gibt, sodass $$(a,b) \in R.$$ \textbf{Rechtstotal} analog (für jedes $b$ ein $a$).
	\end{block}
	
	\pause
	\begin{block}{Eindeutigkeit}
		Eine Relation $R \subseteq A \times B$ heißt \textbf{linkseindeutig}, wenn für jedes $b \in B$ und $a_1, a_2 \in A$ gilt: $$\text{Wenn } \quad (a_1,b) \in R \text{ und } (a_2,b) \in R, \quad \text{dann} \quad a_1 = a_2.$$ \\
		\textbf{Rechtseindeutig} analog $\left(\text{wenn } (a, b_1) \text{ und } (a, b_2) \in R, \text{dann } b_1 = b_2 \right)$.
	\end{block}
\end{frame}

\subsection{Aufgabe 1}
\begin{frame}{Aufgabe 1 Teil 1} % (WS 2010)
	Es sei $A$ die Menge aller Kinobesucher in einer Vorstellung und $B$ die Menge aller Sitzplätze. Die Relation $R$ ordnet den Kinobesuchern die Sitzplätze zu:
	$$ R \subseteq A \times B$$
	Was bedeutet es im Kino, wenn $R$ linkstotal, linkseindeutig, rechtstotal, rechtseindeutig ist?
	\smallskip
		
	\pause
	\begin{tabular}{@{\hspace{-3pt}}r@{\ \ }l}
		\emph{linkstotal}: & jedem Kinobesucher wird mindestens ein Sitzplatz zugeteilt \\
		\pause
		\emph{linkseindeutig}: & jeder Sitzplatz ist von höchstens einem Kinobesucher belegt \\
		\pause
		\emph{rechtstotal}: & jeder Sitzplatz ist von mindestens einem Kinobesucher belegt \\ 
		\pause
		\emph{rechtseindeutig}: & kein Kinobesucher belegt mehr als einen Sitzplatz
	\end{tabular}
\end{frame}

\begin{frame}
	\begin{block}{Was ihr nun wissen solltet}
		\begin{itemize}
			\item wie man mit Mengen umgeht
			\item wie man mit noch mehr Mengen umgeht
			\item was Paare sind
			\item wie man mehr als zwei Elemente geordnet zusammenbringt (Tupel)
			\item wie man mit Relationen Chaos in die Ordnung bringt
		\end{itemize}
	\end{block}

	\begin{block}{Was nächstes Mal kommt}
		\begin{itemize}
			\item wie Funktionen funktionieren
			\item \textcolor{white}{.}\!% sHAME ON YOU, LATEX!!! I FUCKING HATE YOU!
				  \word{ada}: eure ersten Wörter als Informatiker 
		\end{itemize}
	\end{block}
\end{frame}

\only<handout:0>{\slideThanks}

%% Letzte Seite 
\xkcdframevert{2042}{Danke für eure Aufmerksamkeit! \smiley}{2.5}

\only<beamer:0>{\slideThanks}

\end{document}
