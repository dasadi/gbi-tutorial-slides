%beamer

% Start with video if you like it and can get the sound working well:
% https://www.youtube.com/watch?v=Dy0hJWltsyE

% Comment/uncomment this line to toggle handout mode
\newcommand{\handout}{}

%% Beamer-Klasse im korrekten Modus
\ifdefined \handout
\documentclass[handout]{beamer} % Handout mode
\else
\documentclass{beamer}
\fi

%% UTF-8-Encoding
\usepackage[utf8]{inputenc}

\input{../framework/gbi-macros}
\usepackage[blue]{../framework/thwregex}
\usepackage{environ}
\usepackage{bm}
\usepackage{calc}
\usepackage{varwidth}
\usepackage{wasysym}
\usepackage{mathtools}


% Das ist der KIT-Stil
%\usepackage{../TutTexbib/beamerthemekit}
\usepackage[deutsch,titlepage0]{../framework/KIT/beamerthemeKITmod}
\TitleImage[width=\titleimagewd]{../figures/titlepage.jpg}
%\usetheme[deutsch,titlepage0]{KIT}

% Include PDFs
\usepackage{pdfpages}

% Libertine font (Original GBI font)
\usepackage{libertine}
%\renewcommand*\familydefault{\sfdefault}  %% Only if the base font of the document is to be sans serif

% Nicer math symbols
\usepackage{eulervm}
%\usepackage{mathpazo}
\renewcommand\ttdefault{cmtt} % Computer Modern typewriter font, see lecture slides.

\usepackage{csquotes}

%%%%%%

%% Schönere Schriften
\usepackage[TS1,T1]{fontenc}

%% Bibliothek für Graphiken
\usepackage{graphicx}

%% der wird sowieso in jeder Datei gesetzt
\graphicspath{{../figures/}}

%% Anzeigetiefe für Inhaltsverzeichnis: 1 Stufe
\setcounter{tocdepth}{1}

%% Hyperlinks
\usepackage{hyperref}
% I don't know why, but this works and only includes sections and NOT subsections in the pdf-bookmarks.
\hypersetup{bookmarksdepth=subsection} 

%\usepackage{lmodern}
\usepackage{colortbl}
\usepackage[absolute,overlay]{textpos}
\usepackage{listings}
\usepackage{forloop}
%\usepackage{algorithmic} % PseudoCode package 

\usepackage{tikz}
\usetikzlibrary{matrix}
\usetikzlibrary{arrows.meta}
\usetikzlibrary{automata}
\usetikzlibrary{tikzmark}
\usetikzlibrary{positioning}

% Why has no-one come up with this yet? I mean, seriously. -.-
\tikzstyle{loop below right} = [loop, out=-60,in=-30, looseness=7]
\tikzstyle{loop below left} = [loop, out=-150,in=-120, looseness=7]
\tikzstyle{loop above right} = [loop, out=60,in=30, looseness=7]
\tikzstyle{loop above left} = [loop, out=150,in=120, looseness=7]
\tikzstyle{loop right below} = [loop below right]
\tikzstyle{loop left below} = [loop below left]
\tikzstyle{loop right above} = [loop above right]
\tikzstyle{loop left above} = [loop above left]

% Needed for gbi-macros
\usepackage{xspace}

%%%%%%

%% Verbatim
\usepackage{moreverb}

%%%%%%%%%%%%%%%%%%%%%%%%%%%%%%%%%%%% Copy end

%% Tabellen
\usepackage{array}
\usepackage{multicol}
\usepackage{hhline}

%% Bibliotheken für viele mathematische Symbole
\usepackage{amsmath, amsfonts, amssymb}

%% Deutsche Silbentrennung und Beschriftungen
\usepackage[ngerman]{babel}

\usepackage{kbordermatrix}

% kbordermatrix settings
\renewcommand{\kbldelim}{(} % Left delimiter
\renewcommand{\kbrdelim}{)} % Right delimiter

\input{../config.tex}



% define custom \handout command flag if handout mode is toggled  #DirtyAsHellButWell...
\only<beamer:0>{\def\handout{}} %beamer:0 == handout mode

\newcommand{\R}{\mathbb{R}}
\newcommand{\N}{\mathbb{N}}
\newcommand{\Z}{\mathbb{Z}}
\newcommand{\Q}{\mathbb{Q}}
\newcommand{\BB}{\mathbb{B}}
\newcommand{\C}{\mathbb{C}}
\newcommand{\K}{\mathbb{K}}
\newcommand{\G}{\mathbb{G}}
\newcommand{\nullel}{\mathcal{O}}
\newcommand{\einsel}{\mathds{1}}
\newcommand{\Pot}{\mathcal{P}}
\renewcommand{\O}{\text{O}}

\def\word#1{\hbox{\textcolor{blue}{\texttt{#1}}}}
\let\literal\word
\def\mword#1{\hbox{\textcolor{blue}{$\mathtt{#1}$}}}  % math word
\def\sp{\scalebox{1}[.5]{\textvisiblespace}}
\def\wordsp{\word{\sp}}

%\newcommand{\literal}[1]{\textcolor{blue}{\texttt{#1}}}
\newcommand{\realTilde}{\textasciitilde \ }
\newcommand{\setsize}[1]{\ensuremath{\left\lvert #1 \right\rvert}}
\newcommand{\size}[1]{\setsize{#1}}  % Shame on you, TeXStudio...
\newcommand{\set}[1]{\left\{#1\right\}}
\newcommand{\tuple}[1]{\left(#1\right)}
\newcommand{\normalvar}[1]{\text{$#1$}}

% Modified by DJ
\let\oldemptyset\emptyset
\let\emptyset\varnothing % proper emptyset

\newcommand{\boder}{\ensuremath{\mathbin{\textcolor{blue}{\vee}}}\xspace}
\newcommand{\bund}{\ensuremath{\mathbin{\textcolor{blue}{\wedge}}}\xspace}
\newcommand{\bimp}{\ensuremath{\mathrel{\textcolor{blue}{\to}}}\xspace}
\newcommand{\bgdw}{\ensuremath{\mathrel{\textcolor{blue}{\leftrightarrow}}}\xspace}
\newcommand{\bnot}{\ensuremath{\textcolor{blue}{\neg}}\xspace}
\newcommand{\bone}{\ensuremath{\textcolor{blue}{1}}\text{}}
\newcommand{\bzero}{\ensuremath{\textcolor{blue}{0}}\text{}}
\newcommand{\bleftBr}{\ensuremath{\textcolor{blue}{\texttt{(}}}\text{}}
\newcommand{\brightBr}{\ensuremath{\textcolor{blue}{\texttt{)}}}\text{}}

% Fix of \b... commands:

\renewcommand{\boder}{\alor}
\renewcommand{\bund}{\aland}
\renewcommand{\bimp}{\alimpl}
\renewcommand{\bgdw}{\aleqv}
\renewcommand{\bnot}{\alnot}
\renewcommand{\bleftBr}{\alka}
\renewcommand{\brightBr}{\alkz}
\newcommand{\alA}{\word A}
\newcommand{\alB}{\word B}
\newcommand{\alC}{\word C}

\newcommand{\plB}{\plfoo{B}}
\newcommand{\plE}{\plfoo{E}}

\newcommand{\summe}[2]{\sum\limits_{#1}^{#2}}
\newcommand{\limes}[1]{\lim\limits_{#1}}

%\newcommand{\numpp}{\advance \value{weeknum} by -2 \theweeknum \advance \value{weeknum} by 2}
%\newcommand{\nump}{\advance \value{weeknum} by -1 \theweeknum \advance \value{weeknum} by 1}

\newcommand{\mycomment}[1]{}
\newcommand{\Comment}[1]{}

%% DISCLAIMER START 
% It is INSANELY IMPORTANT NOT TO DO THIS OUTSIDE BEAMER CLASS! IN ARTCILE DOCUMENTS, THIS IS VERY LIKELY TO BUG AROUND!
\makeatletter%
\@ifclassloaded{beamer}%
{
	% TODO 
	% no time... later.   (= never -.-)
	% redefine section to ignore multiple \section calls with the same title
}%
{
	\errmessage{ERROR: section command redefinition outside of beamer class document! Please contact the author of this code or read the F-ing disclaimer.}
}%
\makeatother%
%% DISCLAIMER END

\newcounter{abc}
\newenvironment{alist}{
  \begin{list}{(\alph{abc})}{
      \usecounter{abc}\setlength{\leftmargin}{8mm}\setlength{\labelsep}{2mm}
    }
}{\end{list}}


\newcommand{\stdarraystretch}{1.20}
\renewcommand{\arraystretch}{\stdarraystretch}  % for proper row spacing in tables

\newcommand{\morescalingdelimiters}{   % for proper \left( \right) typography
	\delimitershortfall=-1pt  
	\delimiterfactor=1
}

\newcommand{\centered}[1]{\vspace{-\baselineskip}\begin{center}#1\end{center}\vspace{-\baselineskip}}

% for \implitem and \item[bla] stuff to look right:
\setbeamercolor*{itemize item}{fg=black}
\setbeamercolor*{itemize subitem}{fg=black}
\setbeamercolor*{itemize subsubitem}{fg=black}

\setbeamercolor*{description item}{fg=black}
\setbeamercolor*{description subitem}{fg=black}
\setbeamercolor*{description subsubitem}{fg=black}

\renewcommand{\qedsymbol}{\textcolor{black}{\openbox}}

\renewcommand{\mod}{\mathop{\textbf{mod}}}
\renewcommand{\div}{\mathop{\textbf{div}}}

\newcommand{\ceil}[1]{\left\lceil#1\right\rceil}
\newcommand{\floor}[1]{\left\lfloor#1\right\rfloor}
\newcommand{\abs}[1]{\left\lvert #1 \right\rvert}
\newcommand{\Matrix}[1]{\begin{pmatrix} #1 \end{pmatrix}}
\newcommand{\braced}[1]{\left\lbrace #1 \right\rbrace}

% "something" placeholder. Useful for repairing spacing of operator sections, like `\sth = 42`.
\def\sth{\vphantom{.}}

\def\fract#1/#2 {\frac{#1}{#2}} % ! Trailing space is crucial!
\def\dfract#1/#2 {\dfrac{#1}{#2}} % ! Trailing space is crucial!

\newcommand{\Mid}{\;\middle|\;}

\let\after\circ



\def\·{\cdot}
\def\*{\cdot}
\def\?>{\ensuremath{\rightsquigarrow}}  % Fuck you, Latex
\def\~~>{\ensuremath{\rightsquigarrow}}  

\newcommand{\tight}[1]{{\renewcommand{\arraystretch}{0.76} #1}}
\newcommand{\stackedtight}[1]{\renewcommand{\arraystretch}{0.76} \begin{matrix} #1 \end{matrix} }
\newcommand{\stacked}[1]{\begin{matrix} #1 \end{matrix} }
\newcommand{\casesl}[1]{\delimitershortfall=0pt  \left\lbrace\hspace{-.3\baselineskip}\begin{array}{ll} #1 \end{array}\right.}
\newcommand{\casesr}[1]{\delimitershortfall=0pt  \left.\begin{array}{ll} #1 \end{array}\hspace{-.3\baselineskip}\right\rbrace}
\newcommand{\caseslr}[1]{\delimitershortfall=0pt  \left\lbrace\hspace{-.3\baselineskip}\begin{array}{ll} #1 \end{array}\hspace{-.3\baselineskip}\right\rbrace}

\def\q#1uad{\ifnum#1=0\relax\else\quad\q{\the\numexpr#1-1\relax}uad\fi}
% e.g. \q1uad = \quad, \q2uad = \qquad etc.

\newcommand{\qqquad}{\q3uad}
\newcommand{\minusquad}{\hspace{-1em}}

%% Placeholder utils
% \§{#1}   Saves #1 as placeholder and prints it
% \.       Prints an \hphantom with the size of the recalled placeholder.
\def\indentstring{}
\def\§#1{\def\indentstring{#1}#1}
\def\.{{$\hphantom{\text{\indentstring}}$}}
%% Placeholder utils end

\newcommand{\impl}{\ifmmode\ensuremath{\mskip\thinmuskip\Rightarrow\mskip\thinmuskip}\else$\Rightarrow$\fi\xspace}
\newcommand{\Impl}{\ifmmode\implies\else$\Longrightarrow$\fi\xspace}

\newcommand{\derives}{\Rightarrow}

\newcommand{\gdw}{\ifmmode\mskip\thickmuskip\Leftrightarrow\mskip\thickmuskip\else$\Leftrightarrow$\fi\xspace}
\newcommand{\Gdw}{\ifmmode\iff\else$\Longleftrightarrow$\fi\xspace}

% Legacy code from the algo tutorial slides. Perhaps useful. Try with care.
\mycomment{
	\newcommand{\impl}{\ifmmode\ensuremath{\mskip\thinmuskip\Rightarrow\mskip\thinmuskip}\else$\Rightarrow$\xspace\fi}  
	\newcommand{\Impl}{\ifmmode\implies\else$\Longrightarrow$\xspace\fi}
	
	\newcommand{\gdw}{\ifmmode\mskip\thickmuskip\Leftrightarrow\mskip\thickmuskip\else$\Leftrightarrow$\xspace\fi}
	\newcommand{\Gdw}{\ifmmode\iff\else$\Longleftrightarrow$\xspace\fi}
}
	
\newcommand{\gdwdef}{\ifmmode\mskip\thickmuskip:\Leftrightarrow\mskip\thickmuskip\else:$\Leftrightarrow$\xspace\fi}
\newcommand{\Gdwdef}{\ifmmode\mskip\thickmuskip:\Longleftrightarrow\mskip\thickmuskip\else:$\Longleftrightarrow$\xspace\fi}

\newcommand{\symbitemnegoffset}{\hspace{-.5\baselineskip}}
\newcommand{\implitem}{\item[\impl\symbitemnegoffset]}
\newcommand{\Implitem}{\item[\Impl\symbitemnegoffset]}


\newcommand{\forcenewline}{\mbox{}\\}

\newcommand{\bfalert}[1]{\textbf{\alert{#1}}}
\let\elem\in   % I'm a Haskell freak. Don't judge me. :P


\def\|#1|{\text{\normalfont #1}}  % | steht für senkrecht (anstatt kursiv wie sonst im math mode)


% proper math typography
\newcommand{\functionto}{\longrightarrow}
\renewcommand{\geq}{\geqslant}
\renewcommand{\leq}{\leqslant}
\let\oldsubset\subset
\renewcommand{\subset}{\subseteq} % for all idiots out there using subset

\newenvironment{threealign}{%
	\[
	\begin{array}{r@{\ }c@{\ }l}
}{%
	\end{array}	
	\]
}

\newcommand{\concludes}{ \\ \hline  }
\newcommand{\deduction}[1]{
	\begin{varwidth}{.8\linewidth}
		\begin{tabular}{>{$}c<{$}}
			#1
		\end{tabular}
	\end{varwidth}	
}

\definecolor{hoareorange}{rgb}{1,.85,.6}
\newcommand{\hoareassert}[1]{\setlength{\fboxsep}{1pt}\setlength{\fboxrule}{-1.4pt}\fcolorbox{white}{hoareorange}{\ensuremath{\{\;#1\;\}}}\setlength\fboxrule{\defaultfboxrule}\setlength{\fboxsep}{3pt}}

\newcommand{\mailto}[1]{\href{mailto:#1}{{\textcolor{blue}{\underline{#1}}}}}
\newcommand{\urlnamed}[2]{\href{#2}{\textcolor{blue}{\underline{#1}}}}
\renewcommand{\url}[1]{\urlnamed{#1}{#1}}

\newcommand{\hanging}{\hangindent=0.7cm}
\newcommand{\indented}{\hanging}


% \hstretchto prints #2 left-aligned into a box of the width of #1
\def\hstretchto#1#2{%
	\mbox{}\vphantom{#2}\rlap{#2}\hphantom{#1}%
}

\def\vstretchto#1#2{%
	\mbox{}\hphantom{#2}\smash{#2}\vphantom{#1}%
}

% \hstretchtocentered prints #2 centered into a box of the width of #1
\def\hstretchtocentered#1#2{%
	\mbox{}\vphantom{#2}\scalebox{0.5}{\hphantom{#1}}\clap{#2}\scalebox{0.5}{\hphantom{#1}}%
}

% vertical centering
\newcommand{\vertcenter}[1]{%
	\ensuremath{\vcenter{\hbox{#1}}}%
}


%requires \thisyear to be defined (s. config.tex)!
\edef\nextyear{\the\numexpr\thisyear+1\relax}


% --- \frameheight constant ---
\newlength\fullframeheight
\newlength\framewithtitleheight
\setlength\fullframeheight{.92\textheight}
\setlength\framewithtitleheight{.86\textheight}

\newlength\frameheight
\setlength\frameheight{\fullframeheight}

\let\frametitleentry\relax
\let\oldframetitle\frametitle
\def\newframetitle#1{\global\def\frametitleentry{#1}\if\relax\frametitleentry\relax\else\setlength\frameheight{\framewithtitleheight}\fi\oldframetitle{#1}}
\let\frametitle\newframetitle

\def\newframetitleoff{\let\frametitle\oldframetitle}
\def\newframetitleon{\let\frametitle\newframetitle}
% --- \frameheight constant end ---

\newcommand{\fakeframetitle}[1]{%
	\vspace{-2.05\baselineskip}%
	{\Large \textbf{#1}} \\%
	\smallskip
}



\newenvironment{headframe}{\Huge THIS IS AN ERROR. PLEASE CONTACT THE ADMIN OF THIS TEX CODE. (headframe env def failed)}{}
\RenewEnviron{headframe}[1][]{
	\begin{frame}\frametitle{\ }
		\centering
		\Huge\textbf{\textsc{\BODY} \\
		}
		\Large {#1}
		\frametitle{\ }
	\end{frame}
}


\makeatletter
% Provides color if undefined.
\newcommand{\colorprovide}[2]{%
	\@ifundefinedcolor{#1}{\colorlet{#1}{#2}}{}}
\makeatother


\colorprovide{lightred}{red!30}
\colorprovide{lightgreen}{green!40}
\colorprovide{lightyellow}{yellow!50}
\colorprovide{lightblue}{blue!10}
\colorprovide{beamerlightred}{lightred}
\colorprovide{beamerlightgreen}{lightgreen}
\colorprovide{beamerlightyellow}{lightyellow}
\colorprovide{beamerlightblue}{lightblue}
\colorprovide{fullred}{red!60}
\colorprovide{fullgreen}{green}
\definecolor{darkred}{RGB}{115,48,38}
\definecolor{darkgreen}{RGB}{48,115,38}
\definecolor{darkyellow}{RGB}{100,100,0}

\only<handout:0>{\colorlet{adaptinglightred}{beamerlightred}}
\only<handout:0>{\colorlet{adaptinglightgreen}{beamerlightgreen}}
\only<handout:0>{\colorlet{adaptinglightyellow}{beamerlightyellow}}
\only<handout:0>{\colorlet{adaptinglightblue}{beamerlightblue}}
\only<beamer:0>{\colorlet{adaptinglightred}{lightred}}
\only<beamer:0>{\colorlet{adaptinglightgreen}{lightgreen}}
\only<beamer:0>{\colorlet{adaptinglightyellow}{lightyellow}}
\only<beamer:0>{\colorlet{adaptinglightblue}{lightblue}}
\only<handout:0>{\colorlet{adaptingred}{lightred}}
\only<beamer:0>{\colorlet{adaptingred}{fullred}}
\only<handout:0>{\colorlet{adaptinggreen}{lightgreen}}
\only<beamer:0>{\colorlet{adaptinggreen}{fullgreen}}



\newcommand{\TrueQuestion}[1]{
	\TrueQuestionE{#1}{}
}

\newcommand{\YesQuestion}[1]{
	\YesQuestionE{#1}{}
}

\newcommand{\FalseQuestion}[1]{
	\FalseQuestionE{#1}{}
}

\newcommand{\NoQuestion}[1]{
	\NoQuestionE{#1}{}
}

\newcommand{\DependsQuestion}[1]{
	\DependsQuestionE{#1}{}
}

\newcommand{\QuestionVspace}{\vspace{4pt}}
\newcommand{\QuestionParbox}[1]{\begin{varwidth}{.85\linewidth}#1\end{varwidth}}
\newcommand{\ExplanationParbox}[1]{\begin{varwidth}{.97\linewidth}#1\end{varwidth}}
\colorlet{questionlightgray}{gray!23}
\let\defaultfboxrule\fboxrule

% #1: bg color
% #2: fg color short answer
% #3: short answer text
% #4: question
% #5: explanation
\newcommand{\GenericQuestion}[5]{
	\setlength\fboxrule{2pt}
	\only<+|handout:0>{\hspace{-2pt}\fcolorbox{white}{questionlightgray}{\QuestionParbox{#4} \quad\textbf{?}}}
	\visible<+->{\hspace{-2pt}\fcolorbox{white}{#1}{\QuestionParbox{#4} \quad\textbf{\textcolor{#2}{#3}}} \if\relax#5\relax\else\ExplanationParbox{#5}\fi} \\
	\setlength\fboxrule{\defaultfboxrule}
}

% #1: Q text
% #2: Explanation
\newcommand{\TrueQuestionE}[2]{
	\GenericQuestion{adaptinglightgreen}{darkgreen}{Wahr.}{#1}{#2}
}

% #1: Q text
% #2: Explanation
\newcommand{\YesQuestionE}[2]{
	\GenericQuestion{adaptinglightgreen}{darkgreen}{Ja.}{#1}{#2}
}

% #1: Q text
% #2: Explanation
\newcommand{\FalseQuestionE}[2]{
	\GenericQuestion{adaptinglightred}{darkred}{Falsch.}{#1}{#2}
}

% #1: Q text
% #2: Explanation
\newcommand{\NoQuestionE}[2]{
	\GenericQuestion{adaptinglightred}{darkred}{Nein.}{#1}{#2}
}

% #1: Q text
% #2: Explanation
\newcommand{\DependsQuestionE}[2]{
	\GenericQuestion{adaptinglightyellow}{darkyellow}{Je nachdem!}{#1}{#2}
}

% #1: Q text
% #2: Answer
\newcommand{\ContentQuestion}[2]{
	\GenericQuestion{adaptinglightblue}{black}{\minusquad}{#1}{#2}
}

\ifnum\thisyear=2021 \else \errmessage{Old ILIAS link inside preamble. Please update.} \fi

\newcommand{\ILIAS}{\urlnamed{ILIAS}{\myILIASurl}\xspace}
\newcommand{\Klausurtermin}{\myKlausurtermin\xspace}

\newcommand{\Socrative}{\ifdefined\mysocrativeroom \only<handout:0>{socrative.com $\quad \~~> \quad $ Student login \\ Raumname:  \mysocrativeroom\\ \medskip}\else\fi}

\newcommand{\thasse}[1]{
	\ifdefined\ThassesTut #1\xspace \else\fi
}
\newcommand{\daniel}[1]{
	\ifdefined\DanielsTut #1\xspace \else\fi
}
\newcommand{\thassedaniel}[2]{\ifdefined\ThassesTut #1\else\ifdefined\DanielsTut #2\fi\fi\xspace}

\ifdefined\ThassesTut \ifdefined\DanielsTut \errmessage{ERROR: Both ThassesTut and DanielsTut flags are set. This is most likely an error. Please check your config.tex file.} \else \fi \else \ifdefined\DanielsTut \else \errmessage{ERROR: Neither ThassesTut  nor DanielsTut flags are set. This is most likely an error. Please check your config.tex file.} \fi\fi

%\newcommand{\sgn}{\text{sgn}}

%%%%%%%%%%%% INHALT %%%%%%%%%%%%%%%%

%% Wochennummer
\newcounter{weeknum}

%% Titelinformationen
\title[GBI-Tutorium \mytutnumber, Woche \theweeknum]{Grundbegriffe der Informatik \\ Tutorium \mytutnumber}

\subtitle{Woche \theweeknum\xspace |\xspace\mydate{\theweeknum} \\ \myname \ \  \normalfont (\mailto{\mymail})}
\author[\myname]{\myname}
\institute{KIT -- Karlsruher Institut für Technologie}
\date{\mydate{\theweeknum}\ }

% Modified, DJ (better safe than sorry)
\AuthorTitleSep{ – }

%% Titel einfügen
\newcommand{\titleframe}{\frame{\titlepage}}

%% Alles starten mit \starttut{X}
\newcommand{\starttut}[1]{\setcounter{weeknum}{#1}\pdfinfo{
		/Author (\myname)
		/Title  (GBI-Tutorium \mytutnumber, Woche \theweeknum)
	}\titleframe\frame{\frametitle{Inhalt}\tableofcontents} \AtBeginSection[]{%
		\begin{frame}{Wo sind wir gerade?}
		\tableofcontents[currentsection]
	\end{frame}\addtocounter{framenumber}{-1}}}


\newcommand{\framePrevEpisode}{
\begin{headframe}
	\mylasttimestext
\end{headframe}
}

\newcommand{\lastframetitled}[6]{
	\frame{\frametitle{#6}
		\vspace{-#2\baselineskip}
		\begin{figure}[H]
			\centering
			\LARGE \textbf{\textsc{#5}} \\
			\vspace{.2\baselineskip}
			\includegraphics[#1]{#3}
			\vspace{-6pt}
			\begin{center}
				\small \url{#4} 
			\end{center}
		\end{figure} 
	}
}

% #1 number
% #2 title 
% #3 vspace (positive) without unit (\baselineskip)
\newcommand{\xkcdframe}[3]{
	\lastframetitled{width=.96\textwidth}{#3}{xkcd/#1}{http://xkcd.com/#1}{}{#2}
}

\newcommand{\xkcdframevert}[3]
{
	\lastframetitled{height=.96\frameheight}{#3}{xkcd/#1}{http://xkcd.com/#1}{}{#2}
}

% #1 number
% #2 title 
% #3 vspace (positive) without unit (\baselineskip)
% #4 \includegraphics[] optional parameters
\newcommand{\xkcdframecustom}[4]
{
	\lastframetitled{#4}{#3}{xkcd/#1}{http://xkcd.com/#1}{}{#2}
}

\newcommand{\slideThanks}{
	\begin{frame}
	\frametitle{Credits}
	\begin{block}{}
		An der Erstellung des Foliensatzes haben mitgewirkt:\\[1em]
		Daniel Jungkind \\
		Thassilo Helmold \\
		Philipp Basler \\
		Nils Braun \\
		Dominik Doerner \\
		Ou Yue \\
		Max Schweikart
	\end{block}
\end{frame}
}

%% Wörter DEPRECATED! DO NOT USE
\newcommand{\code}[1]{$\mathbf{#1}$}

\begin{document}
\starttut{1}

\section{\thassedaniel{Organisatorisches}{Orga-Kram}}

\aboutMeFrame

\begin{frame}[t]{Tutorium (Def.)}
	\begin{itemize}[<+->]
		\item Inhalte der VL verstehen und anwenden
		\implitem Den Stoff etwas weniger formal behandeln
		\item \textbf{Eure Fragen} klären!
		\item Beispiele, Aufgaben (da kommt \textbf{ihr} ins Spiel! \smiley)
		\item Besprechung häufiger Fehler auf den Übungsblättern
	\end{itemize}
	% No time. Kurz mündlich.
	% \NoQuestionE{Dürft ihr mich siezen?}{Ihr werdet sonst erschossen. \smiley \hspace{.5\linewidth}\mbox{} \\ \impl Sagt einfach \textbf{Du}.} 
\end{frame}

\begin{frame}[t]{Sicherheitshinweise}
	\begin{itemize}[<+->]
		\item Das Tutorium ersetzt die Vorlesung \textbf{nicht!} \\
			Wir werden manche Formalien auslassen, diese können aber dennoch für die Klausur wichtig sein!
		\item This is a test - Ihr seid die Versuchskaninchen für mein erstes Tutorium 
		\item Wenn ich / meine Folien Blödsinn reden: \\
			Schreien! / Mail an mich! \\
			Verbindlich \textbf{nur} Inhalt aus der VL/Übung! 
		\item \textbf{Feedback erwünscht}: \\
			Zu schnell/langsam/leicht/schwer/viel/wenig/... ? \impl \textbf{Sagt es mir!}\\
			Jederzeit per Mail oder direkt im Tut.
	\end{itemize}
\end{frame}

\thasse{   %  Timing issues. Showing that XKCD takes time (esp. reading everything!)
	\only<beamer:0>{
	\begin{frame}{Feedback}
		 Feedback bitte so konkret wie möglich, mit \enquote{Ich gebe dem Tut 4 Sterne.} kann ich nicht viel anfangen.
		
		\begin{figure}
			\centering 
			\includegraphics[scale=0.5]{xkcd/tornadoguard_937}
		\end{figure}
		
	\end{frame}
	}
}

\begin{frame}[t]{Präsenztutorium}
	\begin{itemize}
		\item \textbf{3G-Pflicht} in jedem Präsenz-Tutorium \\
			Gültigkeit: \quad Schnelltest: 24~h \quad PCR-Test: 48~h
		\pause
		\item Bei jedem Tutoriumstermin Kontrolle 
		\implitem Bitte ein paar Minuten vorher da sein
		\item Vor dem Betreten des Raums KONKIT-QR-Code scannen
		\pause
		\item Maskenpflicht im Seminarraum (außer jeder hat mindestens 1,5~m Abstand)
	\end{itemize}
\end{frame}


\begin{frame}[t]{GBI bestehen}
	GBI habt ihr bestanden, wenn ihr Übungsschein* und Klausur bestanden habt:
	\begin{itemize}
		\item \textbf{Übungsschein}: im Laufe des Semesters \\
			  bestanden, wenn ihr min. je 50~\% der Gesamtpunkte in der ersten und zweiten Hälfte der Übungsblätter erreicht \\

		\item \textbf{Klausur}: \Klausurtermin \\
			  schriftlich, keine Hilfsmittel, sicher bestanden ab 50~\% der Punkte
	\end{itemize}
	\pause
	\DependsQuestionE{* Muss ich den Schein machen?}{
		Infos, InWis, Info-Lehramt: \quad  Ja \quad - Alle Anderen: \quad Vielleicht \\
		{\small Das alles \textbf{ohne Gewähr}! Eure Fakultäten haben am Ende Recht.} \\
		\smallskip
		\Impl Trotzdem: ÜBs machen lohnt sich. \smiley \\
	}
	\pause
	GBI ist Teil der \textbf{Orientierungsprüfungen} für Informatiker und Info-Lehramt!
	\begin{itemize}
		\implitem Übungsschein im ersten Semester \textbf{versuchen} (da im Zweiten nicht angeboten), im Dritten bestehen! 
		\implitem Die Klausur könnt ihr auch ohne Schein schreiben und müsst ihr \textbf{spätestens} im zweiten Semester versuchen und im Dritten bestehen!
	\end{itemize}
\end{frame}

\begin{frame}{Übungsblätter}
	Im laufenden Semester werden wöchentlich insgesamt 12 Übungsblätter ausgegeben:
	\begin{itemize}
		\item \hstretchto{\textbf{Rückgabe}:}{\textbf{Ausgabe}:} Freitags, üblicherweise um 12h im \ILIAS \\
		\item \hstretchto{\textbf{Rückgabe}:}{\textbf{Abgabe}:} Spätestens Freitags (7 Tage später), 12:30 Uhr in den GBI-Briefkasten (1.UG Infobau)*
		\item \textbf{Rückgabe}: Montags zum Tutorium
	\end{itemize}
	\pause	
	* Was, wenn der Briefkasten kaputt ist? \\
	\impl Alternativ bei Marko Kleine Büning (Raum 016) abgeben \\
\end{frame}

\begin{frame}{Übungsblätter - Bearbeitungshinweise}
	\begin{itemize}
		\item Bearbeitungen müssen \textbf{handschriftlich} und inklusive \textbf{ausgefülltem Deckblatt} auf Papier abgegeben werden \\
		\pause
		\item Bei verspäteten Abgaben wird das Blatt \textbf{nicht gewertet} und eine Korrektur kann nicht garantiert werden
		\pause
		\item Die ersten sechs Blätter dürft ihr zu zweit bearbeiten, den Rest nur alleine
		\implitem \textbf{Keine} Plagiate, \textbf{keine} Gruppenabgabe (sonst \textbf{Übungsschein weg}!)
		\pause
		\item Aufgabenblätter bitte \textbf{nicht} mit abgeben (spart Papier)
		\pause
		\item Schreiben in Schwarz oder Blau (oder andere dunkle Farbe $\notin \{\text{Rot}, \text{Grün}, \text{Weiß}\}$ )	
		\item Links und rechts bitte einen Rand frei lassen. \\
			  Dann kann ich's auch korrigieren. \smiley
		\item Ordnung behalten: \\
			  \textbf{Welche Aufgabe} ist das gerade?\quad \impl Deutlich machen!
		\item Was ich nicht lesen kann, gibt keine Punkte! \\
			  \small Was ich schwer lesen kann, u.~U. auch... 
	\end{itemize}
\end{frame}



\begin{frame}{Wo bekomme ich Hilfe?}	
	\begin{block}{Fragen}
		\begin{itemize}
			\item \textbf{Hier} im Tut!
			\item Ins Forum im \ILIAS. (Dann haben alle was davon. \smiley) \\
			\pause
			\item Organisatorischer Spezialkram? \\
				  \impl an Kleine Büning/Sinz \\
						(\mailto{marko.kleinebuening@kit.edu} / \mailto{carsten.sinz@kit.edu}) \\
				  Bei Orga-Problemen aber \textbf{Name} und \textbf{Matrikelnummer}  mitangeben! \\
				  Fachliches bitte nicht per Mail, lieber \impl Sprechstunde, \ILIAS...!
			\item Tut-spezifisches an \textbf{mich} (\mailto{\mymail})
		\end{itemize}
	\end{block}
	\pause
	\begin{block}{Content}
		\begin{itemize}
			\item Vorlesungs-, Übungs- und Tutoriums-Folien, Skript, Übungsblätter, Lösungen, ...: im \ILIAS.
			\item Alte Klausuren und Übungsblätter gibt's im Archiv: \url{http://gbi.ira.uka.de}
		\end{itemize}
	\end{block}
\end{frame}

\begin{frame}{Wo bekomme ich Hilfe?}	
	\begin{block}{Und wenn's um persönliche Themen geht?}
		\begin{itemize}
			\item Allgemeine Studienberatung: \url{https://www.sle.kit.edu/imstudium/zib.php}
			\item Informatik Studiengangservice: \url{https://www.informatik.kit.edu/iss.php}
			\item Studium und Behinderung: \url{https://www.studiumundbehinderung.kit.edu/}
			\item Psychotherapeutische Beratung: \url{https://www.sw-ka.de/en/beratung/}
			\item Telefonseelsorge: \url{https://www.telefonseelsorge-karlsruhe.de/}
			\item Anonymes Zuhörtelefon: \url{https://nightline-karlsruhe.de/}
		\end{itemize}
	\end{block}
\end{frame}

% No time. Seriously, no.
%
% Pretty irrelevant. Don't spend a lot of time here or leave it out completely.
%

\section{Signale, Mitteilungen, Informationen, Daten}

\morescalingdelimiters

\begin{frame}{Signale, Mitteilungen, Informationen, Daten}
	\begin{block}{Signal, Mitteilung und Medium}
		Ein Signal überträgt eine Mitteilung. Mitteilungen können in einem Medium gespeichert werden.
	\end{block}
	\pause
	\begin{block}{Nachricht}
		Mitteilung, bei der vom Medium und den Einzelheiten der Signale abstrahiert wird.
	\end{block}
	\pause
	\begin{block}{Information}
		Durch \textit{Interpretation} in einem \textit{Bezugssystem} kann man einer Nachricht eine Information (Bedeutung) zuweisen. \\
		\impl kontextabhängig!
	\end{block}

% Not relevent as of WS 20/21
%	\pause
%	\begin{block}{Informationsgehalt einer Nachricht}
%		Bei gleicher Wahrscheinlichkeit möglicher Nachrichten: \\ \quad In etwa \quad $\log(\text{Anzahl möglicher Nachrichten})$ \\
%		Versch. Einheiten: \\ \vspace{-.2\baselineskip}
%		\begin{itemize}
%			\item Naturalis: $\log_e$ 
%			\item Hartley: $\log_{10}$ 
%			\item Shannon: $\log_2$
%		\end{itemize}
%		\impl Mehr dazu in TGI
%	\end{block}	
\end{frame}


\section{Mengen}

\morescalingdelimiters

\begin{frame}{Mengen}
	\pause % -> HIER Sammeln, was Tutanden bereits wissen (an der Tafel)
	\begin{block}{Definition: Mengen} \vspace{-.5\baselineskip}
		\begin{itemize}
			\item \textbf{Menge} $M$: Ansammlung verschiedener Objekte
			\item $m \in M$: Objekt $m$ ist \textbf{Element} der Menge $M$
			\item Schreibweisen: \\
			$M = \{m_1, m_2, m_3 \}, \qquad M = \{m \mid m\text{ ist toll} \}$
			\item Reihenfolge der Aufzählung wurscht, Duplikate auch: \\
			$\{2,1,3,1,4\} = \{1,2,3,4\}$
			\item \textbf{Leere Menge} $\emptyset$: enthält keine Elemente, \quad also $\{\} = \emptyset$
		\end{itemize}
	\end{block}
	\pause
	
	\begin{exampleblock}{Beispiel}
		Wichtige Mengen sind \\ 
		\centered{$\Z, \Q, \R, \C, \qquad \N_+ = \{1, 2, 3, \dots\}, \quad \N_0 = \{0, 1, 2, 3, \dots\}$} 
		Es gilt:\\ 
		\centered{$-5 \in \Z\qquad {-5} \notin \N_0$}
	\end{exampleblock}
	
\end{frame}

\mycomment{ % Too complicated and too little time.
	\begin{frame}{Anwendung}
		Wir haben ein „Universum“ an Dingen:\\ 
		$U = \{ $ Gurken, Obst, Fußball, Hockey, Äpfel, Maracujas, Tomaten, Bananen, Karotten, Wurzelpetersilie, Sport,  Orangen, Gemüse, Basketball, Ersti-Weitwurf $ \} $  \\[0.5em]
		
		\pause
		Ordnen wir diese in eine Teilmenge $C$ für Kategorien und jeweils eine Teilmenge $E_c$ für die Elemente einer Kategorie $c \in C$. \\ {\small (\textbf{Überhaupt} nicht konstruiert!)} \\[0.5em]
		
		\pause
		$C = \{$ Obst, Gemüse, Sport $\}$ \\[0.3em]
		$E_{\text{Obst}} = \{$ Äpfel, Orangen, Maracujas, Bananen $\}$ \\
		$E_{\text{Gemüse}} = \{$ Tomaten, Karotten, Gurken, Wurzelpetersilie $\}$ \\
		$E_{\text{Sport}} = \{$ Fußball, Basketball, Hockey, Ersti-Weitwurf $\}$ \\
		
		\vspace{.5\baselineskip}
		\Large \alert{
			\textbf{Achtung}: 
			$$\underbrace{\text{Obst}}_{\mathllap{\text{ein Bezeichner, keine Menge}}}
			\neq 
			\underbrace{E_{\text{Obst}}}_{\mathrlap{\text{eine Menge}}}$$
		}
		
	\end{frame}
}


\begin{frame}{Kardinalität, Teilmengen, Gleichheit}
	
	\begin{block}{Definitionen}  \vspace{-.4\baselineskip}
		\begin{itemize}
			\item \textbf{Kardinalität} $\setsize{M}$: \\
			Anzahl der Elemente einer endlichen Menge $M$ \\
			\pause
			\item $N \subseteq M$: \\
			Menge $N$ ist \textbf{Teilmenge} von $M$, also jedes Element aus $N$ auch in $M$ \\
			Es gilt: \qqquad $ N \subseteq M \iff \forall n \in N : n \in M$
			\pause
			\item $N = M$: \\
			Mengen $N$ und $M$ sind \textbf{gleich}, enthalten also die gleichen Elemente \\
			Es gilt: \qqquad $ N = M \iff N \subseteq M \ \text{und} \ M \subseteq N$
		\end{itemize}
	\end{block} 
\end{frame}

\begin{frame}{Kardinalität, Teilmengen, Gleichheit}
	
	\begin{exampleblock}{Beispiel}
		
		\begin{align*}
		\setsize{\{1,2,3,2,1\}} &= \only<2->{3} \\
		\setsize{ \emptyset } &= \only<3->{0} \\
		\setsize{\{\{\}, 1, \{2,3\} \}} &= \only<4->{3} \\
		\{1,2\} \only<1-4|handout:0>{ &\mathrel{?\ }  } \only<5->{&\subseteq}  \{1,2,3\} \\
		\{1,2\} \only<1-5|handout:0>{ &\mathrel{?\ } } \only<6->{&\nsubseteq} \{\text{Hund}, \text{Katze}, \text{Maus}\} \\
		\{1,2,3\} \only<1-6|handout:0>{ &\mathrel{?\ } } \only<7->{&\supseteq} \{1,2\} \\
		\{\text{Sonne}, \text{Mond},\text{Sterne}\} \only<1-7|handout:0>{ &\mathrel{?\ } } \only<8->{&\subseteq} \{\text{Sterne},\text{Planet},\text{Sonne},\text{Mond}\}
		\end{align*}
	\end{exampleblock} 
	

\end{frame}

\begin{frame}{Schnitt und Vereinigung}
	
	\begin{block}{Definition}
		Seien $M$ und $N$ zwei Mengen, so heißen
		$$M \cap N := \left\{x \Mid x \in M \text{ und } x \in N \right\} \qquad M \cup N := \left\{x \Mid x \in M \text{ oder } x \in N \right\} $$
		\textbf{Durchschnitt} bzw. \textbf{Vereinigung} von $M$ und $N$.\\[1em] 
		\pause
		$M$ und $N$ heißen \textbf{disjunkt}, wenn sie keine gemeinsamen Elemente haben, also $M \cap N = \emptyset$ (der Schnitt ist leer).
	\end{block}
	
	\pause
	\begin{exampleblock}{Beispiel}
		\centered{$ \{1,2\} \cup \{2,3\} = \{1,2,3\} \qquad \{1,2\} \cap \{2,3\} = \{2\} $}
	\end{exampleblock}
	
	\pause
	
	\begin{block}{Lemma}
		Es gilt: \\ \centered{$\setsize{M \cup N} \ = \ \setsize{M} + \setsize{N} - \setsize{M \cap N}$}
	\end{block}
	
	
\end{frame}

\begin{frame}{Mengendifferenz}
	\begin{block}{Definition}
		Seien $A$ und $B$ zwei Mengen, so heißt $$ A\setminus B := \left\{ x \Mid x\in A \text{ und } x\notin  B  \right\} $$ die \textbf{Differenz}(-menge) von $A$ und $B$.
	\end{block}
	
	\pause
	
	\begin{block}{Lemma}
		Es gilt: $$ A \subseteq B \iff A \setminus B = \emptyset $$
	\end{block} 
	
	\pause
	
	\begin{exampleblock}{Weitere Beispiele}
		\vspace{-2\baselineskip} % WtF LaTeX??
		\begin{align*}
		A \cup \emptyset &= \uncover<4->{  A }  \\
		A \cap \emptyset &= \uncover<5->{ \emptyset }\\
		\N_+ \cup \{0\} &= \uncover<6->{ \N_0} \\
		\N_0 \setminus \{0\} &= \uncover<7->{ \N_+} 
		\end{align*}
	\end{exampleblock}
	
\end{frame}

\begin{frame}{Eine Menge Mengen...}
	\begin{block}{Aufgabe}
		Wir haben Mengen $A = \{1, 2\}, B = \{3\}, C = \{1, 3\}; A,B,C \subseteq M = \{1, 2, 3\}$.\\
		Dann ist:
		\begin{align*}
		A \cup B &= \visible<2->{ \{1, 2, 3\} }  \\
		A \cap C &= \only<3->{ \{1\} }\\
		A \setminus C &= \only<4->{ \{2\} }\\
		B \setminus A &= \only<5->{ \{3\} }\\
		A \cup (B \setminus C) &= \only<6->{ \{1, 2\} }\\
		C &= \{1, 3\} \\
		(A \setminus C) \cup B &= \only<7->{ \{2, 3\} }\\
		A \cap B &= \only<8->{ \emptyset }
		\end{align*}
	\end{block}
\end{frame}

\begin{frame}[t]{Mengengleichheit: Beispiel}
	\begin{itemize}
		\item<1-> Sei $ A $ und $M$ beliebige Mengen. Zeigt, dass gilt 
		\begin{align*}
		A &=  \underbrace{ \left(A \setminus M \right)}_{T_1} \cup  \underbrace{ \left(A \cap M \right)}_{T_2}  
		\end{align*}		 
		\only<2-5>{
			\item<2-5> \textbf{Richtung}: $ A \subseteq T_1 \cup T_2 $ 
			\item<3-5> Wähle $x\in A$ und unterscheide:
			\item<4-5> Fall 1: Ist $x\in M$, so gilt $x\in A $ und $x\in M$ und damit $x\in A\cap M  = T_2 $
			\item<5-5> Fall 2 : Ist $x\notin M$, so gilt $x\in T_1$, da $T_1 = \left\{x \Mid x \in A \text{ und } x\notin M \right\} $  }
		\only<6->{\item<6-> \textbf{Richtung}: $T_1 \cup T_2 \subseteq A $ 
			\item<7-> Wähle $x\in T_1 \cup T_2$. Dies bedeutet $x\in T_1 $ oder $x\in T_2$. 
			\item<8-> Fall 1: $x\in T_1$ (angenommen $T_1 \neq \emptyset$). Aus Definition folgt $x\in A$.
			\item<9-> Fall 2: $x\in T_2$ (angenommen $T_2 \neq \emptyset$). Somit $x\in A$ und $x\in M$. $\qed$}
	\end{itemize}
\end{frame}

\begin{frame}[t]{Mengengleichheit: Aufgabe}
	\begin{itemize}
		\item<1-> Sei $ A, B, C $ beliebige Mengen. Zeigt, dass gilt 
		\begin{align*}
		A \cup (B \cap C) &=  (A \cup B) \cap (A \cup C)
		\end{align*}		 
		\only<2-5>{
			\item<2-5> \textbf{Richtung}: $ A \cup (B \cap C) \subseteq  (A \cup B) \cap (A \cup C) $ 
			\item<3-5> Sei $ x \in A \cup (B \cap C) $.
			\item<4-5> Fall 1: Ist $ x \in A$, so folgt $ x \in A \cup B $ und $ x\in A \cup C$ und damit $  x\in (A \cup B) \cap (A \cup C) $
			\item<5-5> Fall 2 : Ist $x\notin A$, so gilt $ x \in B \cap C$, also auch $ x \in B$ und $ x \in C$. Dann ist auch $ x \in A \cup B $ und $ x \in A \cup C$ und es folgt $  x\in (A \cup B) \cap (A \cup C) $
		}
		\only<6->{
			\item<6-> \textbf{Richtung}: $ A \cup (B \cap C) \supseteq  (A \cup B) \cap (A \cup C) $
			\item<7-> Wähle $ x \in (A \cup B) \cap (A \cup C) $. Dies bedeutet $x\in A \cup B $ und $ x \in A \cup C $. 
			\item<8-> Fall 1: $ x \in A $. Dann folgt $ x \in A \cup (B \cap C) $
			\item<9-> Fall 2: $ x \notin A$. Dann muss $ x \in B $ und $ x \in C $ gelten, denn sonst wäre $ x \notin (A \cup B) \cap (A \cup C) $. Somit $ x\in B \cap C$ und $ x \in A \cup (B \cap C) $. $\qed$
		}
	\end{itemize}
\end{frame}

\section{Potenzmengen}

\begin{frame}{... in einer Menge! (Potenzmengen)}
	\begin{block}{Definition}
		Die \textbf{Potenzmenge} $2^M$ oder auch $\Pot (M)$ ist die Menge aller möglicher Teilmengen von $M$. 
		\begin{align*}
			2^M := \left\{ N \Mid  N \subseteq M \right\}
		\end{align*}
	\end{block}
	\pause
	
	\begin{exampleblock}{Beispiel}
		Sei $M = \left\{ 1,2,0 \right\} $. \\ \pause
		
		Dann gilt  
		\begin{align*}   
		2^M &= \left\{ \emptyset, \{ 0 \}, \{ 1 \}, \{ 2 \}, \{ 0,1 \} , \{ 0,2 \}, \{ 1,2 \}, \{ 0,1,2 \} \right\}
		\end{align*}
		
		\textbf{Beachte}: Es gilt immer $M \in 2^M \ \text{und} \ \emptyset \in 2^M$.
	\end{exampleblock}
	
\end{frame}

\begin{frame}{Potenzmengen}
	\begin{block}{Aufgabe}
		Wie viele Elemente enthält $2^M$? \\[0.5em]
		
		\pause
		$2^{\setsize{M}}$
	\end{block}
	
	\pause
	\begin{block}{Noch ne Aufgabe}
		Gebt eine Abbildung $\phi \colon 2^{M} \functionto 2^{M}$ so an,
		dass für jedes $L \in 2^{M}$ und für jedes $w \in M$ gilt:
		\begin{equation*}
			w \in L \text{ genau dann, wenn } w \notin \phi(L).
		\end{equation*}
		
		\pause
		\begin{threealign}
			\phi \colon 2^{M} &\functionto& 2^{M},\\
			L &\mapsto& M \setminus L.
		\end{threealign}
	\end{block}
\end{frame}

\section{Paare}

\begin{frame}{Paare}
	\begin{block}{Definition}
		Seien $A$ und $B$ zwei Mengen und $a \in A$, $b \in B$.\\
		$$(a, b)$$ heißt \textbf{Paar} mit der ersten Komponente $a$ und der zweiten Komponente $b$.\\[1em]
		\pause
		Paare sind \textbf{keine Mengen, sondern was \alert{völlig anderes}!} \\
		In Paaren sind Duplikate \textbf{möglich} und die Reihenfolge der Elemente ist \textbf{wichtig}.\\
	\end{block}

	\pause
	\begin{exampleblock}{Beispiel}
		$$ (42, \triangle) = (42, \triangle) \qquad (42, \triangle) \neq (\triangle, 42) \qquad (1, 1) \neq \underbrace{(1)}_{\mathclap{\text{überhaupt gar kein Paar!}}} $$
	\end{exampleblock}
\end{frame}

\begin{frame}
	\centering
	\Huge
	\alert{Runde und geschweifte Klammern 
		   $$\Big(\;\Big) \textbf{\quad vs. \quad} \Big\{\;\Big\}$$ \\
		   \textbf{NICHT VERWECHSELN!}}
\end{frame}

\begin{frame}{Paare}
	Das Konzept der Paare lässt sich auf das Konzept der Mengen zurückführen.
	
	\begin{block}{Aufgabe}
		Gegeben sei die Menge $M = \{m_1, m_2\}$.\\
		Wie kann man die Paare $(m_1, m_2)$ und $(m_2, m_1)$ eindeutig darstellen, allerdings nur unter Verwendung von Mengen und $m_1, m_2$?
	\end{block}
	\pause
	
	\begin{block}{Lösung}
		Wir definieren: \\ 
		$(m_1, m_2) := \left\lbrace m_1, m_2, \{m_1\}\right\rbrace$ und $(m_2, m_1) := \left\lbrace m_1, m_2, \{m_2\}\right\rbrace$
	\end{block}
\end{frame}

\section{Kartesisches Produkt und Relationen}

\begin{frame}{Kartesisches Produkt}
	\begin{block}{Definition}
		Für Mengen $A$ und $B$ ist
		$$A \times B := \set{ (a,b) \Mid a \in A, b \in B }$$
		das \textbf{kartesische Produkt} von $A$ und $B$. \\
		\impl Das kann man auch für beliebig viele Mengen machen: \\
			Das kartesische Produkt von $n$ Mengen $M_1,...,M_n$ enthält $n$-Tupel der Form $(m_1, m_2, ..., m_n)$.
	\end{block}

	\pause
	\begin{exampleblock}{Beispiel}
		\begin{itemize}
			\item $\{\triangle,\square\} \times \{1, 2, 3\} = \left\{(\triangle, 1), (\triangle, 2), (\triangle, 3), (\square, 1), (\square, 2), (\square, 3)\right\}$ 
			\item 	$\{0, 1\}^3 = \{0,1\} \times \{0,1\} \times \{0,1\} = \left\{(0, 0, 0), (0, 0, 1), (0, 1, 0), (0, 1, 1), (1, 0, 0), (1, 0, 1), (1, 1, 0), (1, 1, 1)\right\} $
		\end{itemize}
		
		\pause
	   Im Allgemeinen gilt $ A \times B \neq B \times A $\\
	   Für jede Menge $M$ gilt: $ M \times \emptyset = \pause \emptyset \times M = \emptyset$
	\end{exampleblock}
\end{frame}

\begin{frame}{Relationen}
	\begin{block}{Definition}
		Für Mengen $A$ und $B$ heißt eine Teilmenge 
		$$R \subseteq A \times B$$
		von Paaren $(a,b)$ eine \textbf{Relation} auf $A \times B$. \\
		\smallskip
		$R$ wird meistens durch eine \emph{Set-Comprehension} ($\set{... \vphantom{(} \Mid ......}$) angegeben. \\
		\smallskip
		Wir schreiben für $a \in A$, $b \in B$ 
		$$a \mathrel{R} b, $$
		wenn sie in Relation R zueinander stehen, also $(a, b) \in R$.
	\end{block}
	
	\pause
	\begin{exampleblock}{Beispiel} 
		$\leq$ ist eine Relation auf $\N_0 \times \N_0 $. \\
		Also: $ \text{$\leq$} = \set{(m, n) \Mid m \leq n } = \{(0, 0), (0, 1), (1, 1), (0, 2), ...\} $
	\end{exampleblock}
\end{frame}

\begin{frame}{Relationen}	
	\begin{exampleblock}{Beispiel}
		$ P = \Z \times \N_+ \times \N_0 $ \\
		$ \normalvar{\sim} = \set{\tuple{n, m, r} \in P \Mid n \* m = r } \subseteq P $ \\[0.5em]
		\pause
		$ \normalvar{\sim} = \set{\tuple{0, 1, 0}, \tuple{0, 2, 0}, \tuple{1, 1, 1}, \tuple{1, 2, 2}, \tuple{6, 7, 42}, ...} $ \\[0.5em]
		$ \tuple{-1, 1, -1}, \tuple{42, 0, 0} \pause \notin \normalvar{\sim} $\\[1em]
		\pause
		\impl Immer auf die \textbf{Grundmengen} achten!
	\end{exampleblock}
\end{frame}

\section{Eigenschaften von Relationen}

\begin{frame}{Eigenschaften von Relationen}
	\begin{block}{Totalität}
		Eine Relation $R \subseteq A \times B$ heißt \textbf{linkstotal}, wenn es für \emph{jedes} $a \in A$ ein zugehöriges $b \in B$ gibt, sodass $$(a,b) \in R.$$ \textbf{Rechtstotal} analog (für jedes $b$ ein $a$).
	\end{block}
	
	\pause
	\begin{block}{Eindeutigkeit}
		Eine Relation $R \subseteq A \times B$ heißt \textbf{linkseindeutig}, wenn für jedes $b \in B$ und $a_1, a_2 \in A$ gilt: $$\text{Wenn } \quad (a_1,b) \in R \text{ und } (a_2,b) \in R, \quad \text{dann} \quad a_1 = a_2.$$ \\
		\textbf{Rechtseindeutig} analog $\left(\text{wenn } (a, b_1) \text{ und } (a, b_2) \in R, \text{dann } b_1 = b_2 \right)$.
	\end{block}
\end{frame}

\subsection{Aufgabe 1}
\begin{frame}{Aufgabe 1 Teil 1} % (WS 2010)
	Es sei $A$ die Menge aller Kinobesucher in einer Vorstellung und $B$ die Menge aller Sitzplätze. Die Relation $R$ ordnet den Kinobesuchern die Sitzplätze zu:
	$$ R \subseteq A \times B$$
	Was bedeutet es im Kino, wenn $R$ linkstotal, linkseindeutig, rechtstotal, rechtseindeutig ist?
	\smallskip
		
	\pause
	\begin{tabular}{@{\hspace{-3pt}}r@{\ \ }l}
		\emph{linkstotal}: & jedem Kinobesucher wird mindestens ein Sitzplatz zugeteilt \\
		\pause
		\emph{linkseindeutig}: & jeder Sitzplatz ist von höchstens einem Kinobesucher belegt \\
		\pause
		\emph{rechtstotal}: & jeder Sitzplatz ist von mindestens einem Kinobesucher belegt \\ 
		\pause
		\emph{rechtseindeutig}: & kein Kinobesucher belegt mehr als einen Sitzplatz
	\end{tabular}
\end{frame}

\begin{frame}
	\begin{block}{Was ihr nun wissen solltet}
		\begin{itemize}
			\item wie man mit Mengen umgeht
			\item wie man mit noch mehr Mengen umgeht
			\item was Paare sind
			\item wie man mehr als zwei Elemente geordnet zusammenbringt (Tupel)
			\item wie man mit Relationen Chaos in die Ordnung bringt
		\end{itemize}
	\end{block}

	\begin{block}{Was nächstes Mal kommt}
		\begin{itemize}
			\item wie Funktionen funktionieren
			\item \textcolor{white}{.}\!% sHAME ON YOU, LATEX!!! I FUCKING HATE YOU!
				  \word{ada}: eure ersten Wörter als Informatiker 
		\end{itemize}
	\end{block}
\end{frame}

\only<handout:0>{\slideThanks}

%% Letzte Seite 
\xkcdframevert{2042}{Danke für eure Aufmerksamkeit! \smiley}{2.5}

\only<beamer:0>{\slideThanks}

\end{document}
