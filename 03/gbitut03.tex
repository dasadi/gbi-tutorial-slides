%beamer

% Comment/uncomment this line to toggle handout mode
\newcommand{\handout}{}

\input{../framework/PraeambelTut.tex}

\morescalingdelimiters

\begin{document}
\starttut{3}

\section{Organisatorisches}

\begin{frame}{Zu Übungsblatt \#1}
	Schnitt: \quad 11,8 / 17~P

	\begin{itemize}[<+->]
		\item 19 von 22 Tutanden haben etwas abgegeben. Weiter so!
		\item Die Korrektur und die Musterlösung findet ihr im ILIAS-Aufgaben-Objekt
		\item Nur die Person, die das Übungsblatt abgegeben hat, bekommt die Rückmeldung \impl tauscht euch aus!
		\item Ihr habt alle pünktlich abgegeben :)
	\end{itemize}
\end{frame}

\begin{frame}{Zu Übungsblatt \#1}	
	\begin{itemize}[<+->]
		\item Erstmal: Volle Punktzahl ist utopisch. Fehler macht man, um draus zu lernen – dazu sind ÜBs da
		\item Für viele viele viele falsche Notationen habe ich noch nichts abgezogen - das wird zukünftig nicht mehr so sein!
	\end{itemize}
\end{frame}

\begin{frame}{Zu Übungsblatt \#1}
	Die häufigsten Fehler:
	\begin{itemize}[<+->]
		\item Wenn in der Aufgabenstellung steht: \textit{Geben Sie ein $x$ an sodass gilt: ...}, dann genügt es ein solches $x$ anzugeben
		\implitem Keine Begründung gefordert!
		\item \textbf{All-Aussagen}, also Aussagen mit \textit{Für alle $x$ mit ... gilt ...} könnt ihr mit \textbf{einem Gegenbeispiel} widerlegen.
		\item ...aber ihr könnt sie nicht mit nur einem Beispiel beweisen. Ein Beweis für eine All-Aussage muss für alle solche Elemente gelten!
		\item \textbf{Existenz-Aussagen}, also Aussagen mit \textit{Es existiert ein $x$ sodass ... gilt} könnt ihr mit einem Beispiel beweisen.
		\item ...aber wenn ihr sie widerlegen wollt, müsst ihr das für alle möglichen $x$ tun.
		\item Gebt nur eine Lösung an. Wenn ihr mehrere angebt und eine davon falsch ist, muss ich die falsche Werten.
		\implitem Gebt nur so viel an wie gefordert ist, ihr könnt dafür nur Abzug bekommen!
	\end{itemize}
\end{frame}

\begin{frame}{Zu Übungsblatt \#1}
	Die häufigsten Fehler:
	\begin{itemize}[<+->]
		\item ``Genau-dann-wenn''-Aussagen beschreiben Äquivalzen. Ihr müsst diese Aussagen in beiden Richtungen oder mithilfe von Äquivalenzen zeigen.
		\item \textit{Erinnerung: $ A \Leftrightarrow B $ gilt genau dann wenn $A \Rightarrow B$ und $A \Leftarrow B$ gelten}
		\item Wenn in der Aufgabenstellung steht dass eine Aussage wahr bzw. falsch ist, dann ist sie das auch.
		\item Definiert alle Variablen, die ihr benutzt! \\
			z.B. in Aufgabe 1.2a): \textit{Sei $A=\emptyset$ und $B$ eine beliebige Menge}
		\item Sätze aus der Vorlesung müsst ihr nicht abschreiben, der Name reicht, z.B. $ a + b \stackrel{kommutativ}{=} b + a$
		\item Was sind notwendige und hinreichende Bedingungen?
		\item Für $A \Rightarrow B$ heißt $A$ eine hinreichende Bedingung für $B$ und $B$ eine notwendige Bedingung für $A$
		\item Eine Bedingung ist notwendig \textbf{und} hinreichend wenn sie äquivalent ist.
	\end{itemize}
\end{frame}

\begin{frame}{Zu Übungsblatt \#1}
	Die häufigsten Fehler:
	\begin{itemize}[<+->]
		\item Unterscheidet Mengen und Aussagen! \\
		Passt mit euren \textbf{Operatoren} auf: Für Mengen $A,B$ sind $A \Rightarrow B$, $A \Leftarrow B$ und $A \Leftrightarrow B$ nicht definiert.
		\implitem Ihr könnt euch mit $x \in A \Rightarrow x \in B$, $A \subseteq B$, ... Abhilfe schaffen.
		\item Für eine Menge $A$ haben wir $\bar{A}$ / $\neg A$ / das ``allgemeine'' Komplement nicht definiert.
		\implitem Stattdessen $x \not\in A$ benutzen
		\item Unterscheidet $\in$ und $\subseteq$! \\
			Insbesondere ist $A$ mit $A \subseteq 2^{\mathbb{N}_0}$ eine Menge von Mengen von natürlichen Zahlen und $B$ mit $B \in 2^{\mathbb{N}_0}$ eine Menge von natürlichen Zahlen.
		\item Die \textit{set comprehension} funktioniert so: \\
			Beispiele: $\{x^2 | x \in \mathbb{N} \text{ ist eine Primzahl }\}$ und $\{x \in \mathbb{Z} : 2 \cdot x \text{ ist eine Quadratzahl }\}$
	\end{itemize}
\end{frame}

\begin{frame}{Zu Übungsblatt \#1}
	Zu guter Letzt:
	\begin{itemize}[<+->]
		\item Bearbeitet die Aufgaben selbstständig oder mit eurem Abgabepartner
		\item Bleibt dabei! Übung macht den Meister
	\end{itemize}
\end{frame}


\section{Was beim letzten Mal geschah...}

\framePrevEpisode

\begin{frame}{Kahoot!}
	\begin{itemize}[<+->]
		\item Kahoot! ist ein anonymes Online-Quiz
		\item Ihr bekommt Punkte für schnelles und richtiges raten
		\item Ich schalte das Quiz frei und ihr könnt über \url{https://kahoot.it} beitreten
		\item Das Kahoot! könnt ihr euch später nochmal unter diesem Link angucken: \\
			\url{https://create.kahoot.it/share/gbi-woche-3-einstieg/c0638af4-aaa2-4892-8916-a2eab442312d}
	\end{itemize}
\end{frame}


%\subsection{Wahr oder Falsch?}
%\begin{frame}[t]{Wahr oder Falsch?}
%	\FalseQuestionE{Eine Funktion muss linkseindeutig sein.}{\impl Eine Funktion muss rechtseindeutig \textbf{und linkstotal} sein.}
%	\TrueQuestion{Eine injektive Funktion ist linkseindeutig.}
%	\TrueQuestion{Eine surjektive Funktion ist rechtstotal.}
%	\FalseQuestion{Jede Relation ist eine Funktion.}
%	\TrueQuestion{Jede Funktion ist eine Relation.}
%\end{frame}

%\begin{frame}[t]{Wahr oder Falsch?}
%	\delimitershortfall=0pt
%	\FalseQuestionE{$\word{aaba} \in \{ \word a, \word b\}^2\times\{\word a,\word b\}^2$}{Aber: $(\word{aa}, \word{ba}) \in \{\word a, \word b\}^2\times\{\word a, \word b\}^2$}
%	\FalseQuestionE{$\setsize{\{ \eps \}} = 0$}{$\{ \eps \} \neq \{\}$}
%	\FalseQuestionE{Wenn $\eps \in A^*$, dann $\size{\eps^3} = 3$.}{$\size{\eps^3} = 0$; außerdem gilt $\eps \in A^*$  immer, also für alle Alphabete $A$.}
%\end{frame}

\input{../Bloecke/BinaereOperationen.tex}

\section{Induktive Definitionen}

\begin{frame}{Zum Aufwärmen: Domino}
	Drei Dominosteine sind mit gleichem Abstand (kleiner halbe Größe) in einer Reihe aufgestellt. Wir stoßen den ersten Stein der Reihe in Richtung des zweiten Steins um. \\
	Wird der dritte Stein umfallen? Kann man das (einfach) beweisen? \\[1em]
	\pause
	Nun stehen (abzählbar) unendlich viele Dominosteine wie oben hintereinander. Wieder stoßen wir den ersten Stein um. \\
	Wird jeder Stein irgendwann umfallen? Kann man das (einfach) beweisen? \\[1em]
	\pause
	Werden irgendwann alle Steine umgefallen sein? \\
	\pause Nein, denn für jeden Zeitpunkt können wir (mindestens) einen Stein angeben, der noch nicht umgefallen ist.\\
	Wir sehen also: Alle Steine fallen um, aber es sind niemals alle umgefallen.
\end{frame}

\newcommand{\Fib}{\mathcal{F}\hspace{-1pt}}

\begin{frame}{Induktive Definitionen}
	\begin{exampleblock}{Beispiel: \emph{Fibonacci}-Reihe}
		\begin{align*}
		\Fib_0 &:= 0 \\
		\Fib_1 &:= 1 \\
		\text{Für } n \geq 0: \quad \Fib_{n+2} &:= \Fib_{n+1} + \Fib_n 		
		\end{align*}
		\pause
		\begin{table}
			\centering
			\begin{tabular}{|c|c|c|c|c|c|c|c|c|c|c|}
				\hline
				$n$ & 0 & 1 & 2 & 3 & 4 & 5 & 6 & 7 & 8 & 9 \\ \hline
				$\Fib_n$ & 0 & 1 & 1 & 2 & 3 & 5 & 8 & 13 & 21 & 34 \\ \hline
			\end{tabular}
		\end{table}
		
		\pause
		\textbf{Wohldefiniertheit}: 
		\begin{itemize}
			\item Für alle Fälle etwas definieren 
			\item Nicht für einen Fall was Widersprüchliches definieren
		\end{itemize}
		
	\end{exampleblock}
	
\end{frame}

\input{../Bloecke/Induktion.tex}

%\input{../Bloecke/Aussagenlogik.tex}

%\input{../Bloecke/Aussagenlogik2}

\begin{frame}	
	\begin{block}{Was ihr nun wissen solltet}
		\begin{itemize}
			\item Welche Fehler man auf dem Übungsblatt lieber vermeidet \smiley
			\item Wie vollständige Induktion geht
%			\item Aussagenlogik: Syntax und Semantik
%			\item Wie man wahre Aussagen konstruiert \qquad (\#NoFakeNews!) 
		\end{itemize}
	\end{block}
	
	\begin{block}{Was nächstes Mal kommt}
		\begin{itemize}
			\item Aussagenlogik
			\item Beweisbarkeit
			\item Formale Sprachen
%			\item Aus \word2 mach \word{10}: Übersetzungen
		\end{itemize}
	\end{block}
\end{frame}

\xkcdframe{589}{Danke für eure Aufmerksamkeit! \smiley}{2}
\slideThanks

\end{document}
