%beamer

% Comment/uncomment this line to toggle handout mode
\newcommand{\handout}{}

\input{../framework/PraeambelTut.tex}

\morescalingdelimiters

\begin{document}
\starttut{8}

\mycomment{
	\begin{frame}{Schwarzes Brett}
		\textbf{Bonusaufgaben} von Blatt 4 auch im \ILIAS im Ordner zu Tutorium~\mytutnumber\ ganz unten einreichbar! \\
		\medskip
		(\impl Weniger Chaos als per Mail. \smiley)
	\end{frame}
}

\section{Rückblick}

\begin{frame}{Zu Übungsblatt \#6}
	Schnitt: \quad 17,7 / 23~P

	\begin{itemize}[<+->]
		\item 19 von 23 Tutanden haben etwas abgegeben
		\item Die Korrektur und die Musterlösung findet ihr im ILIAS-Aufgaben-Objekt
		\item Nur die Person, die das Übungsblatt abgegeben hat, bekommt die Rückmeldung \impl tauscht euch aus!
		\item Ihr habt alle pünktlich abgegeben :)
	\end{itemize}
\end{frame}

\begin{frame}{Zu Übungsblatt \#6}
	Die häufigsten Fehler:
	\begin{itemize}[<+->]
		\item Wie immer: Vorsicht bei der Notation!
		\item Aufgabe 6.1a): Bei Huffman-Bäumen muss man die beiden freien Knoten verbinden, die die niedrigste Summe bilden.
		\implitem Wenn Knoten mit den Anzahlen $4$, $3$ und $3$ frei sind, muss man die Dreien verbinden.
		\item Aufgabe 6.2a): Es war gefordert, ``die Definition von $P_c$ höchstens einmal'' pro Schritt anzuwenden. Beim Rest dürft ihr abkürzen.
		\implitem Spart euch Schreibarbeit und mit Korrekturarbeit \smiley
		\item Aufgabe 6.2c): Bei einer Induktion müsst ihr in GBI \textbf{immer} I.A., I.V. und I.S. angeben
		\item[] in der I.V. nehmt ihr die zu beweisende Aussage für \textbf{ein} festes aber beliebiges $n \in \mathbb{N}$
	\end{itemize}
\end{frame}

\begin{frame}{Zu Übungsblatt \#6}
	Die häufigsten Fehler:
	\begin{itemize}[<+->]
		\item Aufgabe 6.3c): Übersetzungen zwischen Zweierkomplementdarstellungen verschiedener Längen:
			\begin{threealign}
				Ü_k : A^\ell &\functionto& A^{\ell + k} \qquad (\ell \in \mathbb{N}_+, k \in \mathbb{N}_0) \\
				w &\mapsto& \begin{cases}
					\word 0^kw, & w(0) = \word 0 \\
					\word 1^kw, & w(0) = \word 1
				\end{cases}
			\end{threealign}
		\item[] Der Wert der durch $bw \in A^\ell$ mit $b\in A, w \in A^{\ell - 1}$ im ZKPL dargestellten Zahl ist $-2^{|w|}\cdot num_2(b) + Num_2(w)$
		\item[] Der Wert der durch $b^kbw$ im ZKPL dargestellten Zahl ist $-2^{|w| + k}\cdot num_2(b) + Num_2(b^kw)$
	\end{itemize}
\end{frame}

\begin{frame}{Zu Übungsblatt \#6}
	\begin{itemize}
		\item[] Der Wert der durch $bw \in A^\ell$ mit $b\in A, w \in A^{\ell - 1}$ im ZKPL dargestellten Zahl ist $-2^{|w|}\cdot num_2(b) + Num_2(w)$
		\item[] Der Wert der durch $b^kbw$ im ZKPL dargestellten Zahl ist $-2^{|w| + k}\cdot num_2(b) + Num_2(b^kw)$
		\implitem Es gilt:
		\begin{align*}
			& -2^{|w| + k}\cdot num_2(b) + Num_2(b^kw) \\
			=& -2^{|w| + k}\cdot num_2(b) + Num_2(b^k) \cdot 2^{|w|} + Num_2(w) \\
			=& -2^{|w| + k}\cdot num_2(b) + \left(2^k-1\right) \cdot num_2(b) \cdot 2^{|w|} + Num_2(w) \\
			=& \left(-2^{|w| + k} + \left(2^k-1\right)\cdot 2^{|w|}\right) \cdot num_2(b) + Num_2(w) \\
			=& \left(-2^{|w| + k} + 2^{|w| + k} -2^{|w|}\right) \cdot num_2(b) + Num_2(w) \\
			=& -2^{|w|} \cdot num_2(b) + Num_2(w)
		\end{align*}
	\end{itemize}
\end{frame}

\begin{frame}{Zu Übungsblatt \#6}
	Die häufigsten Fehler:
	\begin{itemize}[<+->]
		\item Aufgabe 6.5a): Man sagt ``höchstwertiges'' Bit, nicht ``vorderstes''
		\item Aufgabe 6.5b): Geschenk vom Worschnachtsmann
	\end{itemize}
\end{frame}

\begin{frame}{Übungsschein, die Zweite}
	Die erste Hälfte ist rum. Zeit für die Zweite.
	\begin{itemize}[<+->]
		\item Jeder, der es ernsthaft versucht hat, hat die erste Hälfte geschafft. Weiter so!
		\item Auf Blatt 7 bis 12 müsst ihr wieder mindestens 50\% der Punkte erreichen
		\item Ihr dürft ab sofort nur noch alleine abgeben
		\implitem knappere Korrekturen
	\end{itemize}
\end{frame}

\begin{frame}{Klausur}
	\begin{itemize}[<+->]
		\item Termin: 06. März 2021 von 09:00 Uhr bis 11:00
		\item Einlass: Vermutlich bis zu einer Stunde früher, ihr werdet eine Uhrzeit zugewiesen bekommen
		\item Ort: In der Gartenhalle (beim Zoo) und in den Zelten (vor dem Audimax)
		\item
	\end{itemize}
\end{frame}

\framePrevEpisode

\begin{frame}{Kahoot!}
	\begin{itemize}[<+->]
		\item Kahoot! ist ein anonymes Online-Quiz
		\item Ihr bekommt Punkte für schnelles und richtiges raten
		\item Ich schalte das Quiz frei und ihr könnt über \url{https://kahoot.it} beitreten
		\item Das Kahoot! könnt ihr euch später nochmal unter diesem Link angucken: \\
			\url{https://create.kahoot.it/share/gbi-woche-8-einstieg/7a48a425-5e21-414c-b4ed-8d5418cb7d30}
	\end{itemize}
\end{frame}

%\begin{frame}{Achtung, Probeklausur!}
%	\textbf{Mi, den 16.01.19} \\
%	gewohnte GBI-Zeit und -Ort \\
%	Zählt \textbf{nicht}! {\small (keine Bonuspunkte, keine Prüfungsnote)}\\
%	Wird aber trotzdem von mir korrigiert. \smiley
%\end{frame}

\mycomment{
	\begin{frame}{Rückblick: MIMA}
		\begin{itemize}[<+->]
			\item Ein idealisierter Prozessor
			\item Einfach zu verstehen, aufwändig zu programmieren
			\item Hardware-Details beachten: Keine negativen Konstanten mit LDC möglich!
			\item Programme sind oftmals mit \enquote{Bit-Magie} einfacher und kürzer (aber auch schwerer zu verstehen)
		\end{itemize}
	\end{frame}
}

\begin{frame}{Rückblick: Kontextfreie Grammatiken}
	\begin{itemize}[<+->]
		\item Ein Vier-Tupel: $G = (N, T, S, P)$
		\item[] z.B. $N=\left\{X,Y\right\}, T=\left\{\word{a}, \word{b}\right\}, S=X, P=\left\{X \to \word a Y, Y \to Y \mid \word b \right\}$
		\item Produktionen definieren Ersetzungen eines Nichtterminals mit Wörtern über $N \cup T$
		\item Wir wenden Produktionen in Ableitungsschritten an: $v \Rightarrow w, v \Rightarrow^* w$
		\item[] z.B. $X \derives \word a Y$, $Y \derives Y$, $Y \derives^{2020} Y$, $\word{aaa}Y\word{bbb} \derives \word{aaabbbb}$, $X \derives^2 \word{ab}$ oder $X \derives^\ast \word{ab}$
		\item $L(G)$ sind alle aus $S$ ableitbaren Wörter über $T$ (die also nur aus Terminalsymbolen bestehen)
	\end{itemize}
\end{frame}

\section{Fortsetzung kontextfreie Grammatiken}

%\begin{frame}[t]{Wahr oder Falsch?}
%	\Socrative
%	Sei $G=(\{X,Y\},\, \{\word a, \word b\},\, X,\, P)$ eine kontextfreie Grammatik. \\
%	\FalseQuestion{Die Produktion $XY \to \word a$ könnte eine gültige Produktion sein.}
%	\FalseQuestion{Die Produktion $\word a \to XY$ könnte eine gültige Produktion sein.}
%	\TrueQuestionE{Die Produktion $X \to X\word aX$ könnte eine gültige Produktion sein.}{}
%	\TrueQuestionE{Wenn $X \to w$ eine gültige Produktion ist, dann gilt $X \derives^* w$.}{}
%	\FalseQuestionE{Wenn $X \derives^* w$ gilt, dann ist $X \to w$ eine Produktion in P.}{Sei $P=\{X \to XX \mid \word a\}$. Dann gilt $X \derives^* X\word a$, aber $X \to X\word a \notin P$.}
%\end{frame}

\input{../Bloecke/Grammatiken.tex}

% Nicht unbed. nötig. Je nach Zeitbudget.
\input{../Bloecke/Relationen2.tex}

%\input{../Bloecke/Praedikatenlogik.tex}

%\input{../Bloecke/Praedikatenlogik2}

\begin{frame}	
	\begin{block}{Was ihr nun wissen solltet}
		\begin{itemize}
			\item Wie man Sprachen mit Grammatiken beschreiben kann
			\item Wie man Grammatiken arbeiten kann
			\item Welche Eigenschaften Relationen haben können
			%\item Wie Prädikatenlogische Formeln aufgebaut sind
			%\item Wie man damit präzise Aussagen trifft
			%\item Wie man sie auswertet
		\end{itemize}
	\end{block}
	
	\begin{block}{Was nächstes Mal kommt}
		\begin{itemize}
			\item Wie man damit prädikatenlogische Formeln auswertet
			\item Wie man damit präzise Aussagen trifft
			\item Alles korrekt? –- Beweise mit dem Hoare-Kalkül
		\end{itemize}
	\end{block}
\end{frame}

%\begin{frame}[plain]
%	\begin{center}
%		\large
%		Nicht vergessen, Kinder: Nächste Woche findet noch ein Tutorium statt! \smiley
%	\end{center}
%\mycomment{	\bigskip
%	Für alle die nicht kommen: \\
%	Frohe Weihnachten und einen guten Start in das neue Jahr!}
%\end{frame}

\xkcdframe{704}{Danke für eure Aufmerksamkeit! \smiley}{2.5}
%\lastframe{0.50}{0}{xkcd/logic_principle_of_explosion.png}{https://www.xkcd.com/}
%\lastframe{0.50}{0}{xkcd/christmastree.png}{https://www.xkcd.com/835} % Dieses Jahr dürften noch genügend Tutanden beim nächsten Termin kommen
\slideThanks

\end{document}