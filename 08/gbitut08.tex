%beamer

% Comment/uncomment this line to toggle handout mode
\newcommand{\handout}{}

\input{../framework/PraeambelTut.tex}

\morescalingdelimiters

\begin{document}
\starttut{8}

\mycomment{
	\begin{frame}{Schwarzes Brett}
		\textbf{Bonusaufgaben} von Blatt 4 auch im \ILIAS im Ordner zu Tutorium~\mytutnumber\ ganz unten einreichbar! \\
		\medskip
		(\impl Weniger Chaos als per Mail. \smiley)
	\end{frame}
}

\section{Rückblick}

\begin{frame}{Zu Übungsblatt \#6}
	Schnitt: \quad 9,5 / 17~P

	\begin{itemize}[<+->]
		\item 16 von 23 Tutanden haben etwas abgegeben
		\item Die Musterlösung findet ihr im \ILIAS unter Übungsblätter
		\item Korrekturen gibt es jetzt!
		\item Ihr habt alle pünktlich abgegeben :)
		\item Es gab eine Änderung der Punkte: Aufgabe 6.2b) wurde Zusatzaufgabe mit 3 Punkten
		\item Viele haben bei Aufgabe 6.3 nichts / Teillösungen abgegeben \impl etwas niedriger Durchschnitt
	\end{itemize}
\end{frame}

\begin{frame}{Zu Übungsblatt \#6}
	Die häufigsten Fehler:
	\begin{itemize}[<+->]
		\item Wie immer: Vorsicht bei der Notation!
		\item Aufgabe 6.3: $x \elem w$ ist \textbf{keine} korrekte Notation, wenn $w$ ein Wort ist!!!
		\item Aufgabe 6.2a): Ein Huffmann-Baum hat an den Blättern \textbf{immer} das Zeichen \& die Häufigkeit stehen!
		\item Aufgabe 6.2b): Bei einem ternären Alphabet kommt es bei $\abs{M_0}$ zu einem Problem
		\implitem Anpassung notwendig
		\item Aufgabe 3: Jetzt nochmal gemeinsam zum Üben!
	\end{itemize}
\end{frame}

\begin{frame}{Übungsschein, die Zweite}
	Die erste Hälfte ist rum. Zeit für die Zweite.
	\begin{itemize}[<+->]
		\item Fast jeder, der es ernsthaft versucht hat, hat die erste Hälfte geschafft. Weiter so!
		\item Es gab in der ersten Hälfte 116,5~P \impl Bestehen ab 55~P
		\item Auf Blatt 7 bis 12 müsst ihr wieder mindestens 50\% der Punkte erreichen
		\item Ihr dürft ab sofort nur noch alleine abgeben
		\implitem knappere Korrekturen mit weniger Erklärungen
	\end{itemize}
\end{frame}

\mycomment{%Not yet known
	\begin{frame}{Klausur}
		\begin{itemize}[<+->]
			\item Termin: 06. März 2021 von 09:00 Uhr bis 11:00
			\item Einlass: Vermutlich bis zu einer Stunde früher, ihr werdet eine Uhrzeit zugewiesen bekommen
			\item Ort: In der Gartenhalle (beim Zoo) und in den Zelten (vor dem Audimax)
			\item
		\end{itemize}
	\end{frame}
}

\framePrevEpisode

\begin{frame}{Kahoot!}
	\begin{itemize}[<+->]
		\item Kahoot! ist ein anonymes Online-Quiz
		\item Ihr bekommt Punkte für schnelles und richtiges raten
		\item Ich schalte das Quiz frei und ihr könnt über \url{https://kahoot.it} beitreten
		\item Das Kahoot! könnt ihr euch später nochmal unter diesem Link angucken: \\
			\url{https://create.kahoot.it/share/gbi-woche-8/f4c8d9b1-38fe-41e1-97f9-9576703e02ae}
	\end{itemize}
\end{frame}

%\begin{frame}{Achtung, Probeklausur!}
%	\textbf{Mi, den 16.01.19} \\
%	gewohnte GBI-Zeit und -Ort \\
%	Zählt \textbf{nicht}! {\small (keine Bonuspunkte, keine Prüfungsnote)}\\
%	Wird aber trotzdem von mir korrigiert. \smiley
%\end{frame}

\begin{frame}{Rückblick: MIMA}
	\begin{itemize}[<+->]
		\item Ein idealisierter Prozessor
		\item Einfach zu verstehen, aufwändig zu programmieren
		\item Hardware-Details beachten: Keine negativen Konstanten mit LDC möglich!
		\item Programme sind oftmals mit \enquote{Bit-Magie} einfacher und kürzer (aber auch schwerer zu verstehen)
	\end{itemize}
\end{frame}
\begin{frame}{Noch offen: Klammerausdrücke}
	A long, long time ago, in a land far away:\\
	\medskip
	Formale Sprachen angeben durch $\set{}, \*, {}^+, {}^*$...\\
	Was ist mit der \textbf{Sprache aller gültigen Klammerausdrücke}? Können wir die auch so angeben?\\[1em]
	\pause
	\impl Jetzt wissen wir: \textbf{Nein}, das geht nicht! (Siehe VL)\\[1em]
	
	\begin{figure}[H]
		\centering
		\includegraphics[scale=0.7]{xkcd/(.png}
		\vspace{-7pt}
		\caption{ \texttt{\url{https://xkcd.com/859/}} }
	\end{figure}
\end{frame}

\section{Dokumente}
\begin{frame}{Dokumente}
	\begin{itemize}
		\item Dokumente bestehend aus drei Aspekten
		\pause
		\item \textbf{Inhalt}: Was steht eigentlich da?
		\item \textbf{Struktur}: Wie ist der Text gegliedert/strukturiert?
		\item \textbf{Form}: Wie ist das Erscheinungsbild ?
	\end{itemize}
\end{frame}

\input{../Bloecke/Grammatiken.tex}

\mycomment{%maybe next tut
	\begin{frame}{Rückblick: Kontextfreie Grammatiken}
		\begin{itemize}[<+->]
			\item Ein Vier-Tupel: $G = (N, T, S, P)$
			\item[] z.B. $N=\left\{X,Y\right\}, T=\left\{\word{a}, \word{b}\right\}, S=X, P=\left\{X \to \word a Y, Y \to Y \mid \word b \right\}$
			\item Produktionen definieren Ersetzungen eines Nichtterminals mit Wörtern über $N \cup T$
			\item Wir wenden Produktionen in Ableitungsschritten an: $v \Rightarrow w, v \Rightarrow^* w$
			\item[] z.B. $X \derives \word a Y$, $Y \derives Y$, $Y \derives^{2020} Y$, $\word{aaa}Y\word{bbb} \derives \word{aaabbbb}$, $X \derives^2 \word{ab}$ oder $X \derives^\ast \word{ab}$
			\item $L(G)$ sind alle aus $S$ ableitbaren Wörter über $T$ (die also nur aus Terminalsymbolen bestehen)
		\end{itemize}
	\end{frame}
}
%\begin{frame}[t]{Wahr oder Falsch?}
%	\Socrative
%	Sei $G=(\{X,Y\},\, \{\word a, \word b\},\, X,\, P)$ eine kontextfreie Grammatik. \\
%	\FalseQuestion{Die Produktion $XY \to \word a$ könnte eine gültige Produktion sein.}
%	\FalseQuestion{Die Produktion $\word a \to XY$ könnte eine gültige Produktion sein.}
%	\TrueQuestionE{Die Produktion $X \to X\word aX$ könnte eine gültige Produktion sein.}{}
%	\TrueQuestionE{Wenn $X \to w$ eine gültige Produktion ist, dann gilt $X \derives^* w$.}{}
%	\FalseQuestionE{Wenn $X \derives^* w$ gilt, dann ist $X \to w$ eine Produktion in P.}{Sei $P=\{X \to XX \mid \word a\}$. Dann gilt $X \derives^* X\word a$, aber $X \to X\word a \notin P$.}
%\end{frame}


%\input{../Bloecke/Praedikatenlogik.tex}

%\input{../Bloecke/Praedikatenlogik2}

\begin{frame}	
	\begin{block}{Was ihr nun wissen solltet}
		\begin{itemize}
			\item Was kontextfreie Grammatiken sind
			\item Wie man Sprachen mit Grammatiken beschreiben kann
			\item Wie man Grammatiken arbeiten kann
			%\item Welche Eigenschaften Relationen haben können
			%\item Wie Prädikatenlogische Formeln aufgebaut sind
			%\item Wie man damit präzise Aussagen trifft
			%\item Wie man sie auswertet
		\end{itemize}
	\end{block}
	
	\begin{block}{Was nächstes Mal kommt}
		\begin{itemize}
			\item Weitere Eigenschaften von Relationen
			\item Wie Prädikatenlogische Formeln aufgebaut sind
			\item Wie man prädikatenlogische Formeln auswertet
			\item Wie man damit präzise Aussagen trifft
		\end{itemize}
	\end{block}
\end{frame}

%\begin{frame}[plain]
%	\begin{center}
%		\large
%		Nicht vergessen, Kinder: Nächste Woche findet noch ein Tutorium statt! \smiley
%	\end{center}
%\mycomment{	\bigskip
%	Für alle die nicht kommen: \\
%	Frohe Weihnachten und einen guten Start in das neue Jahr!}
%\end{frame}

\xkcdframe{704}{Danke für eure Aufmerksamkeit! \smiley}{2.5}
%\lastframe{0.50}{0}{xkcd/logic_principle_of_explosion.png}{https://www.xkcd.com/}
%\lastframe{0.50}{0}{xkcd/christmastree.png}{https://www.xkcd.com/835} % Dieses Jahr dürften noch genügend Tutanden beim nächsten Termin kommen
\slideThanks

\end{document}