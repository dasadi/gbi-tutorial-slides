%beamer

% Comment/uncomment this line to toggle handout mode
\newcommand{\handout}{}

\input{../framework/PraeambelTut.tex}

\morescalingdelimiters

\begin{document}
\starttut{14}

\section{Rückblick}

\begin{frame}{Zu Übungsblatt \#11}
	Schnitt: \quad 5,9 / 16~P

	\begin{itemize}[<+->]
		\item 10 von 23 TutandInnen haben etwas abgegeben
		\item Die Musterlösung findet ihr im \ILIAS unter Übungsblätter
		\item Korrekturen gibt es jetzt!
		\item Ihr habt alle pünktlich abgegeben :)
		\item Die eine Online-Abgabe gebe ich dann noch online zurück
	\end{itemize}
\end{frame}

\begin{frame}{Zu Übungsblatt \#11}
	Die häufigsten Fehler:
	\begin{itemize}[<+->]
		\item Schreibt nichts zusätzliches hin, das nicht gefragt ist \impl nur Möglichkeit für Fehler
		\item[1c)] Beim Modifizieren war nur erlaubt: 
		\begin{itemize}
			\item hinzufügen \textbf{neuer Zustände mit Übergängen}
			\item entfernen nicht benötigter \textbf{Übergänge} von ursprünglichen Zuständen
		\end{itemize}
		\item[2a)] Es war \textbf{formale} Definition gefordert (Also Tupelschreibweise)
		\item[] Da für viele die Aufgabenstellung ungenau war \impl auch ohne formale Definition volle Punkte 
		\item[4a)] \textbf{Endlichkeit} der Zustandsmenge und Bandalphabet hat bei allen gefehlt
	\end{itemize}
\end{frame}

\begin{frame}{Organisation}
	\begin{itemize}[<+->]
		\item Übungsblatt 12 Rückgabe am: Freitag, 11.02.22, 11~:~30-12~:~00
		\item Zusätzliches Übungstutorium nächste Woche gewünscht? \impl Ja, 14.02.22, 12:00
		\item Erinnerung: Alle bei Übungsschein und Klausur angemeldet?
		\end{itemize}
\end{frame}

\framePrevEpisode


%\begin{frame}[t]{Wahr oder Falsch?}
%	\FalseQuestionE{Das (komplizierte) Master-Theorem kann man immer anwenden.}{ Nur bei rekursiven Algorithmen, bei denen das Problem in gleich große Teilprobleme aufgeteilt wird.}
%	\FalseQuestionE{Jeder Moore-Automat kann in einen Mealy-Automaten umgewandelt werden, der für jedes Wort die gleiche Ausgabe produziert.}{ Für das leere Wort kann ein Mealy-Automat niemals eine Ausgabe produzieren.}
%	\TrueQuestionE{Endliche Akzeptoren sind Moore-Automaten mit dem Ausgabealphabet $\{\word 0,\word 1\}$.}{}
%	\FalseQuestionE{Mit endlichen Automaten kann jede beliebige Sprache erkannt \\ werden.}{Tatsächlich ist die Menge der akzeptierbaren Sprachen sogar sehr eingeschränkt.}	
%\end{frame}
\begin{frame}{Asymptotisches Wachstum}
	\begin{Definition}
		Zwei Funktionen $f,g: \N_0 \to \R_0^+$ wachsen asymptotisch gleich schnell, wenn es zwei Konstanten $c, c' \in \R^+$ gibt, so dass gilt $$\exists n_0 \in \N_0: \ \forall n \geq n_0: \ c \* f(n) \leq g(n) \leq c' \* f(n) $$
		Man schreibt dafür 
		\begin{align*}
			f &\asymp g \\
			\textbf{oder} \quad f(n) &\asymp g(n) 
			% Hashtag: „n^2“ ist ein Term und keine Funktion etc... Man müsse doch [n \mapsto n^2] schreiben... VL erlaubt ersteres aber auch.
		\end{align*}
	\end{Definition} \pause
	Diese Relation ist eine Äquivalenzrelation!
\end{frame}


\input{../Bloecke/GrossO.tex}

\input{../Bloecke/Graphen}

\input{../Bloecke/Graphen2}

\begin{frame}{Übung: Adjazenzmatrix}
	\begin{columns}
		\column{0.4\linewidth}
		$$ A = 
		\begin{pmatrix} 
		1 & 1 & 0 & 1 & 0 & 0 \\ 
		0 & 0 & 0 & 1 & 0 & 0 \\ 
		0 & 1 & 1 & 1 & 0 & 0 \\ 
		0 & 0 & 1 & 0 & 0 & 0 \\
		1 & 0 & 0 & 0 & 0 & 0 \\
		0 & 0 & 0 & 0 & 0 & 0 \\
		\end{pmatrix} $$
		
		\column{0.4\linewidth}
		\only<7->{
			\begin{tikzpicture}[->,>=stealth,baseline=-5mm]
			\matrix[matrix of math nodes,nodes={draw,circle,minimum size=6mm,inner sep=2pt},row sep=13mm,column sep=10mm,ampersand replacement=\&]
			{
				|(0)| 0 \& |(1)| 1 \& |(2)| 2 \\
				|(4)| 4 \& |(3)| 3 \& |(5)| 5 \\
			};
			\draw  (0) -- (1);
			\draw  (0) -- (3);
			\draw  (4) -- (0);
			\draw  (1) -- (3);
			\draw  (2)  to [bend left] (3);
			\draw  (2) -- (1);
			\draw  (3) to [bend left] (2);
			\path  (2) edge [loop right] ();
			\path  (0) edge [loop left] ();
			\end{tikzpicture}
		}
	\end{columns}
	
	\bigskip
	Was kann man an der Adjazenzmatrix ablesen?
	\begin{itemize}
		\item Gerichtet oder ungerichtet? \\ \pause
		\impl gerichtet (weil $A$ nicht symm.) \pause
		\item Schlingen? \\ \pause
		\impl Auf der Diagonalen von $A$: \quad Knoten 0, 2 \pause
		\item Zusammenhängend? \\ \pause
		\impl Nein (5 ist isoliert). 
	\end{itemize}
\end{frame}


\input{../Bloecke/Graphen3Matrix.tex}

\begin{frame}	
	\begin{block}{Was ihr nun wissen solltet}
		\begin{itemize}
			\item Asymptotisches Wachstum und Laufzeitabschätzungen \impl mehr in Algo1
			\item Was sind Graphen
			\item Wie kann man Graphen darstellen?
			\item Ein einfacher Algorithmus in Graphen
		\end{itemize}
	\end{block}

\end{frame}

% TODO ?
%input{../Bloecke/StrukturelleInduktion}

%TODO replacement?
{\xkcdframevert{1185}{Danke für eure Aufmerksamkeit! \smiley}{2.5}}
%\thasse{\lastframe{0.5}{0}{xkcd/proofs_1724.png}{https://www.xkcd.com/1724/}}

\slideThanks

\end{document}