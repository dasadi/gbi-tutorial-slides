%beamer

% Comment/uncomment this line to toggle handout mode
\newcommand{\handout}{}

\input{../framework/PraeambelTut.tex}

\morescalingdelimiters

\begin{document}
\starttut{9}

\section{Rückblick}

\begin{frame}{Zu Übungsblatt \#7}
	Schnitt: \quad 15,6 / 20~P

	\begin{itemize}[<+->]
		\item 14 von 23 Tutanden haben etwas abgegeben
		\item Die Musterlösung findet ihr im \ILIAS unter Übungsblätter
		\item Korrekturen gibt es jetzt!
		\item Ihr habt alle pünktlich abgegeben :)
		\item Gut gemacht, das Blatt ist sehr gut ausgefallen!
	\end{itemize}
\end{frame}

\begin{frame}{Zu Übungsblatt \#7}
	Ein häufiger Fehler:
	\begin{itemize}[<+->]
		\item Aufgabe 7.3d): Fallunterscheidung für $m(0) = 1$ notwendig: 
		$$ M_{m'} = 
		\casesl{M_m \setminus \set{(0, m(0)), (m(0)-1, last(m))} \cup \set{(0,m(0)-1)}, & m(0) > 1 \\
				M_m, & m(0) = 1}$$
		\item Im Allgemeinen: Bei Definitionen \textbf{keine Pünktchen} („...“) verwenden 
		\item Im Allgemeinen: Bei Variablen genau angeben, aus welcher Menge: z.B. $a\elem \N_+$
	\end{itemize}
\end{frame}

\begin{frame}{Organisatorisches}
	\begin{itemize}[<+->]
		\item Bestehensgrenze Blatt 1-6: 52~P
		\item Übungsblatt 9: Abgabe auf 14.Januar 2022 verschoben\\
			\quad \impl Kleines Weihnachtsgeschenk
		\item Am 7. Januar 2022 aber schon wieder Vorlesungen\\
			\quad \impl in GBI ausnahmsweise online
		\item Besonders bei Einzelabgaben darauf achten, dass es nicht zu viele Ähnlichkeiten zwischen den Abgaben gibt
	\end{itemize}
\end{frame}

\mycomment{
	\begin{frame}{Zu Blatt \#3}
		Durchschnitt: \quad etwa \thassedaniel{58}{53}~\% der Punkte \\
		\begin{itemize}
			\item \textbf{Induktionen}: Schreibt mir bitte die Aussage hin, über die ihr die Induktion macht. Wenn IV falsch, hab ich sonst keinen Plan, was ihr zeigen wollt.
			\item \textbf{A3.1}: Schaut euch Injektivität/Surjektivität nochmal an...
			\item \textbf{A3.2}: Huffman-Bäume: Es werden IMMER die zwei KLEINSTEN Knoten verbunden. Auch „über Kreuz“. 
			\item \textbf{A3.3}: Terme mit zu vielen Pünktchen („...“) sind keine Definition. \\ Alles, was nicht rekursiv ist, muss falsch sein wegen Klammerausdrücken.
			\item \textbf{A3.6}: Induktion über $n = \size{w_1} = \size{w_2}$, also beide Wortlängen gleichzeitig, geht NICHT! (Gibt nämlich auch Wörter, wo beide Längen nicht gleich sind... :P)
			
		\end{itemize}
	\end{frame}
}

\framePrevEpisode

\begin{frame}{Rückblick: Kontextfreie Grammatiken}
	\begin{itemize}[<+->]
		\item Ein Vier-Tupel: $G = (N, T, S, P)$
		\item[] z.B. $N=\left\{X,Y\right\}, T=\left\{\word{a}, \word{b}\right\}, S=X, P=\left\{X \to \word a Y, Y \to Y \mid \word b \right\}$
		\item Produktionen definieren Ersetzungen eines Nichtterminals mit Wörtern über $N \cup T$
		\item Wir wenden Produktionen in Ableitungsschritten an: $v \Rightarrow w, v \Rightarrow^* w$
		\item[] z.B. $X \derives \word a Y$, $Y \derives Y$, $Y \derives^{2020} Y$, $\word{aaa}Y\word{bbb} \derives \word{aaabbbb}$, $X \derives^2 \word{ab}$ oder $X \derives^\ast \word{ab}$
		\item $L(G)$ sind alle aus $S$ ableitbaren Wörter über $T$ (die also nur aus Terminalsymbolen bestehen)
	\end{itemize}
\end{frame}


	%	\begin{frame}{Rückblick: Prädikatenlogik}
	%		\begin{itemize}[<+->]
	%			\item Deutlich komplizierterer Aufbau als Aussagenlogik
	%			\item Auswertung mit Interpretation und Variablenbelegung
	%			\item Quantoren erlauben allgemeine Aussagen
	%		\end{itemize}
	%	\end{frame}
	
%\begin{frame}[t]{Wahr oder Falsch?}
%	Sei $\RPL = \{\word R, \word S\}$ mit $\ar(\word R) = 2$ und $\ar(\word S) = 1$, \\
%	$\FPL = \{\word f, \word g\}$ mit $\ar(\word f) = 1$ und $\ar(\word g) = 2$ . \\
%	\TrueQuestionE{$\word{R(y,g(x,y))}$ ist präd.log. syntaktisch korrekt.}{}
%	\FalseQuestionE{$\word{f(S(x))}$ ist präd.log. syntaktisch korrekt.}{Eine Relation kann nicht innerhalb einer Funktion auftauchen: Das geht nur mit Termen, nicht mit atomaren Formeln.}
%	\FalseQuestionE{\enquote{Nicht alle Kinder spielen nicht.} $\equiv \plall \word{x\,(child(x)} \alimpl \word{play(x)}\plkz $}{Der Text sagt nur, dass es mindestens ein Kind gibt, das spielt.}
%	\FalseQuestionE{\enquote{Wenn Person $a$ Person $b$ liebt, liebt $b$ nicht unbedingt $a$.} \\ $\equiv \plall \word a \plall \word b \, \plka \word{liebt(a,b)} \alimpl \alnot \word{liebt(b,a)}\plkz $}{Das hieße, dass es nur unerwiderte Gefühle gäbe. \textit{*schnief*} \\ Richtig wäre $\alnot \plall \word a \plall \word b \,\plka \word{liebt(a,b)} \alimpl \word{liebt(b,a)}\plkz $. \\ Umgeformt: „Es gilt nicht unbedingt, dass wenn $a$ $b$ liebt, dann auch $b$ $a$ liebt.“}
%	%TODO: Eine W/F-Frage zu freien/gebundenen Variablen
%	%TODO: Eine W/F-Frage zu PL Substitution
%\end{frame}	

\input{../Bloecke/Relationen2.tex}

\input{../Bloecke/Praedikatenlogik.tex}

\input{../Bloecke/Praedikatenlogik2}

\thassedaniel{
%	\input{../Bloecke/DrMetaNordpol}
%	\input{../Bloecke/Hoare.tex}
}{
%	\input{../Bloecke/Hoare.tex}
%	\input{../Bloecke/DrMetaNordpol}
}

\begin{frame}	
	\begin{block}{Was ihr nun wissen solltet}
		\begin{itemize}
			\item Wie Prädikatenlogische Formeln aufgebaut sind
			\item Wie man damit präzise Aussagen trifft
			\item Wie man sie auswertet
			%\item Wie der Hoare-Kalkül funktioniert % TODO maybe
			%\item Wie man mit dem Hoare-Kalkül ein Programm beweist.
		\end{itemize}
	\end{block}
	
	\begin{block}{Was nächstes Mal kommt}
		\begin{itemize}
			\item Alles korrekt? –- Beweise mit dem Hoare-Kalkül
			\item Graphen -- Alles vernetzt
			\item Systematisches Suchen und Wandern -- Algorithmen auf Graphen
		\end{itemize}
	\end{block}
\end{frame}

%\xkcdframevert{835}{Frohe Weihnachten und bis nächstes Jahr! \smiley}{2.2}
%\lastframetitled{0.47}{0}{xkcd/christmastree.png}{https://www.xkcd.com/835}{\vspace{-1.66\baselineskip}\\Frohe Weihnachten und einen \\ guten Start ins neue Jahr! \smiley}
\slideThanks

\end{document}