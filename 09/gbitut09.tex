%beamer

% Comment/uncomment this line to toggle handout mode
\newcommand{\handout}{}

\input{../framework/PraeambelTut.tex}

\morescalingdelimiters

\begin{document}
\starttut{9}

\section{Rückblick}

\begin{frame}{Zu Übungsblatt \#7}
	Bisheriger Schnitt: \quad 12.5 / 22~P

	\begin{itemize}[<+->]
		\item Das Übungsblatt bekommt ihr voraussichtlich heute Abend noch zurück
		\item Häufigste Fehler: heute + nächste Woche
	\end{itemize}
\end{frame}

\begin{frame}{Zu Übungsblatt \#7}
	Ein häufiger Fehler:
	\begin{itemize}[<+->]
		\item Aufgabe 7.1: Es sei $A = \left\{\word a,\word b\right\}$. Zeigen Sie, dass die folgenden Gleichungen für $L\subseteq A^*$ jeweils mindestens eine Lösung besitzen:
		\item[a)] $L = \left\{\varepsilon\right\} \cup \left\{a\right\} \cdot L \cdot \left\{b\right\}$
		\item[] Intuitiv oder durch Ausprobieren: $L=\left\{\word a^n\word b^n \middle| n \in \mathbb N_ 0\right\}$ ist eine Lösung.
		\item[] Das zeigen wir, indem wir die obige Gleichung für dieses $L$ beweisen:
		\item[``$\subseteq$''] Sei $w \in L$. Dann ist $w=\word a^n\word b^n$ für ein festes $n\in\mathbb N_0$.
		\item[] \textbf{Falls $n=0$}, dann ist $w=\varepsilon$ und damit $w \in \left\{\varepsilon\right\} \cup \left\{a\right\} \cdot L \cdot \left\{b\right\}$
		\item[] \textbf{Falls $n\not=0$}, dann ist $n \geq 1$ und damit $w = \word a \cdot \word a^{n-1}\word b^{n-1} \cdot \word b$.
		\item[] \quad Wegen $\word a^{n-1}\word b^{n-1} \in L$ folgt $w \in \left\{\varepsilon\right\} \cup \left\{a\right\} \cdot L \cdot \left\{b\right\}$.
	\end{itemize}
\end{frame}

\begin{frame}{Zu Übungsblatt \#7}
	\begin{itemize}
		\item Aufgabe 7.1: Es sei $A = \left\{\word a,\word b\right\}$. Zeigen Sie, dass die folgenden Gleichungen für $L\subseteq A^*$ jeweils mindestens eine Lösung besitzen:
		\item[a)] $L = \left\{\varepsilon\right\} \cup \left\{a\right\} \cdot L \cdot \left\{b\right\}$
	\end{itemize}
	\begin{itemize}[<+->]
		\item[``$\supseteq$'] Sei $w \in \left\{\varepsilon\right\} \cup \left\{a\right\} \cdot L \cdot \left\{b\right\}$
		\item[] \textbf{Falls $w \in \left\{\varepsilon\right\}$}, dann ist $w=\varepsilon=\word a^0\word b^0$ und damit $w \in L$
		\item[] \textbf{Falls $w \in \left\{a\right\} \cdot L \cdot \left\{b\right\}$}, dann existiert ein $w' \in L$ sodass $w=\word a \cdot w' \cdot \word b$.
		\item[] \quad Nach Def. von $L$ existiert dann ein $n\in \mathbb N_0$ sodass $w'=\word a^n\word b^n$.
		\item[] \quad Dann ist $w = \word a \cdot w' \cdot \word b ) \word a \cdot \word a^n\word b^n \cdot b = \word a^{n+1} \word b^{n+1} \in L$.
		\item[] Damit ist die Gleichung gezeigt. \qed
	\end{itemize}
\end{frame}

\mycomment{
	\begin{frame}{Zu Blatt \#3}
		Durchschnitt: \quad etwa \thassedaniel{58}{53}~\% der Punkte \\
		\begin{itemize}
			\item \textbf{Induktionen}: Schreibt mir bitte die Aussage hin, über die ihr die Induktion macht. Wenn IV falsch, hab ich sonst keinen Plan, was ihr zeigen wollt.
			\item \textbf{A3.1}: Schaut euch Injektivität/Surjektivität nochmal an...
			\item \textbf{A3.2}: Huffman-Bäume: Es werden IMMER die zwei KLEINSTEN Knoten verbunden. Auch „über Kreuz“. 
			\item \textbf{A3.3}: Terme mit zu vielen Pünktchen („...“) sind keine Definition. \\ Alles, was nicht rekursiv ist, muss falsch sein wegen Klammerausdrücken.
			\item \textbf{A3.6}: Induktion über $n = \size{w_1} = \size{w_2}$, also beide Wortlängen gleichzeitig, geht NICHT! (Gibt nämlich auch Wörter, wo beide Längen nicht gleich sind... :P)
			
		\end{itemize}
	\end{frame}
}

\framePrevEpisode

\begin{frame}{Kahoot!}
	\begin{itemize}[<+->]
		\item Kahoot! ist ein anonymes Online-Quiz
		\item Ihr bekommt Punkte für schnelles und richtiges raten
		\item Ich schalte das Quiz frei und ihr könnt über \url{https://kahoot.it} beitreten
		\item Das Kahoot! könnt ihr euch später nochmal unter diesem Link angucken: \\
			\url{https://create.kahoot.it/share/gbi-woche-9-einstieg/6dd1cef1-785f-4e61-a11b-bf515867d02c}
	\end{itemize}
\end{frame}


	%	\begin{frame}{Rückblick: Prädikatenlogik}
	%		\begin{itemize}[<+->]
	%			\item Deutlich komplizierterer Aufbau als Aussagenlogik
	%			\item Auswertung mit Interpretation und Variablenbelegung
	%			\item Quantoren erlauben allgemeine Aussagen
	%		\end{itemize}
	%	\end{frame}
	
%\begin{frame}[t]{Wahr oder Falsch?}
%	Sei $\RPL = \{\word R, \word S\}$ mit $\ar(\word R) = 2$ und $\ar(\word S) = 1$, \\
%	$\FPL = \{\word f, \word g\}$ mit $\ar(\word f) = 1$ und $\ar(\word g) = 2$ . \\
%	\TrueQuestionE{$\word{R(y,g(x,y))}$ ist präd.log. syntaktisch korrekt.}{}
%	\FalseQuestionE{$\word{f(S(x))}$ ist präd.log. syntaktisch korrekt.}{Eine Relation kann nicht innerhalb einer Funktion auftauchen: Das geht nur mit Termen, nicht mit atomaren Formeln.}
%	\FalseQuestionE{\enquote{Nicht alle Kinder spielen nicht.} $\equiv \plall \word{x\,(child(x)} \alimpl \word{play(x)}\plkz $}{Der Text sagt nur, dass es mindestens ein Kind gibt, das spielt.}
%	\FalseQuestionE{\enquote{Wenn Person $a$ Person $b$ liebt, liebt $b$ nicht unbedingt $a$.} \\ $\equiv \plall \word a \plall \word b \, \plka \word{liebt(a,b)} \alimpl \alnot \word{liebt(b,a)}\plkz $}{Das hieße, dass es nur unerwiderte Gefühle gäbe. \textit{*schnief*} \\ Richtig wäre $\alnot \plall \word a \plall \word b \,\plka \word{liebt(a,b)} \alimpl \word{liebt(b,a)}\plkz $. \\ Umgeformt: „Es gilt nicht unbedingt, dass wenn $a$ $b$ liebt, dann auch $b$ $a$ liebt.“}
%	%TODO: Eine W/F-Frage zu freien/gebundenen Variablen
%	%TODO: Eine W/F-Frage zu PL Substitution
%\end{frame}	

\input{../Bloecke/Praedikatenlogik.tex}

\input{../Bloecke/Praedikatenlogik2}

\thassedaniel{
%	\input{../Bloecke/DrMetaNordpol}
%	\input{../Bloecke/Hoare.tex}
}{
%	\input{../Bloecke/Hoare.tex}
%	\input{../Bloecke/DrMetaNordpol}
}

\begin{frame}	
	\begin{block}{Was ihr nun wissen solltet}
		\begin{itemize}
			\item Wie Prädikatenlogische Formeln aufgebaut sind
			\item Wie man damit präzise Aussagen trifft
			\item Wie man sie auswertet
			%\item Wie der Hoare-Kalkül funktioniert % TODO maybe
			%\item Wie man mit dem Hoare-Kalkül ein Programm beweist.
		\end{itemize}
	\end{block}
	
	\begin{block}{Was nächstes Mal kommt}
		\begin{itemize}
			\item Alles korrekt? –- Beweise mit dem Hoare-Kalkül
			\item Graphen -- Alles vernetzt
			\item Systematisches Suchen und Wandern -- Algorithmen auf Graphen
		\end{itemize}
	\end{block}
\end{frame}

%\xkcdframevert{835}{Frohe Weihnachten und bis nächstes Jahr! \smiley}{2.2}
%\lastframetitled{0.47}{0}{xkcd/christmastree.png}{https://www.xkcd.com/835}{\vspace{-1.66\baselineskip}\\Frohe Weihnachten und einen \\ guten Start ins neue Jahr! \smiley}
\slideThanks

\end{document}