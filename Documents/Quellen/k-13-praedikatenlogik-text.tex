%=======================================================================
\Tut\chapter{Pr\"adikatenlogik}
\label{k:praedikatenlogik}

In Kapitel~\ref{k:aussagenlogik} haben wir Syntax und Semantik
aussagenlogischer Formeln kennengelernt.
%
Nun widmen wir uns der sogenannten \emph{Prädikatenlogik erster Stufe}.
%
Dabei wird auch Aussagenlogik wieder eine Rolle spielen. 

%-----------------------------------------------------------------------
\Tut\section{Syntax pr\"adikatenlogischer Formeln}
\label{sec:praedikatenlogik-syntax}

Prädikatenlogische Formeln sind komplizierter aufgebaut als aussagenlogische.
%
Man geht drei Schritten vor:
%
\begin{itemize}
\item Zunächst definiert man sogenannte \emph{Terme}, die aus
  Konstanten, Variablen und Funktionssymbolen zusammengesetzt werden.
\item Mit Hilfe von Relationssymbolen und Termen konstruiert man dann
  \emph{atomare Formeln}.
\item Aus ihnen werden mittels der schon bekannten aussagenlogischen
  Konnektive und zweier sogenannter Quantoren allgemeine
  \emph{prädikatenlogische Formeln} gebildet.
\end{itemize}
%
Für die Definition von Termen sind drei Alphabete gegeben:
%
\begin{itemize}
\item ein Alphabet $\CPL$ sogenannter
  \mdefine[Konstantensymbol]{Konstantensymbole}, notiert als $\plc_i$
  (für endliche viele $i\in\N_0$) oder kurz als $\plc$, $\pld$.
\item ein Alphabet $\VPL$ sogenannter
  \mdefine[Variablensymbol]{Variablensymbole}, notiert als $\plx_i$
  (für endliche viele $i\in\N_0$) oder kurz als $\plx$, $\ply$, $\plz$.
\item ein Alphabet $\FPL$ sogenannter
  \mdefine[Funktionsymbol]{Funktionsymbole}, notiert als $\plf_i$
  (für endliche viele $i\in\N_0$) oder kurz als $\plf$, $\plg$, $\plh$.
  %
  Für jedes $\plf_i\in\FPL$ bezeichne $\ar(\plf_i)\in\N_+$ die
  \mdefine{Stelligkeit} (oder \mdefine{Arität}) des Funktionssysmbols.
\end{itemize}
%
Das Alphabet $\ATer$ der Symbole, aus denen Terme zusammengesetzt
sind, umfasst neben den oben genannten außerdem Symbole "`Klammer
auf"', "`Komma"' und "`Klammer zu"':
\[
  \ATer = \{ \plka, \plcomma, \plkz \} \cup \CPL \cup \VPL \cup \FPL \;.
\]
%
Der syntaktische Aufbau von Termen wird mit Hilfe einer kontextfreien
Grammatik $(\NTer, \ATer, \#T, \PTer)$ definiert.
%
Dazu sei nun $m$ die größte Stelligkeit, die ein Funktionssymbol in
$\FPL$ oder ein Relationssymbol in $\RPL$ (siehe nächste Seite) hat.
%
Als $m+1$ Nichtterminalsymbole benutzen wir 
\[
  \NTer = \{ \,\#T \,\} \cup \{ \, \#L_i \mid i\in\N_+ \text{ und } i\leq m \,\} \;.
\]
%
Die folgenden Produktionen erzeugen dann zusammen die Terme:
\begin{alignat*}{2}
  \#L_{i+1} &\to \#L_i \plcomma \#T &\qquad& \text{für jedes } i\in\N_+\text{ mit } i<m   \\
  \#L_1  &\to \#T \\
  \#T &\to \plc_i && \text{für jedes } \plc_i\in\CPL\\
  \#T &\to \plx_i && \text{für jedes } \plx_i\in\VPL\\
  \#T &\to \plf_i\plka\#L_{\ar(\plf_i)} \plkz && \text{für jedes } \plf_i\in\FPL
\end{alignat*}
%
Aus \#T kann man alle Terme ableiten und aus jedem $\#L_i$ Listen mit
genau $i$ durch Kommata getrennten Termen. 
%
Wir bezeichnen die Menge aller Terme auch mit $\LTer$.

Ein Term heißt \mdefine{Grundterm}, wenn in ihm keine Variablensymbole
vorkommen, wenn also bei seiner Ableitung niemals eine der
Produktionen $\#T\to\plx_i$ benutzt wird.

Wenn $\plf$ ein zweistelliges und $\plg$ ein einstelliges
Funktionssymbol sind, dann sind zum Beispiel die folgenden Ausdrücke
Terme:

\setlength{\tabcolsep}{0pt}
\begin{tabularx}{\textwidth}{*{5}{>{\textbullet\ }X}}
$\plc$& $\ply$
& $\plg\plka\plx\plkz$
& $\plf\plka \plx\plcomma \plg\plka\plz\plkz\plkz$
& $\plf\plka \plc\plcomma \plg\plka \plg\plka\plz\plkz\plkz \plkz$ \\
\end{tabularx}
% 
Syntaktisch falsch sind dagegen

\begin{tabularx}{\textwidth}{*{5}{>{\textbullet\ }X}}
 $\plx\ply$ & $\plc\plka\plx\plkz$ & $\plf\plkz\plx\plcomma \ply\plka$
 &$\plg\plka \plc\plcomma\plc\plcomma\plc\plcomma\plx  \plkz$ & \multicolumn{1}{X}{}\\
\end{tabularx}
% 
und so weiter.
%
\begin{tutorium}
  \paragraph{Syntax prädikatenlogischer Terme}

  Machen Sie ein Beispiel, bei dem
  \begin{itemize}
  \item $\plf$ ein zweistelliges und $\plg$ ein einstelliges
    Funktionssymbol sind,
  \item $\plc$ das einzige Konstantensymbol und
  \item $\plx$ und $\ply$ Variablensymbole
  \end{itemize}
  Dann
  \begin{itemize}
  \item $\NTer=\{ \#T, \#L_1, \#L_2 \}$
  \item $\begin{aligned}[t]
      \PTer=\{  \#L_2 & \to \#L_1\plcomma\#T \\
                \#L_1 & \to \#T  \\
                \#T   & \to \plc \\
                \#T   & \to \plx \\
                \#T   & \to \ply \\
                \#T   & \to \plg\plka \#L_1 \plkz \\
                \#T   & \to \plf\plka \#L_2 \plkz \}
              \end{aligned}
              $
  \end{itemize}


  \begin{itemize}
  \item Konstruieren Sie \zB den Ableitungsbaum für
    $\plf\plka\plc\plcomma\plg\plka\plx\plkz\plkz$
  \item Machen Sie klar, was aus $\#T$ und was aus $\#L_1,\#L_2$
    ableitbar ist:
    \begin{itemize}
    \item aus $\#T$ sind \textbf{T}erme ableitbar
    \item aus $\#L_i$ sind \textbf{L}isten von $i$ Termen ableitbar
    \end{itemize}
  \item nicht übersehen: das Terminalsymbol "`Komma"': $\plcomma$
  \end{itemize}
\end{tutorium}

\mdefine[atomare Formel]{Atomare Formeln} werden aus
Relationssymbolen und Termen zusammengesetzt.
%
Dazu wird ein Alphabet $\RPL$ sogenannter
\mdefine[Relationsymbol]{Relationsymbole} festgelegt.
%
\begin{itemize}
\item Es enthält zum einen Symbole, die wir in der Form
  $\plR_i$ notieren (für endliche viele $i\in\N_0$) oder
  kurz als $\plR, \plS$.
  %
  Für jedes $\plR_i\in\RPL$ bezeichne $\ar(\plR_i)\in\N_+$ die
  \mdefine{Stelligkeit} (oder \mdefine{Arität}) des Relationssysmbols.
\item Außerdem enthalte $\RPL$ \emph{immer} ein Symbol, für das wir
  $\pleq$ schreiben, also ein "`Gleichheitszeichen mit einem Punkt
  darüber"'.
  %
  Wie man an der Grammatik weiter unten sehen wird, ist $\pleq$ ein
  zweistelliges Relationssymbol, das immer infix notiert wird.
  %
  (Diese Notation ist aus dem Skript zur Vorlesung "`Formale Systeme"'
  übernommen.)
\end{itemize}
%
Das Alphabet $\ARel$ der Symbole, aus denen atomare Formeln
zusammengesetzt sind, umfasst neben den Symbolen für Terme zusätzlich
nur die Relationssymbole:
\[
  \ARel = \ATer \cup \RPL \;.
\]
%
Für die kontextfreie Grammatik $(\NRel, \ARel, \#A, \PRel)$, die
atomare Formeln erzeugt, sei wieder $m$ die größte Stelligkeit,
die ein Funktions- oder ein Relationssymbol in $\FPL$ \bzw $\RPL$
haben.
%
Dann ist
\[
  \NRel = \{ \,\#A, \#T \,\} \cup \{ \, \#L_i \mid i\in\N_+ \text{ und } i\leq m \,\} = \{\,\#A\,\} \cup \NTer  \;.
\]
% 
Die Produktionenmenge für atomare Formeln erweitert die für Terme wie
folgt:
%
\begin{align*}
  \PRel = \PTer &\cup \{ \#A \to \plR_i \plka \#L_{\ar(\plR_i)} \plkz \mid \text{ für jedes } \plR_i \in\RPL \} \\
  &\cup \{ \, \#A \to \#T \pleq \#T \, \} \;.
\end{align*}
%
Die aus \#A ableitbaren Wörter sind die atomaren Formeln.
%
Die Menge aller atomaren Formeln bezeichnen wir mit $\LRel$.

Wenn $\plf$ ein zweistelliges und $\plg$ ein einstelliges
Funktionssymbol sind, $\plR$ ein dreistelliges und $\plS$ ein
einstelliges Relationssymbol, dann sind zum Beispiel die folgenden
Wörter atomare Formeln:

\begin{tabularx}{\textwidth}{*{3}{>{\textbullet\ }X}}
 $\plS\plka\plc\plkz$
& $\plR\plka\ply,\plc,\plg\plka\plx\plkz\plkz$
& $\plg\plka\plx\plkz \pleq \plf\plka \plx\plcomma \plg\plka\plz\plkz\plkz$ \\
\end{tabularx}
% 
Syntaktisch falsch sind dagegen

\begin{tabularx}{\textwidth}{*{3}{>{\textbullet\ }X}}
   $\plx\pleq\ply\pleq\plz$
 & $\plR\pleq\plR$
 & $\plS\plka\plx\plkz \pleq \plS\plka\plx\plkz$ \\
   $\plR\plka\plx\plcomma\ply\plkz$
 & $\plf\plka\plS\plka\plx\plkz\plkz$
 & $\plR\plka\plS\plka\plx\plkz\plcomma\plx\plcomma\plx\plkz$ \\
   $\plka\plS\plka\plS\plkz\plkz\plka\plx\plkz$
 & $\plR\plkz\plx\plcomma \ply\plka$
 & $\plg\plR\plf\plka \plkz$ \\
   $\plx \alimpl \plR\plka\plz\plkz$
 & $\plx \alor \ply$
\end{tabularx}
% 
und so weiter.

Das Alphabet $\AFor$ der Symbole, aus denen prädikatenlogische Formeln
zusammengesetzt sind, umfasst neben den Symbolen für atomare Formeln
zusätzlich die aussagenlogischen Konnektive und den sogenannten
\mdefine{Allquantor} $\plall$ und den sogenannten \mdefine{Existenzquantor}:
\[
  \AFor = \ARel \cup \{\, \alnot, \aland, \alor, \alimpl, \plall, \plexist  \,\} \;.
\]
%
Für die kontextfreie Grammatik $(\NFor, \AFor, \#F, \PFor)$ ist
\[
  \NFor = \{ \,\#F \,\} \cup \NRel
\]
% 
Die Produktionenmenge für prädikatenlogische Formeln erweitert die für
atomare Formeln wie folgt:
%
\begin{align*}
  \PFor = \PRel &\cup \{\, \#F \to \#A \,\} \\
  &\cup \{\, \#F \to \plka  \alnot \#F \plkz \,,\, \#F \to \plka \#F \aland \#F \plkz \,,\,
                  \#F \to \plka \#F \alor \#F \plkz \,,\, \#F \to \plka \#F \alimpl \#F \plkz\, \} \\
  &\cup \{ \, \#F \to \plka\plall \plx_i \#F \plkz \mid \plx_i\in\VPL \, \} %\\
                %&\cup \{ \, \#F \to \plka\plexist \plx_i \#F \plkz \mid \plx_i\in\VPL \, \}
\;.
\end{align*}
%
Die aus \#F ableitbaren Wörter sind die prädikatenlogischen Formeln.
%
Die Menge aller prädikatenlogischen Formeln bezeichnen wir mit $\LFor$.

Wie bei aussagenlogischen Formeln stehe für alle Formeln $G$ und $H$
das Wort $\alka G \aleqv H \alkz$ für
$\alka \alka G \alimpl H \alkz \aland \alka H \alimpl G \alkz \alkz$.
%
Außerdem schreiben wir als weitere Abkürzung
$\plka\plexist\plx_i F\plkz$ statt
$\plka\alnot\plka\plall\plx_i \plka\alnot F\plkz\plkz\plkz$.
%
Das Zeichen $\plexist$ heißt \mdefine{Existenzquantor}.

Und bei größeren Formeln verliert man wegen der vielen Klammern leicht
den Überblick.
%
Deswegen erlauben wir ganz analog zum Fall aussagenlogischer Formeln
folgende Klammereinsparungsregeln.
%
(Ihre "`offizielle"' Syntax bleibt die gleiche!)
%
\begin{itemize}
\item Die äußerten umschließenden Klammern darf man immer weglassen.
  %
  Zum Beispiel ist $\alv{P}\alimpl\alv{Q}$ die Kurzform von
  $\alka\alv{P}\alimpl\alv{Q}\alkz$.
\item Wenn ohne jede Klammern zwischen mehrere Aussagevariablen immer
  das gleiche Konnektiv steht, dann bedeute das "`implizite
  Linksklammerung"'.
  %
  Zum Beispiel ist $\alv{P}\aland\alv{Q}\aland\alv{R}$ die Kurzform von
  $\alka\alka\alv{P}\aland\alv{Q}\alkz\aland\alv{R}\alkz$.
\item Wenn ohne jede Klammern zwischen mehrere Aussagevariablen
  verschiedene Konnektive oder/und Quantoren stehen, dann ist von
  folgenden "`Bindungsstärken"' auszugehen:
  \begin{enumerate}[a)]
  \item $\plall$ und $\plexist$ binden am stärksten
  \item $\alnot$ bindet am zweitstärksten
  \item $\aland$ bindet am drittstärksten
  \item $\alor$ bindet am viertstärksten
  \item $\alimpl$ bindet am fünftstärksten
  \item $\aleqv$ bindet am schwächsten
  \end{enumerate}
  % 
  Zum Beispiel ist
  %
  $\plall\plx \plR\plka\plx\plcomma\ply\plkz \aland \plS\plka\plx\plkz$ 
  %
  die Kurzform von
  %
  $\plka\plall\plx \plR\plka\plx\plcomma\ply\plkz\plkz \aland \plS\plka\plx\plkz$.
\end{itemize}
%
\begin{tutorium}
  \paragraph{Klammersparregeln bei aussagenlogischen Formeln}

  Beispiele
  \begin{itemize}
  \item $\alv{P}\alor\alv{Q}\aland\alv{R}$ steht für %
    $\alka \alv{P}\alor \alka\alv{Q}\aland\alv{R}\alkz \alkz$
  \end{itemize}
\end{tutorium}


%-----------------------------------------------------------------------
\Tut\section{Semantik pr\"adikatenenlogischer Formeln}
\label{sec:praedikatenlogik-semantik}

% Unser Ziel ist es nun, jeder aussagenlogischen Formel im wesentlichen
% eine boolsche Funktion als Bedeutung zuzuordnen.

Es seien Alphabete $\CPL$, $\FPL$ und $\RPL$ gegeben.
%
Eine dazu passende \mdefine{Interpretation} $(D,I)$ ist durch folgende
Bestandteile festgelegt:
\begin{itemize}
\item eine nichtleere Menge $D$, das sogenannte \mdefine{Universum}
  der Interpretation,
\item für jedes $\plc_i\in\CPL$ ein Wert $I(\plc_i)\in D$,
\item für jedes $\plf_i\in\FPL$ eine Abbildung
  $I(\plf_i): D^{\ar(\plf_i)} \to D$ und
\item für jedes $\plR_i\in\RPL$ eine Relation
  $I(\plR_i)\subseteq  D^{\ar(\plR_i)}$.
\end{itemize}
%
Zu jedem Alphabet $\VPL$ und einer Interpretation $(D,I)$ ist eine
\mdefine{Variablenbelegung} eine Abbildung $\beta:\VPL\to D$.

Sind eine Interpretation $(D,I)$ und eine Variablenbelegung $\beta$
festgelegt, so kann man
\begin{itemize}
\item jedem Term einen Wert aus $D$ und
\item jeder Formel einen Wahrheitswert zuordnen.
\end{itemize}
%
Die entsprechende Abbildung $\valDIb:\LTer\cup\LFor \to D\cup \Bool$
definieren wir schrittweise induktiv zunächst für Terme und
anschließend für Formeln wie folgt.

Für jeden Term $t\in\LTer$ und alle Terme $t_1,\dots,t_k\in\LTer$ sei
\[
  \valDIb(t) = \begin{dcases*}
    \beta(\plx_i), & falls $t=\plx_i\in\VPL$\\
    I(\plc_i), & falls $t=\plc_i\in\CPL$\\
    I(\plf_i) (\valDIb(t_1), \dots, \valDIb(t_k)), & falls $t=\plf_i\plka t_1\plcomma\dots\plcomma t_k\plkz$
  \end{dcases*}
\]
%
Wie man sieht, ist für jeden Term $t$ der Funktionswert $\valDIb(t)$
definiert und ein Element von $D$.
%
Wir haben uns an dieser Stellen erlaubt, Pünktchen"=Notation zu
verwenden.
%
Das hätte man durch zusätlichen Aufwand vermeiden können, der das
Ganze aber schlechter lesbar gemacht hätte.
%
Deshalb machen wir diese Ausnahme hier \dots\ und gleich noch einmal.

Jede atomare Formel ist entweder von der Form
$\plR_i\plka t_1,\dots, t_k\plkz$ für Terme
$(t_1,\dots, t_k)\in \LTer^{\ar(\plR_i)}$ oder von der Form $t_1\pleq t_2$
für Terme $(t_1,t_2)\in \LTer^2$. 
%
Für sie wird festgelegt:
\begin{align*}
  \valDIb(\plR_i\plka t_1,\dots, t_k\plkz) &= \begin{dcases*}
    \W, & falls $(\valDIb(t_1), \dots, \valDIb(t_k))\in I(\plR_i)$ \\
    \F, & falls  $(\valDIb(t_1), \dots, \valDIb(t_k))\notin I(\plR_i)$ 
  \end{dcases*}\\
  \valDIb(t_1\pleq t_2) &=
  \begin{dcases*}
    \W, & falls $\valDIb(t_1) = \valDIb(t_2)$ \\
    \F, & falls $\valDIb(t_1) \not= \valDIb(t_2)$ 
  \end{dcases*}
\end{align*}

Damit muss nur noch für nicht-atomare Formeln $F$ definiert werden,
was $\valDIb(F)$ sein soll.
%
Falls $F$ von einer der Formen ist, die wir schon aus der
Aussagenlogik kennen, sei $\valDIb(F)$ dementsprechend definiert. Zum
Beispiel sei für alle prädikatenlogischen Formeln $G$ und $H$
festgelegt:
$\valDIb(H_1 \aland H_2) = b_{\aland}(\valDIb(H_1), \valDIb(H_2))$.

Damit bleibt nur noch, für jede Formel $F$ zu definieren, was
$\valDIb(\plall\plx_i F)$ sein soll.
%
Dafür führen wir noch folgende Hilfsdefinition ein.
%
Für jede Variablenbelegung $\beta:\VPL\to D$, jedes $\plx_i\in \VPL$
und jedes $d\in D$ sei
\[
  \beta_{\plx_i}^d : \VPL \to D : 
  \plx_j \mapsto
           \begin{dcases*}
             \beta(\plx_j) & falls $j\not=i$ \\
             d & falls $j=i$
           \end{dcases*}
\]
%
diejenige Variablenbelegung, die mit $\beta$ für alle Variablen
ungleich $\plx_i$ übereinstimmt, und für $\plx_i$ den Wert $d$
vorschreibt.

Damit ist dann
\[
  \valDIb(\plall\plx_i F) =
  \begin{dcases*}
    \W, & falls für jedes $d\in D$ und $\beta'=\beta_{\plx_i}^d$ gilt: $\val_{D,I,\beta'}(F)=\W$ \\
    \F, & sonst
  \end{dcases*}
\]
Man kann sich überlegen, dass dann auch gilt:
\[
  \valDIb(\plexist\plx_i F) =
  \begin{dcases*}
    \W, & falls für mind.~ein $d\in D$ und $\beta'=\beta_{\plx_i}^d$ gilt: $\val_{D,I,\beta'}(F)=\W$ \\
    \F, & sonst
  \end{dcases*}
\]

Eine prädikatenlogische Formel $F$ heißt
\mdefine[allgemeingültige\\Formel]{allgemeingültig}, wenn für jede
passende Interpretation $(D,I)$ und jede passende Variablenbelegung
$\beta$ gilt: $\valDIb(F)=\W$.

Eine einfache Methode, um zu allgemeingültigen Formeln zu gelangen,
besteht darin, eine aussagenlogische Tautologie $G$ zu nehmen und für
jede in ihr vorkommende Aussagevariable $\alP_i$ eine beliebige
prädikatenlogische Formel $G_i$ und dann in $G$ alle Vorkommen von
$\alP_i$ syntaktisch durch $G_i$ zu ersetzen.
%
Wir wollen so entstehende Formeln als prädikatenlogische Tautologien
bezeichnen.

Aber es gibt noch andere allgemeingültige prädikatenlogische Formeln.
%
Ein einfaches Beispiel ist die Formel
\[
  \plka t_1\pleq t_2\plkz \alimpl \plka t_2 \pleq t_1 \plkz \;,
\]
%
andere sind \zB von der Form
%
\[
  \plka\plall\plx_i\;\plka G\alimpl H\plkz\plkz \alimpl \plka
  \plka\plall\plx_i \; G\plkz \alimpl \plka\plall\plx_i \; H\plkz
  \plkz
\]
%
für beliebige prädikatenlogische Formeln $G$ und $H$.

Ist $(D,I)$ eine Interpretation für eine prädikatenlogische Formel
$G$, dann nennen wir $(D,I)$ ein \mdefine[Modell einer
Formel]{Modell}\index{Modell!für prädikatenlogische Formeln} von $G$,
wenn für jede Variablenbelegung $\beta$ gilt, dass $\valDIb(G)=\W$ ist.
%
Ist $(D,I)$ eine Interpretation für eine Menge $\Gamma$
prädikatenlogischer Formeln, dann nennen wir $(D,I)$ ein
\mdefine[Modell einer Formelmenge]{Modell} von $\Gamma$, wenn $(D,I)$
Modell jeder Formel $G\in \Gamma$ ist.

Ist $\Gamma$ eine Menge prädikatenlogischer Formeln und $G$ ebenfalls
eine, so schreibt man auch genau dann $\Gamma\models G$, wenn jedes
Modell von $\Gamma$ auch Modell von $G$ ist.
%
Enthält $\Gamma=\{H\}$
nur eine einzige Formel, schreibt man einfach $H\models G$.
%
Ist $\Gamma=\{\}$ die leere Menge, schreibt man einfach $\models G$.
%
Die Bedeutung soll in diesem Fall sein, dass $G$ für \emph{alle}
Interpretationen überhaupt wahr ist, \dh dass $G$ allgemeingültig ist.

Als Beispiel betrachten wir die Formel $G$
\[
   \plall \plx \; \plf\plka\plx \plcomma\plc\plkz \pleq \plx
\]
und mehrere Interpretationen:
\begin{itemize}
\item Es sei $D=\N_0$, $I(\plc)=0$ und $I(\plf)$ die Addition von
  Zahlen.
  % 
  Dann ist $(D,I)$ ein Modell der Formel, denn anschaulich bedeutet
  $G$ dann: Für jede nichtnegative ganze Zahl $x$ ist $x+0=x$.
  
  Genauer muss man sich fragen, ob für jedes $\beta$ gilt:
  $\valDIb(G)=\W$.
  %
  Dazu muss man für jedes $d\in\N_0$ und $\beta'=\beta_{\plx}^d$
  prüfen, ob
  $\val_{D,I,\beta'}(\plf\plka\plx \plcomma\plc\plkz \pleq \plx) =\W$
  ist.
  %
  Das ist genau dann der Fall, falls
  $\val_{D,I,\beta'}(\plf\plka\plx \plcomma\plc\plkz) =
  \val_{D,I,\beta'}(\plx)$ ist.

  Die linke Seite ist
  $I(\plf)(\beta'(\plx), I(\plc)) = \beta'(\plx)+0=\beta'(\plx)$, was
  gerade gleich der rechten Seite ist.
\item Ein anderes Modell von $G$ erhält man, wenn $D=\{\#a,\#b\}^*$,
  $I(\plc)=\eps$ und $I(\plf)$ die Konkatenation von Wörtern ist.
\item Ist dagegen $D=\N_0$, $I(\plc)=0$ und $I(\plf)$ die
  Multiplikation von Zahlen, dann liegt kein Modell für $G$ vor, denn
  über den nichtnegativen ganzen Zahlen ist \emph{nicht} stets
  $x\cdot 0=x$.
\end{itemize}
%
Betrachtet man andererseits die Formel
$\plall \plx\; \plall\ply\; \plf\plka\plx\plcomma\ply\plkz \pleq
\plf\plka\ply\plcomma\plx\plkz$, dann ist die erste der obigen
Interpretationen wieder ein Modell (weil die Addition von Zahlen
kommutativ ist), die zweite Interpretation ist aber kein Modell (weil
die Konkatenation von Wörtern nicht kommutativ ist).

% -----------------------------------------------------------------------
\Tut\section{Freie und gebundene Variablenvorkommen und Substitutionen}
\label{sec:praedikatenlogik-frei-geb-subst}

Dieser Abschnitt ist sehr technisch.
%
Wofür das alles nötig ist, und dass sich der Aufwand tatsächlich
lohnt, werden wir in späteren Abschnitten dieses Kapitels und auch in
weiteren Kapiteln der Vorlesung sehen.
%
Bis dahin müssen Sie sich beim Lesen etwas gedulden.

Wenn in einer prädikatenlogischen Formel $G$ in einem Term eine
Variable $\plx$ steht, dann spricht man auch von einem
\mdefine[Vorkommen einer Variablen]{Vorkommen}\index{Vorkommen einer Variablen}%
\index{Variable!Vorkommen in einer Formel} der Variablen $\plx$ in
$G$.
%
(Die Anwesenheit einer Variablen unmittelbar hinter einem Quantor
zählt \emph{nicht} als Vorkommen.)

Die gleiche Variable kann natürlich an mehreren Stellen in $G$
vorkommen.
%
Genauer unterscheidet man sogenannte \mdefine[freies Vorkommen]{freie}
und \mdefine[gebundenes Vorkommen]{gebundene} Vorkommen von Variablen
in $G$.
%
Außerdem definiert man die Menge $\fv(G)$ der frei in $G$ vorkommenden
Variablen und die Menge $\bv(G)$ der gebunden in $G$ vorkommenden
Variablen.
%
Diese Konzepte sind wie folgt definiert.

Für jede Formel $G$, die atomar ist, sind alle Vorkommen von
Variablen frei und es ist $\bv(G)=\{\}$ und $\fv(G)$ die Menge aller
in $G$ an mindestens einer Stelle vorkommenden Variablen.

Für jede Formel $G$ der Form $\alnot H$ ist $\bv(G)=\bv(H)$ und
$\fv(G)=\fv(H)$ und jedes freie \bzw gebundene Vorkommen einer
Variablen in $H$ ist auch ein freies \bzw gebundenes Vorkommen in $G$.

Für jede Formel $G$, die eine der Formen $H_1\aland H_2$,
$H_1\alor H_2$ oder $H_1\alimpl H_2$ hat, ist
$\bv(G)=\bv(H_1)\cup \bv(H_2)$ und  $\fv(G)=\fv(H_1)\cup \fv(H_2)$ und
jedes freie \bzw gebundene Vorkommen einer Variablen in $H_1$ oder
$H_2$ ist auch ein freies \bzw gebundenes Vorkommen in $G$.

Für jede Formel $G$ der Form $\plka \plall \plx_i \, H\plkz$ oder
$\plka \plexist \plx_i \, H\plkz$ ist
$\fv(G)=\fv(H)\smallsetminus\{\plx_i\}$ und
$\bv(G)=\bv(H)\cup\left( \{\plx_i\} \cap \fv(H)\right)$, \dh
expliziter ausgedrückt
\[
\bv(G)=\begin{dcases*}
  \bv(H)\cup \{\plx_i\} & falls $\plx_i\in\fv(H)$ \\
  \bv(H) & sonst \\
\end{dcases*}
\]
%
Außerdem sind in $G$ alle Vorkommen der Variablen $\plx_i$ gebunden
und man sagt auch, dass sich die in $H$ freien Vorkommen von $\plx_i$
im \mdefine[Wirkungsbereich\\eines Quantors]{Wirkungsbereich}%
\index{Wirkungsbereich eines Quantors}\index{Quantor!Wirkungsbereich}
des Quantors davor befinden und durch ihn gebunden werden.
%
Alle Vorkommen anderer Variablen in $G$ sind frei \bzw gebunden, je
nachdem ob es in $H$ freie oder gebundene Vorkommen  sind.

Eine Formel $G$ heiße \mdefine[geschlossene Formel]{geschlossen}, wenn
$\fv(G)=\{\}$ ist.

Als Beispiel betrachten wir die Formel $G$
\[
  \plall \plx \, \plka \plR\plka\plx\plcomma\ply\plkz \aland
  \plexist\ply\, \plR\plka\plx\plcomma\ply\plkz  \plkz
\]
\begin{itemize}
\item Das sechste Zeichen ist das erste Vorkommen von $\plx$; es ist
  ein gebundenes Vorkommen, denn dieses $\plx$ wird durch den
  Allquantor am Anfang der Formel gebunden.
\item Das fünfzehnte Zeichen ist das zweite Vorkommen von $\plx$; es
  ist ein gebundenes Vorkommen, denn dieses $\plx$ wird ebenfalls
  durch den Allquantor am Anfang der Formel gebunden.
\item Das achte Zeichen ist das erste Vorkommen von $\ply$; es ist
  ein freies Vorkommen.
\item Das siebzehnte Zeichen ist das zweite Vorkommen von $\ply$; es
  ist ein gebundenes Vorkommen, dieses $\ply$ wird durch den
  Existenzquantor nach dem $\aland$ gebunden.
\item Es ist also $\bv(G)=\{\plx,\ply\}$ und $\fv(G)=\{\ply\}$.
  % 
  Wie man sieht, müssen die Menge der frei und die Menge der gebunden
  vorkommenden Variablen nicht disjunkt sein.
\item Die Formel ist nicht geschlossen (denn $\ply$ kommt frei darin
  vor).
\end{itemize}

\noindent
Eine \mdefine{Substitution}\index{Substitution} ist eine Abbildung
$\sigma:\VPL\to \LTer$.
%
Sind die $k$ Variablen $\plx_{i_j}$ mit $1\leq j\leq k$ die einzigen
Variablen mit $\sigma(\plx_{i_j})\not=\plx_{i_j}$, dann ist $\sigma$
durch die Menge $S$ der Paare
$\{\plx_{i_j}/\sigma(\plx_{i_j}) \mid 1\leq j\leq k\}$ eindeutig
bestimmt.
%
(Es ist üblich, die Paare in der Form
$\plx_{i_j}/\sigma(\plx_{i_j})$ und nicht in der Form
$(\plx_{i_j},\sigma(\plx_{i_j}))$ zu notieren.)
%
Wir notieren $\sigma$ dann auch in der Form $\sigma_S$, wobei
$S$ stets rechtseindeutig ist, da $\sigma$ eine Abbildung ist.
%
Zum Beispiel bedeutet
$\sigma_{\{\plx/\plc, \ply/\plf\plka\plx\plkz \}}$ die Abbildung mit
\begin{align*}
  \sigma(\plx) &= \plc\\
  \sigma(\ply) &= \plf\plka\plx\plkz\\
  \sigma(z) &= z \text{ für jedes } z\notin\{\plx,\ply\}
\end{align*}
%
Man erweitert $\sigma_S$ zu einer Abbildung $\sigma_S':\LTer\to\LTer$,
indem man die in einem Term vorkommenden Variablen "`alle
gleichzeitig"' gemäß $S$ ersetzt:
\[
  \sigma_S'(t) = \begin{dcases*}
    \sigma_S(x), & falls $t=x$ mit $x\in\VPL$ \\
    c, & falls $t=c$ mit $c\in\CPL$ \\
    f\plka\sigma_S'(t_1),\dots,\sigma_S'(t_k)\plkz & falls $t=f\plka t_1,\dots,t_k\plkz$ mit $f\in\FPL$ und $t_1,\dots,t_k\in\LTer$
  \end{dcases*}
\]
%
Statt $\sigma'_S$ schreibt man üblicherweise wieder einfach $\sigma_S$.

Zum Beispiel ist 
\begin{align*}
  \sigma_{\{\plx/\plc, \ply/\plf\plka\plx\plkz \}}(\plx) 
  &= \plc \\
  \sigma_{\{\plx/\plc, \ply/\plf\plka\plx\plkz \}}(\ply) 
  &= \plf\plka\plx\plkz \\
  \sigma_{\{\plx/\plc, \ply/\plf\plka\plx\plkz \}}(\plg\plka\ply\plcomma\plx\plkz) 
  &= \plg\plka\plf\plka\plx\plkz\plcomma\plc\plkz \\
  \sigma_{\{\plx/\plc, \ply/\plf\plka\plx\plkz \}}(\plf\plka\plz\plcomma\plz\plkz) 
  &= \plf\plka\plz\plcomma\plz\plkz
\end{align*}

\noindent
Eine Substitution $\sigma_S:\LTer\to\LTer$ erweitert man schließlich in
einem zweiten Schritt zu einer Abbildung $\sigma_S'':\LFor\to\LFor$ (für
die man hinterher ebenfalls einfach wieder $\sigma_S$ schreibt).
%
Das macht man induktiv und bei Formeln ohne Quantoren auf naheliegende
Art und Weise:
\[
  \sigma_S''(G) = \begin{dcases*}
    R\plka\sigma_S'(t_1),\dots,\sigma_S'(t_k)\plkz, & falls $G=R\plka t_1,\dots,t_k\plkz$ mit $R\in\RPL$ und $t_1,\dots,t_k\in\LTer$ \\
    \sigma_S'(t_1)\pleq \sigma_S'(t_2), & falls $G= t_1\pleq t_2$ mit $t_1,t_2\in\LTer$\\
    \alnot \sigma_S''(H), & falls $G=\alnot H$  \\
    \sigma_S''(H_1) \aland \sigma_S''(H_2), & falls $G= H_1\aland H_2$  \\
    \sigma_S''(H_1) \alor \sigma_S''(H_2), & falls $G= H_1\alor H_2$  \\
    \sigma_S''(H_1) \alimpl \sigma_S''(H_2), & falls $G= H_1\alimpl H_2$  \\
  \end{dcases*}
\]
%
Für den verbleibenden Fall einer quantifizierten Formel definieren wir
vorbereitend zu jeder Substitution $\sigma_S$ und jedem $x\in\VPL$ eine
Substitution $\sigma_{S-x}$ vermöge der Festlegung:
\begin{align*}
  \sigma_{S-x}(x) &= x \\
  \sigma_{S-x}(y) &= \sigma_S(y) \text{ für jedes } y\in\VPL \text{ mit }y\not=x
\end{align*}
%
Damit können wir nun noch definieren:
\begin{align*}
  \sigma_S''( \plall x \; H ) &= \plall x \; \sigma''_{S-x}(H) \\
  \sigma_S''( \plexist x \; H ) &= \plexist x \; \sigma''_{S-x}(H)
\end{align*}
%
Das bedeutet nichts anderes als dass bei einer Substitution gebundene
Vorkommen von Variablen nicht  ersetzt werden.
%
Wir betrachten dazu als Beispiel die Formel
$G = \plS\plka\plx\plkz \aland \plall\plx \;
\plR\plka\plx\plcomma\ply\plkz$ und eine beliebige Substitution
$\sigma_S$.
%
Dann erhält man Schritt für Schritt zunächst einmal
\begin{align*}
  \sigma_S(G) 
  &= \sigma_S(\plS\plka\plx\plkz \aland \plall\plx \; \plR\plka\plx\plcomma\ply\plkz) \\
  &= \sigma_S(\plS\plka\plx\plkz) \aland \sigma_S(\plall\plx \; \plR\plka\plx\plcomma\ply\plkz) \\
  &= \plS\plka\sigma_S(\plx)\plkz \aland \plall\plx \; \sigma_{S-\plx}(\plR\plka\plx\plcomma\ply\plkz) \\
  &= \plS\plka\sigma_S(\plx)\plkz \aland \plall\plx \; \plR\plka\sigma_{S-\plx}(\plx)\plcomma\sigma_{S-\plx}(\ply)\plkz
\end{align*}
%
Sei nun konkret die Substitution
$\sigma_S=\sigma_{\{\plx/\plc, \ply/\plf\plka\plx\plkz\}}$.
%
Dann ist $\sigma_{S-\plx}$ nichts anderes als
$\sigma_{\{\ply/\plf\plka\plx\plkz\}}$.
%
Folglich ergibt sich in obigem Beispiel weiter:
%
\begin{align*}
  \sigma_S(G) 
  &= \plS\plka\sigma_S(\plx)\plkz \aland \plall\plx \; \plR\plka\sigma_{S-\plx}(\plx)\plcomma\sigma_{S-\plx}(\ply)\plkz \\
  &= \plS\plka\sigma_{\{\plx/\plc, \ply/\plf\plka\plx\plkz\}}(\plx)\plkz \aland \plall\plx \; \plR\plka\sigma_{\{\ply/\plf\plka\plx\plkz\}}(\plx)\plcomma\sigma_{\{\ply/\plf\plka\plx\plkz\}}(\ply)\plkz \\
  &= \plS\plka\plc\plkz \aland \plall\plx \; \plR\plka\plx\plcomma\plf\plka\plx\plkz\plkz
\end{align*}
%
In diesem Beispiel ist etwas passiert, was man bei Substitutionen
häufig vermeiden will.
%
Nach der Substitution war das zweite Argument von $\plR$ durch den
Allquantor gebunden, vor der Substitution aber noch nicht.
%
Häufig möchte man nur über Substitutionen reden, bei denen das nicht
passiert.
%
Deshalb erhalten sie einen besonderen Namen.

Eine Substitution $\sigma$ heiße \mdefine[kollisionsfreie\\
Substitution]{kollisionsfrei}\index{Substitution!kollisiosnfrei}%
\index{kollisionsfreie Substitution} für eine Formel $G$, wenn für
jede Variable $\plx_i$, die durch $\sigma$ verändert wird (also
$\sigma(\plx_i)\not=\plx_i$) und  jede Stelle eines freien Vorkommens von $\plx_i$ in
$G$ gilt: Diese Stelle liegt nicht im Wirkungsbereich eines Quantors
$\plall \plx_j$ oder $\plexist \plx_j$, wenn $\plx_j$ eine Variable ist, die in
$\sigma(\plx_i)$ vorkommt.


Zum Abschluss dieses Abschnitts geht es erst einmal darum, eine ganze
Reihe allgemeingültiger Formeln kennenzulernen.
%
Dazu verallgemeinern wir noch einen Begriff, der schon bei
aussagenlogischen Formeln eine Rolle spielte.
%
Zwei prädikatenlogische Formeln $G$ und $H$ heißen \mdefine[logisch\\
äquivalente\\Formeln]{logisch äquivalent}, wenn für jede passende
Interpretation $(D,I)$ und jede passende Variablenbelegung $\beta$ gilt:
%
$\valDIb(G) = \valDIb(H)$.

Man kann sich überlegen, dass zwei Formeln $G$ und $H$ genau dann
logisch äquivalent sind, wenn $\models G\aleqv H$ gilt, wenn also
$G\aleqv H$ allgemeingültig ist.

Nachfolgend führen wir eine Reihe logisch äquivalenter Formelpaare
auf, ohne auf Beweise für die logische Äquivalenz einzugehen.
%
(Interessierte seien auf das Skript der Vorlesung "`Formale Systeme"'
verwiesen.)
%
Es seien jeweils $G$ und $H$ beliebige prädikatenlogische Formeln.

\begin{enumerate}
\item $\alnot\plall\plx_i \,G$ und $\plexist\plx_i \,\alnot G$
\item $\alnot\plexist\plx_i \,G$ und $\plall\plx_i \,\alnot G$
\item $\plall\plx_i\,\plall\plx_j \,G$ und  $\plall\plx_j\,\plall\plx_i \,G$ 
\item $\plexist\plx_i\,\plexist\plx_j \,G$ und $\plexist\plx_j\,\plexist\plx_i \,G$
\item $\plall\plx_i\,\plka G \aland H\plkz$ und  $\plall\plx_i\, G \aland \plall\plx_i\,H$
\item $\plexist\plx_i\,\plka G \alor H\plkz$ und  $\plexist\plx_i\, G \alor \plexist\plx_i\,H$
\item wenn $\plx_j\notin\fv(G)$ und $\sigma_{\{\plx_i/\plx_j\}}$
  kollisionsfrei für $G$, dann sind äquivalent
  \begin{itemize}
  \item $\plall\plx_i \,G$ und $\plall x_j \, \sigma_{\{\plx_i/\plx_j\}}(G)$
  \item $\plexist\plx_i \,G$ und $\plexist x_j \, \sigma_{\{\plx_i/\plx_j\}}(G)$
  \end{itemize}
  Man spricht in diesem Fall von \mdefine{gebundener Umbenennung} der Variablen.
\item wenn $\plx_i\notin\fv(G)$, dann sind äquivalent
  \begin{itemize}
  \item $G\aland \plall \plx_i \, H$ und $\plall \plx_i \, \plka G\aland H\plkz$ 
  \item $G\aland \plexist \plx_i \, H$ und $\plexist \plx_i \, \plka G\aland H\plkz$ 
  \item $G\alor \plall \plx_i \, H$ und $\plall \plx_i \, \plka G\alor H\plkz$ 
  \item $G\alor \plexist \plx_i \, H$ und $\plexist \plx_i \, \plka G\alor H\plkz$ 
  \item $G\alimpl \plall \plx_i \, H$ und $\plall \plx_i \, \plka G\alimpl H\plkz$ 
  \item $G\alimpl \plexist \plx_i \, H$ und $\plexist \plx_i \, \plka G\alimpl H\plkz$ 
  \end{itemize}
\item wenn $\plx_i\notin\fv(H)$, dann sind äquivalent
  \begin{itemize}
  \item $\plall\plx_i \,G \alimpl H$ und $\plexist\plx_i\, \plka G\alimpl H\plkz$
  \item $\plexist\plx_i \,G \alimpl H$ und $\plall\plx_i\, \plka G\alimpl H\plkz$
  \end{itemize}
\end{enumerate}
%
Weitere allgemeingültige Formeln, die aber nicht von der Form
$G\aleqv H$ sind, werden auch im nächsten Abschnitt benötigt.
%
Es seien $G\in\LFor$ und $\plx_i\in\VPL$.
\begin{enumerate}
\item Wenn die Substitution $\sigma_{\{\plx_i/t\}}$ kollisionsfrei für
  $G$ ist, dann ist
  \begin{itemize}
  \item $\plka \plall\plx_i\, G\plkz \alimpl \sigma_{\{\plx_i/t\}}(G)$
  \end{itemize}
  allgemeingültig.
\item Wenn $\plx_i\notin\fv(G)$ ist, dann ist
  \begin{itemize}
  \item $\plall\plx_i \,\plka G\alimpl H\plkz \alimpl\plka G\alimpl\plall\plx_i\, H\plkz$
  \end{itemize}
  allgemeingültig.
\item Für Variablen $\plx_i, \plx_j, \plx_k$ sind die Formeln
  \begin{itemize}
  \item $\plx_i\pleq\plx_i$
  \item $\plx_i\pleq\plx_j \alimpl \plx_j\pleq\plx_i$
  \item $\plx_i\pleq\plx_j \alimpl \plka  \plx_j\pleq\plx_k \alimpl  \plx_i\pleq\plx_k \plkz$
  \end{itemize}
  allgemeingültig.
\item Wenn $x_1,\dots, x_k\in\VPL$ und $y_1,\dots,y_k\in\VPL$ ,
  $\plf_i$ ein $k$-stelliges Funktionssymbol und $\plR_i$ ein
  $k$-stelliges Relationssymbol, dann sind die Formeln
  \begin{itemize}
  \item $x_1\pleq y_1 \alimpl\alka x_2\pleq y_2 \alimpl\alka\cdots \plka x_k\pleq y_k \alimpl
    \plf_i\plka x_1\plcomma\dots\plcomma x_k\plkz \pleq\plf_i\plka y_1\plcomma\dots\plcomma y_k\plkz\plkz \cdots\alkz\alkz$
  \item $x_1\pleq y_1 \alimpl\alka x_2\pleq y_2 \alimpl\alka\cdots \plka x_k\pleq y_k \alimpl
    \plka\plR_i\plka x_1\plcomma\dots\plcomma x_k\plkz \alimpl \plR_i\plka y_1\plcomma\dots\plcomma y_k\plkz\plkz\plkz \cdots\alkz\alkz$
  \end{itemize}
  allgemeingültig.
\end{enumerate}

%-----------------------------------------------------------------------
\section{Beweisbarkeit}
\label{sec:praedikatenlogik-beweisbarkeit}

Der grundlegende Begriff im Zusammenhang mit Beweisbarkeit sowohl in
der Aussagenlogik als auch in der Prädikatenlogik ist der des Kalküls.
%
In Kapitel~\ref{k:aussagenlogik} hatten wir bereits die prinzipielle
Struktur eines Kalküls anhand der Aussagenlogik kennengelernt:
\begin{itemize}
\item In der Menge aller syntaktisch korrekten Formeln wurde
\item eine Menge von \emph{Axiomen} ausgezeichnet, aus denen mithilfe
\item von \emph{Ableitungsregeln} in endliche vielen Schritten
\item die \emph{Theoreme} des Kalküls ableitbar sind.
\end{itemize}
%
Der wesentliche Punkt war dabei, dass man den Kalkül so konstuieren
kann --- und dann eben auch so konstruiert --- dass die Menge der
Theoreme mit der Menge der allgemeingültigen Formeln übereinstimmt.
%
Man kann also im Kalkül genau die Formeln beweisen, die in jeder
Interpretation wahr sind.

Dieses Programm werden wir nun für die Prädikatenlogik erster Stufe
analog vorstellen.
%
Dabei werden wir (wie in der Aussagenlogik so auch hier erst recht)
darauf verzichten, den zentralen Satz zu beweisen.
%
Das würde über den Rahmen dieser Grundlagenvorlesung hinausgehen.

Für die Prädikatenlogik gibt es, wie für die Aussagenlogik, ganz
unterschiedliche Kalküle, die das gleiche leisten.
%
Wir beschreiben nachfolgend eine Variante, die auf David Hilbert
zurückgeht.
%
Dazu sei
\begin{itemize}
\item $\AFor$ ein Zeichenvorrat (mit Variablen-, Konstanten-,
  Funktions- und Relationssymbolen) für prädikatenlogische Formeln
\item $\LFor$ die Menge der syntaktisch korrekten prädikatenlogischen
  Formeln über dem Alphabet $\AFor$,
\item $\AxPL\subseteq\LFor$ eine nachfolgend definierte Menge von
  Axiomen,
\item und zwei Schlussregeln, nämlich neben dem schon bekannten Modus
  ponens noch "`Generalisierung"'.
\end{itemize}
%
Als Menge $\AxPL$ der Axiome für den Hilbert"=Kalkül wählen wir die
Vereinigung der folgenden Mengen von Formeln.
%
Wir haben im vorangegangenen Abschnitt für die Formeln jeder Menge
angemerkt, dass es sich jeweils um allgemeingültige prädikatenlogische
Formeln handelt.
%
\begin{align*}
  {\AxAL}_1 &= \bigl\{\alka G\alimpl \alka H\alimpl  G\alkz\alkz
            \bigm| G,H\in\LFor \bigr\} \\
  {\AxAL}_2 &= \bigl\{\alka G\alimpl \alka H\alimpl  K\alkz\alkz
            \alimpl \alka\alka G\alimpl H\alkz\alimpl \alka G\alimpl  K\alkz\alkz \bigm| G,H,K\in\LFor \bigr\}\\
  {\AxAL}_3 &=  \bigl\{
            \alka\alnot H\alimpl \alnot G\alkz\alimpl \alka\alka\alnot H\alimpl G\alkz\alimpl  H\alkz
            \bigm| G,H \in\LFor 
            \bigr\} \\
  {\AxPL}_1 &=  \bigl\{
            \plka\plall \plx_i \, G \plkz \alimpl \sigma_{\{\plx_i/t\}}(G)
            \bigm| G \in\LFor, \plx_i\in\VPL, t\in\LTer \text{ und $\sigma_{\{\plx_i/t\}}$ kollisionsfrei für $G$}
            \bigr\} \\
  {\AxPL}_2 &=  \bigl\{
            \plka\plall \plx_i \, \plka G \alimpl H \plkz\plkz \alimpl \plka G \alimpl \plall \plx_i \, H \plkz
            \bigm| G,H \in\LFor, \plx_i\in\VPL, \text{ und } \plx_i\notin\fv(G) 
            \bigr\} \\
 {\AxEq}_1 &=  \bigl\{ \plx_i\pleq \plx_i \bigm| \plx_i\in\VPL \bigr\} \\
 {\AxEq}_2 &=  \bigl\{ \plx_i\pleq \plx_j  \alimpl \plx_j\pleq \plx_i
          \bigm| \plx_i, \plx_j\in\VPL 
          \bigr\} \\
 {\AxEq}_3 &=  \bigl\{ \plx_i\pleq \plx_j  \alimpl \plka \plx_j\pleq \plx_k \alimpl \plx_i\pleq \plx_k\plkz
          \bigm| \plx_i, \plx_j, \plx_k\in\VPL 
          \bigr\} \\
 {\AxEq}_4 &=  \bigl\{ \plx_{i_1}\pleq \plx_{j_1}  \alimpl \plka \plx_{i_2}\pleq \plx_{j_2}\alimpl \plka \cdots\plka \plx_{i_n}\pleq \plx_{j_n} \alimpl
           \plf_i\plka\plx_{i_1}\plcomma\ldots\plcomma\plx_{i_n}\plkz\pleq \plf_i\plka\plx_{j_1}\plcomma\ldots\plcomma\plx_{j_n}\plkz
           \plkz\cdots\plkz\plkz\\
         &\mathrel{\hphantom{=}}\;\; \bigm| \plx_{i_1},\cdots,\plx_{i_n}, \plx_{j_1},\cdots,\plx_{j_n}\in\VPL, \plf_i\in\FPL
          \bigr\} \\
 {\AxEq}_5 &=  \bigl\{ \plx_{i_1}\pleq \plx_{j_1}  \alimpl \plka \plx_{i_2}\pleq \plx_{j_2} \alimpl\plka \cdots\plka \plx_{i_n}\pleq \plx_{j_n} \alimpl
           \plka\plR_i\plka\plx_{i_1}\plcomma\ldots\plcomma\plx_{i_n}\plkz\alimpl \plR_i\plka\plx_{j_1}\plcomma\ldots\plcomma\plx_{j_n}\plkz\plkz
           \plkz\cdots\plkz\plkz\\
         &\mathrel{\hphantom{=}} \;\;\bigm| \plx_{i_1},\cdots,\plx_{i_n}, \plx_{j_1},\cdots,\plx_{j_n}\in\VPL, \plR_i\in\RPL
          \bigr\}
\end{align*}
%
Im Fall der Prädikatenlogik gibt es zwei Schlussregeln.
%
Die eine ist wieder Modus Ponens\index{Modus ponens}%
\index{Schlussregel!Modus ponens}, aber dieses Mal natürlich für
prädikatenlogische Formeln.
% 
Als Relation geschrieben ist 
$\MP\subseteq\LFor^3$ mit
\[
  \MP = \{ (G\alimpl H, G, H) \mid  G, H \in\LFor \} \text{\qquad\bzw\qquad}
  \text{\begin{tabular}{c}
          $G \alimpl H$ \qquad $G$ \\
          \midrule
          $H$
        \end{tabular}} 
\]
%
Die zweite Ableitungsregel heißt
\mdefine{Generalisierung}\index{Generalisierung}%
\index{Schlussregel!Generalisierung}. 
%
$\GEN\subseteq\LFor^2$ ist so definiert:
%
\[
  \GEN = \{ (G, \plka\plall\plx_i \, G\plkz) \mid  G \in\LFor \text{ und }\plx_i\in\VPL\} \text{\qquad\bzw\qquad}
  \text{\begin{tabular}{c}
          $G$ \\
          \midrule
          $\;\plka\plall\plx_i \, G\plkz\;$
        \end{tabular}} 
\]
%
Man mache sich bitte klar, dass die Anwendung von sowohl Modus ponens
als auch Generalisierung auf allgemeingültige Formeln wieder
allgemeingültige Formeln liefert.

Die Formalisierung des Begriffs \mdefine{Ableitung}\index{Ableitung!im
  Hilbert-Kalkül}\index{Kalkül!Ableitung}\index{Hilbert-Kalkül!Ableitung} oder auch
\mdefine{Beweis}\index{Beweis}\index{Kalkül!Beweis} erweitert die
Vorgehensweise aus der Aussagenlogik um die Möglichkeit, in einem
Schritt Generalisierung anzuwenden.

Dazu sei $\Gamma$ eine Formelmenge sogenannter
\mdefine[Hypothese]{Hypothesen} oder \mdefine[Prämisse]{Prämissen} und
$G$ eine Formel.
%
Eine \mdefine{Ableitung} von $G$
aus $\Gamma$
ist eine endliche Folge $(G_1,\dots,G_n)$
von $n$
Formeln mit der Eigenschaft, dass erstens $G_n=G$
ist und auf jede Formel $G_i$ einer der folgenden Fälle zutrifft:

\begin{itemize}
\item Sie ist ein Axiom: $G_i\in\AxPL$.
\item Oder sie ist eine Prämisse: $G_i\in\Gamma$.
\item Oder es gibt Indizes $i_1$ und $i_2$ echt kleiner $i$, für die gilt:
  $(G_{i_1},G_{i_2},G_i)\in\MP$.
\item Oder es gibt einen Indix $i_1$ echt kleiner $i$, für den gilt:
  $(G_{i_1},G_i)\in\GEN$.
\end{itemize}
%
Wir schreiben dann $\Gamma\vdash G$.
%
Ist $\Gamma=\{\}$, so heißt eine entsprechende Ableitung auch ein
\mdefine{Beweis}\index{Beweis} von $G$ und $G$ ein \mdefine{Theorem}%
\index{Theorem}\index{Kalkül!Theorem}\index{Hilbert-Kalkül!Theorem}
des Kalküls, in Zeichen: $\vdash G$.

Es ist nicht Aufgabe dieser Vorlesung in Hilberts Kalkül besonders
viele oder besonders komplizierte Theoreme zu beweisen.
%
Ein etwas ausführlicheres Beispiel soll aber klar machen, dass es der
Kalkül erlaubt, Beweise im obigen Sinn zu führen, die "`informell"'
geführten Beweisen entsprechen.
%
Das soll auch eine Andeutung sein der Tatsache, dass alle unsere
Beweise, die wir ja immer nicht präzise in einem Kalkül führen, eine
solide Grundlagen haben.

Zuvor seien noch zwei wichtige Theoreme aufgeführt, deren Beweise Sie
an anderen Stellen in der Literatur oder \zB in der Vorlesung
"`Formale Systeme"' finden können:

\begin{theorem}
  Für jede Formelmenge $\Gamma\subseteq\LFor$ und jede Formel
  $G\in\LFor$ gilt:
  \[
    \text{ wenn } \Gamma \vdash G \text{, dann auch } \Gamma
    \models G \;.
  \]
\end{theorem}
%
Man sagt auch, dass der Hilbert"=Kalkül \mdefine[Korrektheit des
Hilbert-Kalküls]{korrekt} sei.

\begin{theorem}
  Für jede Formelmenge $\Gamma\subseteq\LFor$ und jede Formel
  $G\in\LFor$ gilt:
  \[
    \text{ wenn } \Gamma \models G \text{, dann auch } \Gamma
    \vdash G \;.
  \]
\end{theorem}
%
Man sagt auch, dass der Hilbert"=Kalkül \mdefine[Vollständigkeit des
Hilbert-Kalküls]{vollständig} sei.

Außerdem sei noch eine Warnung ausgesprochen.
%
Im Fall der Aussagenlogik hatten wir das Deduktionstheorem
kennengelernt.
%
Ohne an dieser Stelle auf Details einzugehen, sei erwähnt, dass man es
\emph{nicht} in voller Allgemeinheit auf die Prädikatenlogik
übertragen kann.
%
Wir erwähnen nur das folgende Analogon, das eine relativ starke
Voraussetzung macht (es gibt schwächere hinreichende Voraussetzungen).
\begin{theorem}
  Für jede \emph{geschlossene} Formel $G\in\LFor$ und jedes
  $H\in\LFor$ gilt $G\vdash H$ genau dann, wenn
  $\vdash \alka G\alimpl H\alkz$ gilt.
\end{theorem}
%
Zum Abschluss dieses Abschnitts betrachten wir nun beispielhaft einen
Fall, in dem $\FPL=\{\plf\}$ und $\CPL=\{\plc\}$. Außerdem sei
$\Gamma=\{\;\plall\plx\, \plka\plf\plka\plc\plcomma\plx\plkz \pleq
\plx \aland \plf\plka\plx\plcomma\plc\plkz \pleq\plx\plkz\;\}$.
%
Jedes Modell $(D,I)$ von $\Gamma$ besitzt also eine auf $D$ definierte
zweistellige Operation $I(\plf)$ und eine Konstante $I(\plc)$, die
neutrales Element bezüglich der Operation ist.

Ein Modell erhält man \zB durch die Festlegungen
\begin{itemize}
\item $D=\N_0$, $I(\plf):\N_0\x\N_0\to\N_0:(x,y)\mapsto x+y$ und $I(\plc)=0$,
\end{itemize}
ein anderes durch
\begin{itemize}
\item $D=\{\#a,\#b\}^*$,
  $I(\plf):\{\#a,\#b\}^*\x
  \{\#a,\#b\}^*\to\{\#a,\#b\}^*:(w_1,w_2)\mapsto w_1\cdot w_2$ und
  $I(\plc)=\eps$.
\end{itemize}
%
Wir wollen nun zeigen, dass in jedem solchen Modell das neutrale
Element immer eindeutig ist, \dh dass für jedes Modell von $\Gamma$
die folgende Aussage $G$ wahr ist:
\[
  \plall\ply \, \plka \; \plall\plx\,
  \plka\plf\plka\ply\plcomma\plx\plkz \pleq \plx \aland
  \plf\plka\plx\plcomma\ply\plkz \pleq\plx\plkz \alimpl \ply\pleq\plc
  \plkz
\]
%
Den Nachweis wollen wir unter Ausnutzung der Korrektheit des
Hilbert"=Kalküls zeigen.
%
Wir müssen also eine Ableitung von $G$ aus den Axiomen und den
Hypothesen in $\Gamma$ finden.

Wir werden nicht einen genauen Beweis im Hilbert"=Kalkül aufschreiben;
das ist zu aufwändig.
%
Aber wir werden Formulierungen, wie wir sie "`normalerweise"' nutzen,
so erweitern, dass interessierte Leser den Rest hoffentlich selbst
erledigen können.
%
Salopp gesprochen beruht die Aussage auf der Beobachtung, dass
\[
  y = f(c,y) = c \;,
\]
sobald $c$ linksneutrales Element ist und $y$ rechtsneutrales Element.
%
Genauer kann man in folgenden Schritten vorgehen:
%
\begin{description}%[Schr{i}tt 1:]
\item[Schritt 0:] Man beginnt mit einem "`beliebigen"' $\ply$ und muss zeigen: \\
  $\plall\plx\, \plka\plf\plka\ply\plcomma\plx\plkz \pleq \plx \aland
  \plf\plka\plx\plcomma\ply\plkz \pleq\plx\plkz \alimpl \ply\pleq\plc$
\item[Schritt 1:] zeige:  $\ply\pleq \plf\plka\plc\plcomma\ply\plkz  $
  \begin{description}
  \item[Schritt 1.1:] wegen
    $\plall\plx\, \plka\plf\plka\plc\plcomma\plx\plkz \pleq \plx \aland
    \plf\plka\plx\plcomma\plc\plkz \pleq\plx\plkz$ gilt insbesondere \\
    $\plf\plka\plc\plcomma\ply\plkz \pleq \ply \aland
    \plf\plka\ply\plcomma\plc\plkz \pleq\ply$
  \item[Schritt 1.2:] Also gilt $\plf\plka\plc\plcomma\ply\plkz \pleq \ply $.
  \item[Schritt 1.3:] Also gilt $\ply\pleq \plf\plka\plc\plcomma\ply\plkz  $.
  \end{description}
\item[Schritt 2:] zeige:  $\plall\plx\, \plka\plf\plka\ply\plcomma\plx\plkz \pleq \plx \aland
  \plf\plka\plx\plcomma\ply\plkz \pleq\plx\plkz \alimpl 
  \ply \pleq\plf\plka\plc\plcomma\ply\plkz $ 
  % \\ dazu nutzt man die Tautologie $G\alimpl\alka H\alimpl G\alkz$
\item[Schritt 3:] zeige:  $\plall\plx\, \plka\plf\plka\ply\plcomma\plx\plkz \pleq \plx \aland
  \plf\plka\plx\plcomma\ply\plkz \pleq\plx\plkz \alimpl
\plf\plka\plc\plcomma\ply\plkz \pleq\plc $
  \begin{description}
  \item[Schritt 3.1:] zeige
    $\plall\plx\, \plka\plf\plka\ply\plcomma\plx\plkz \pleq \plx \aland
    \plf\plka\plx\plcomma\ply\plkz \pleq\plx\plkz \alimpl 
    % 
    \plall\plx\, \plka\plf\plka\ply\plcomma\plx\plkz \pleq \plx\plkz \aland
    \plall\plx\, \plka\plf\plka\plx\plcomma\ply\plkz \pleq\plx\plkz
    $
  \item[Schritt 3.2:] zeige $\plall\plx\, \plka\plf\plka\ply\plcomma\plx\plkz \pleq \plx\plkz \aland
    \plall\plx\, \plka\plf\plka\plx\plcomma\ply\plkz \pleq\plx\plkz
    \alimpl
    \plall\plx\, \plka\plf\plka\plx\plcomma\ply\plkz \pleq\plx \plkz
    $
  \item[Schritt 3.3:] also $\plall\plx\, \plka\plf\plka\ply\plcomma\plx\plkz \pleq \plx \aland
    \plf\plka\plx\plcomma\ply\plkz \pleq\plx\plkz
    \alimpl
    \plall\plx\, \plka\plf\plka\plx\plcomma\ply\plkz \pleq\plx \plkz
    $
  \item[Schritt 3.4:] es gilt $
    \plall\plx\, \plka\plf\plka\plx\plcomma\ply\plkz \pleq\plx \plkz \alimpl
    \plf\plka\plc\plcomma\ply\plkz \pleq\plc 
    $
  \item[Schritt 3.5:] also $\plall\plx\, \plka\plf\plka\ply\plcomma\plx\plkz \pleq \plx \aland
    \plf\plka\plx\plcomma\ply\plkz \pleq\plx\plkz
    \alimpl
    \plf\plka\plc\plcomma\ply\plkz \pleq\plc$
  \end{description}
\item[Schritt 4:] zeige $\plall\plx\, \plka\plf\plka\ply\plcomma\plx\plkz \pleq \plx
    \aland \plf\plka\plx\plcomma\ply\plkz \pleq\plx\plkz \alimpl \ply\pleq\plc$
  \begin{description}
  \item[Schritt 4.1]
    mit Ergebnissen von Schritt 2 und Schritt 3 zeige: \\
    $\plall\plx\, \plka\plf\plka\ply\plcomma\plx\plkz \pleq \plx
    \aland \plf\plka\plx\plcomma\ply\plkz \pleq\plx\plkz \alimpl \plka
    \ply \pleq\plf\plka\plc\plcomma\ply\plkz \aland
    \plf\plka\plc\plcomma\ply\plkz \pleq\plc \plkz$
  \item[Schritt 4.2:] zeige $\plka
    \ply \pleq\plf\plka\plc\plcomma\ply\plkz \aland
    \plf\plka\plc\plcomma\ply\plkz \pleq\plc \plkz 
    \alimpl \ply\pleq\plc$
  \item[Schritt 4.3:] also $\plall\plx\, \plka\plf\plka\ply\plcomma\plx\plkz \pleq \plx
    \aland \plf\plka\plx\plcomma\ply\plkz \pleq\plx\plkz \alimpl \ply\pleq\plc$
  \end{description}
\item[Schritt 5:] Aus dem Ergebnis von Schritt 4 folgt durch Generalisierung: \\
  $\plall\ply\, \plka \plall\plx\, \plka\plf\plka\ply\plcomma\plx\plkz \pleq
  \plx \aland \plf\plka\plx\plcomma\ply\plkz \pleq\plx\plkz \alimpl
  \ply\pleq\plc \plkz$ 
\end{description}

%-----------------------------------------------------------------------
\section*{Zusammenfassung und Ausblick}

In diesem Kapitel wurde zunächst die Syntax prädikatenlogischer Formeln
festgelegt.
%
Anschließend haben wir  Interpretationen und Modelle eingeführt.
%
Und am Ende haben wir gesehen, wie man dem semantischen Begriff der
Allgemeingültigkeit den syntaktischen Begriff der Beweisbarkeit, \zB
im Hilbert-Kalkül, so gegenüberstellen kann, dass sich die beiden
entsprechen.

\printunitbibliography

\cleardoublepage

%-----------------------------------------------------------------------
%%%
%%% Local Variables:
%%% fill-column: 70
%%% mode: latex
%%% TeX-master: "../k-13-praedikatenlogik/skript.tex"
%%% TeX-command-default: "XPDFLaTeX"
%%% End:
