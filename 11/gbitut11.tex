%beamer

% Comment/uncomment this line to toggle handout mode
\newcommand{\handout}{}

\input{../framework/PraeambelTut.tex}

\morescalingdelimiters

\begin{document}
\starttut{11}

\section{Organisatorisches}
\begin{frame}{Klausuranmeldung}
	\begin{itemize}
		\item Prüfungstermin: 07. März 2022, 15:00-17:00 Uhr
		\item Prüfungsanmeldung ab \textbf{jetzt} im Campusportal möglich
		\pause
		\item Übungsschein: Anmeldung ab \textbf{jetzt} im Campusportal möglich
		\item Anmelden bis \textbf{06. März 2022}
	\end{itemize}
\end{frame}

\begin{frame}{Neue Corona-Regeln}
	\begin{itemize}
		\item \textbf{FFP2-Maskenpflicht} in allen KIT-Gebäuden
		\implitem unter anderem in Vorlesung, Übung und Tutorium nur Zutritt mit FFP2- oder vergleichbarer Maske
		\item Die restlichen Regeln gelten weiterhin
	\end{itemize}
\end{frame}

\section{Rückblick}
\begin{frame}{Zu Übungsblatt \#9}
	\begin{itemize}[<+->]
		\item Ich habe es leider nicht geschafft alle Blätter zu korrigieren
		\implitem Rückgabe der Blätter nächste Woche
		\item Musterlösung trotzdem im \ILIAS verfügbar
	\end{itemize}
\end{frame}


% Zeit und Anlass
\mycomment{
	\only<handout:0>{
		\lastframe{0.85}{30}{xkcd/the_bdlpswdks_effect_1531.png}{https://www.xkcd.com/1531}
	}
	\only<beamer:0>{
		\lastframe{0.57}{30}{xkcd/the_bdlpswdks_effect_1531.png}{https://www.xkcd.com/1531}
	}
} 

%\begin{frame}{Schwarzes Brett}
%	
%	Zum Bestehen des Übungsscheins reichen \textbf{121.5}~Punkte aus. \\
%	\bigskip
%	(Kann sein, dass weniger auch reicht, aber wer mind. 121.5~P hat, besteht auf jeden Fall.)
%\end{frame}

\framePrevEpisode


%\begin{frame}{Rückblick: Graphen}
%	\begin{itemize}[<+->]
%		\item Gerichtete und ungerichtete Graphen
%		\item Knoten, Kanten, Schlingen, Isomorphie
%		\item Pfade, Wege, Zyklen, Kreise
%		\item Bäume
%		%\item Adjazenzlisten und Adjazenzmatrizen
%	\end{itemize}
%\end{frame}

%\begin{frame}[t]{Wahr oder Falsch?}
%	\begin{figure}
%		\begin{tikzpicture}[->,>=stealth,baseline=-5mm]
%		\matrix[matrix of math nodes,nodes={draw,circle,minimum size=5mm,inner sep=2pt},row sep=10mm,column sep=10mm,ampersand replacement=\&]
%		{
%			|(0)| 0 \& |(1)| 1 \& |(2)| 2 \\
%			\& |(3)| 3 \& \\
%		};
%		\draw  (0) -- (1);
%		\draw  (0) -- (3);
%		\draw  (2)  to [bend left] (3);
%		\draw  (2) -- (1);
%		\draw  (3) to [bend left] (2);
%		\draw  (0) to [bend left]  (2);
%		\path (2) edge [loop right] ();
%		\draw (1) -- (3);
%		\end{tikzpicture}
%	\end{figure}
%	
%	\TrueQuestionE{$G$ ist gerichtet.}{}
%	\FalseQuestionE{$(1,3,2)$ ist wiederholungsfreier Weg in $G$.}{Nein, aber ein Pfad!}
%	\TrueQuestionE{$(1)$ ist ein Pfad in $G$.}{Aber $(1,1)$ nicht!}
%	\FalseQuestion{$()$ ist ein Pfad/Weg in $G$.}
%	\FalseQuestionE{$G$ ist ein DAG.}{$(2,3,2)$ ist ein Zyklus in $G$.}
%\end{frame}

\begin{frame}{Rückblick: Reguläre Ausdrücke}
	Reuläre Ausdrücke bestehen aus:
	\begin{itemize}[<+->]
		\item \hstretchto{\q15uad}{dem leeren Ausdruck $\rx O$} $\lang{\rx{O}} = \emptyset$\pause
		\item \hstretchto{\q15uad}{den einzelnen Symbolen $x \in A$}  $\lang{x}=\{x\}$ \pause
		\item zwei regulären Ausdrücken $R_1$ und $R_2$ mit\\ 
			\hstretchto{\q8uad}{$\rx(R_1 R_2\rx)$} $\lang{R_1 R_2} = \lang{R_1} \cdot \lang{R_2}$\\
			\hstretchto{\q8uad}{$\rx(R_1\rx|R_2\rx)$} $\lang{R_1 \rx| R_2} = \lang{R_1} \cup \lang{R_2}$ \pause
		\item \hstretchto{\q8uad}{einem Stern $R\rx*$} $\lang{R\rx*} = \lang{R}^*$\pause
	\end{itemize}
\end{frame}
\begin{frame}{Aufgabe: Sprachen regulärer Ausdrücke}
	\begin{itemize}
		\item $\lang{\rx{(a|b)*abb(a|b)*}} = \visible<2-|handout:2>{\{\word a, \word b\}^* \cdot \{\word a\word b\word b\} \cdot \{\word a, \word b\}^*}$
		\item $\lang{\rx{a**}} = \visible<3-|handout:2>{\{\word a\}^*}$
		\item $\lang{\visible<4-|handout:2>{R\rx{(}R\rx{)*}}} = \lang{R}^+$ \quad (für bel. reg. Ausdruck $R$)
		%\item $\lang{\visible<5-|handout:2>{\rx{O*}}} = \{\eps\}$
		\item $\lang{\visible<5-|handout:2>{\rx{a*ba*ba*b(a|b)*}}} = \set{ w \in \{\word a, \word b\}^* \Mid \size{w}_{\word b} > 2 } $
		\item $\lang{\visible<6-|handout:2>{\rx{b*a*}}} =$ Sprache aller Wörter über $\set{\word a, \word b}$, in denen das Teilwort \word{ab} nicht vorkommt.
	\end{itemize}
\end{frame}

\input{../Bloecke/RechtslineareGrammatiken.tex}

\input{../Bloecke/Turing.tex}

% Hier CS50 Binary Search (Telefonbuchszene) zeigen! (Spulen!!!)
% https://www.youtube.com/watch?v=o4SGkB_8fFs&list=PLhQjrBD2T382VRUw5ZpSxQSFrxMOdFObl

% Live kam das ganze bisher immer sehr gut an, also wer ein Telefonbuch hat...
% Video geht aber auch, es braucht also nicht zwingend ein Telefonbuch...
% Aber wer eines hat (die Post hat welche!), Live wäre das ganze natürlich schon besser...

% QUESTION (DJ): Wieso hier? Mastertheorem ist doch erst später?
% ANSWER   (TH): Geht auch gar nicht mit dem vereinfachten MT aus dem Tut.
%                Es geht hier eher darum, zur Motivation ein Gefühl für "es gibt langsame und schnelle Algorithmen" zu entwickeln 
%                und außerdem binäre Suche zu verstehen, und wie logarithmische Laufzeiten entstehen und warum sie toll sind.

%\input{../Bloecke/GrossO}

\begin{frame}	
	\begin{block}{Was ihr nun wissen solltet}
		\begin{itemize}
			\item Rechtslineare Grammatiken und Reguläre Ausdrücke
			\item Aufbau + Funktion von Turingmaschinen
			\item Entscheidbarkeit und Aufzählbarkeit
		\end{itemize}
	\end{block}
	
	\begin{block}{Was nächstes Mal kommt}
		\begin{itemize}
			\item Vorschriften zum Rechnen: Algorithmen
			\item Lizenz zum Schludern: O-Kalkül
			\item Einfache Laufzeitabschätzungen
			%TODO nexttut
			%\item Grenzen endlicher Automaten -- Wer kann zählen?
			%\item Ein Regelwerk für einen Ausdruck -- Reguläre Ausdrücke
		\end{itemize}
	\end{block}
\end{frame}

\xkcdframevert{612}{Danke für eure Aufmerksamkeit! \smiley}{2.5}
%\lastframe{0.65}{15}{xkcd/file_transfer_612.png}{http://www.xkcd.com/612}
\slideThanks

\end{document}