%beamer

% Comment/uncomment this line to toggle handout mode
\newcommand{\handout}{}

\input{../framework/PraeambelTut.tex}

\morescalingdelimiters

\begin{document}
\starttut{6}

\section{Rückblick}

\begin{frame}{Zu Übungsblatt \#4}
	\begin{itemize}[<+->]
		\item Ist noch nicht fertig korrigiert, sorry dafür!
		\item Kommt in den nächsten Tagen
	\end{itemize}
\end{frame}

%\thasse{\lastframe{0.65}{25}{xkcd/advent_calendar.png}{https://www.xkcd.com/994/}}

%\begin{frame}{Übungsschein-/Klausuranmeldung}
%	\centering \Large Übungsschein- und Klausuranmeldung sind freigeschaltet! \\
%	\bigskip
%	\Impl \textbf{ANMELDEN}! \textbf{Rechtzeitig}! Am besten \textbf{sofort}, falls man jetzt schon weiß, dass man die Klausur schreibt. \\
%	\bigskip
%	(Übungsschein sowieso anmelden, weil nix zu verlieren!)
%\end{frame}

 \framePrevEpisode

\begin{frame}{Kahoot!}
	\begin{itemize}[<+->]
		\item Kahoot! ist ein anonymes Online-Quiz
		\item Ihr bekommt Punkte für schnelles und richtiges raten
		\item Ich schalte das Quiz frei und ihr könnt über \url{https://kahoot.it} beitreten
		\item Das Kahoot! könnt ihr euch später nochmal unter diesem Link angucken: \\
			\url{https://create.kahoot.it/share/gbi-woche-6-einstieg/8fcb5877-e3d1-4099-b6ef-c0d8193d1ef7}
	\end{itemize}
\end{frame}

%\begin{frame}[t]{Wahr oder falsch?} 
%		% Socrative: https://b.socrative.com/teacher/#import-quiz/31589970
%		
%		\Socrative
%		\TrueQuestionE{$\forall x \in \Z: \quad x \mod 2 = 0 \iff x \text{ ist gerade}$.}{ $\text{Und auch: \quad} x \mod 2 = 1 \iff \text{x ist ungerade}$.}
%		\FalseQuestionE{$4 \cdot (x \div 4) = x$.}{ $ 4 \cdot (x \div 4) + (x \mod 4) = x$}
%		% \TrueQuestionE{$\left(\{\word a\}^\ast\cdot \{\word b,\eps\}\right)^\ast = \{\word a,\word b\}^\ast$}{}  % not fitting here
%		% \FalseQuestionE{Jede Abbildung besitzt eine Umkehrabbildung.}{ Nur bijektive Abbildungen besitzen eine Umkehrabbildung.}  % ditto
%		% \TrueQuestionE{Für jede Abbildung können wir das Urbild angeben.}{}  % ditto
%		\TrueQuestionE{Das Zweierkomplement ist gut zum Rechnen.}{(Für den Rechner schon!)}
%		\FalseQuestionE{Für jede Codierung gilt $h(\eps) = \eps$.}{Das muss nur für Homomorphismen gelten.}
%		\FalseQuestion{Jeder Homomorphismus ist präfixfrei.}
%		\TrueQuestion{Jede Huffman-Codierung ist präfixfrei.}
%		\FalseQuestionE{Jeder $\eps$-freie Homomorphismus ist präfixfrei.}{ Aber: Jeder präfixfreie Homomorphismus ist $\varepsilon$-frei.}		
%		\TrueQuestionE{Präfixfreie Codes sind einfach zu decodieren.}{}
%\end{frame}


\begin{frame}{Rückblick: Codierungen}
	\begin{itemize}[<+->]
		\item Codierung: Injektive Abbildungs (oft, aber nicht immer Homomorphismen)
		\item Homomorphismus $h:A^*\rightarrow B^*$: strukturerhaltende Abbildung ($\forall x,y\in A^*:h(x\cdot y) = h(x)\cdot h(y)$)
		\item $h$ heißt $\eps$-frei gdw. $\forall x\in A^1:h(x)\not = \eps$
		\item $h$ heißt präfixfrei gdw. für keine $x_1,x_2\in A^1$ gilt :$ h(x_1) \text{ ist Präfix von } h(x_2)$
		\implitem präfixfreie Homomorphismen sind $\eps$-frei und injektiv (also Codierungen!)
%		\item Huffman-Codierungen: Präfixfreie Codes mit optimaler Länge
	\end{itemize}
\end{frame}

\input{../Bloecke/Huffmann}

\input{../Bloecke/Zweierkomplement.tex}

%% Inhalt

\input{../Bloecke/Speicher.tex}

% next time: \input{../Bloecke/MIMA.tex}


\begin{frame}	
	\begin{block}{Was ihr nun wissen solltet}
		\begin{itemize}
			\item Wie die Huffman-Codierung funktioniert
			\item Wie man ganze Zahlen ins Zweierkomplement umrechnet
			\item Was man in der theoretischen Informatik unter Speicher versteht
			\item Abbildungen, die Abbildungen auf Abbildungen abbilden!
		\end{itemize}
	\end{block}
	
	\begin{block}{Was nächstes Mal kommt}
		\begin{itemize}
			\item MIMA und Prozessoren – Den Bits beim Arbeiten zuschauen
			%\item Grammatiken – mehr als Subjekt, Objekt, Prädikat
		\end{itemize}
	\end{block}
\end{frame}

\xkcdframevert{835}{Danke für eure Aufmerksamkeit! \smiley}{2.5}
\slideThanks

\end{document}