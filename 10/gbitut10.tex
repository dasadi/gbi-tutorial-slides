%beamer

% TODO:
% - VL-Folien einarbeiten
% - Weniger f_* etc. Theorie, mehr Aufgaben zwischendrin und zu Akzeptoren
% - Bessere Hinleitung zu Akzeptoren

% Comment/uncomment this line to toggle handout mode
\newcommand{\handout}{}

\input{../framework/PraeambelTut.tex}

\morescalingdelimiters

\begin{document}
\starttut{10}

\section{Rückblick}

\begin{frame}{Zu Übungsblatt \#8}
	Bisheriger Schnitt: \quad 12.8 / 20~P

	\begin{itemize}[<+->]
		\item 14 von 23 Tutanden haben etwas abgegeben
		\item Die Musterlösung findet ihr im \ILIAS unter Übungsblätter
		\item Korrekturen gibt es jetzt!
		\item Ihr habt alle pünktlich abgegeben :)
	\end{itemize}
\end{frame}

\begin{frame}{Zu Übungsblatt \#8}
	Die häufigsten Fehler:
	\begin{itemize}[<+->]
		\item[1b)] $x \mod k$ ist \textbf{Falsch}, weil $k$ eine Adresse ist.
		\implitem Richtig: $x \mod k^*$, wobei $k^*$ der Wert an Adresse $k$ ist. 
		\item[2b)] Überschreiben von gegebenen Adressen nicht erlaubt, da hier eine neuer Befehl definiert wird
		\item Außerdem Verwendung von \textit{INC} erlaubt, da in Teil a) definiert. 
		\item[3)] \textit{EQL} vergleicht Wert des Akkumulators mit dem Wert an der gegebenen \textbf{Adresse}
		\implitem Zum Vergleich mit Endadresse muss diese erst abgespeichert werden, damit mit \textit{EQL} vergleichbar 
	\end{itemize}
\end{frame}

\mycomment{ % keine Probeklausur im WS20/21
\daniel{
\begin{frame}{Probeklausur}
	\daniel{Schnitt: \quad \thassedaniel{XX}{22.8}~/~38~P \quad $\hat{=}$ Note \thassedaniel{XX}{3.2} \\}
	\begin{columns}[T] 
		\hspace{.5\baselineskip}
		\begin{column}[T]{.30\textwidth} 
			Skala: \\
			\begin{tabular}{|r|l|}
				\hline
				Note & Punkte \\
				\hline
				5.0	& \hphantom{0}0.0 – 18.5 	\\ \hline
				4.0 & 19.0 – 20.0	\\ \hline
				3.7 & 20.5 – 22.0	\\ \hline
				3.3	& 22.5 – 24.0	\\ \hline
				3.0	& 24.5 – 26.0	\\ \hline
				2.7	& 26.5 – 28.0	\\ \hline
				2.3	& 28.5 – 30.0	\\ \hline
				2.0	& 30.5 – 32.0   \\ \hline
				1.7	& 32.5 – 24.0	\\ \hline
				1.3	& 34.5 – 36.0	\\ \hline
				1.0	& 36.5 – 38.0	\\ \hline
			\end{tabular}
		\end{column}
		\hspace{-\baselineskip}
		\begin{column}[T]{.68\textwidth} 
			\begin{itemize}
				\item Wer regelmäßig Blätter machte, hatte mehr Punkte! :P {\small (As promised...)}
				\item Generell: Schaut euch alte Blätter an und vor allem, \alert{was ich in rot dazugekritzelt} hab. \\
				\impl Manche wiederholen ihre Fehler.
				\item Aufgabenstellung \textbf{genau} lesen! \\
				Z.~B.: „Geben Sie an...“ – Ergebnis langt! \\ \impl Keine Zeit verschwenden! 
				
			\end{itemize}
		\end{column}
	\end{columns}
	
\end{frame}
}

\mycomment{
	\begin{frame}{Schwarzes Brett}
		\begin{itemize}
			\item Es wird insgesamt \textbf{6~Blätter} geben \impl Gesamtpunkte $=$ 205~P
			\implitem Wer \textbf{$\geq$ 102.5~P} hat, \textbf{hat den Schein garantiert}. \\
			{\small (Wer etwas weniger hat: vielleicht auch, keine Ahnung. Stay tuned.)}
			\item Es wird ein \textbf{Bonusübungsblatt} geben, auf dem ihr zusätzl. Punkte sammeln könnt (aber nicht müsst). \\
			Das wird dann (nach Ende der VL-Zeit) beim Übungsleiter Zenkel hinterlegt, sobald korrigiert.
		\end{itemize}
	\end{frame}
}
}

\framePrevEpisode

\begin{frame}{Kahoot!}
	\begin{itemize}[<+->]
		\item Kahoot! ist ein anonymes Online-Quiz
		\item Ihr bekommt Punkte für schnelles und richtiges raten
		\item Ich schalte das Quiz frei und ihr könnt über \url{https://kahoot.it} beitreten
		\item Das Kahoot! könnt ihr euch später nochmal unter diesem Link angucken: \\
			\url{https://create.kahoot.it/share/gbi-woche-10-einstieg/5df0a7d9-d7f8-43c3-bb8a-7484dcc53a55}
	\end{itemize}
\end{frame}

\mycomment{
\begin{frame}{Rückblick: Laufzeitabschätzungen}
	\begin{itemize}[<+->]
		\item Asymptotisches Wachstum
		\item $O, \Theta, \Omega$
		\item Beweisverfahren
		\item Rechenregeln im O-Kalkül
	\end{itemize}

	\pause
	\begin{block}{Typische Laufzeiten}
		Lineare Suche: $\Th{n}$ \\
		Binäre Suche:  $\Th{\log n}$ \\
		Potenzmenge berechnen: $\Th{2^n}$
	\end{block}
\end{frame}

\begin{frame}{Rückblick}
	\centering
	\includegraphics[scale=0.5]{laufzeit/polyVsExp}
\end{frame}

\begin{frame}[t]{Wahr oder Falsch?}
	\TrueQuestionE{$x^4 \in \Oh{{(x^3)}^3}$}{ $ {(x^3)}^3 = x^9$}
	\FalseQuestionE{$\sqrt{n} \in \Om{2^n}$}{}
	\TrueQuestionE{$log_{5000} n \in \Th{\log_2{n^4}}$}{ $ \log_2{n^4} = 4 \cdot \log_2{n}$}
	%
	% ACHTUNG: Sehr aufwendig zu erklären. Besser weglassen, um Zeit zu sparen!
	\FalseQuestionE{$O(f_1) + f_2 = O(f_1 + f_2)$}{Z.~B.: \\ $\Oh{n} + 4 = \set{f(n) + 4 \Mid f(n) \in \Oh{n}} \neq \set{f(n) \Mid f(n) \in \Oh{n}} = \Oh{n+4}$. \\ \textbf{Aber} es gilt: $O(f_1) + O(f_2) = O(f_1 + f_2)$}
\end{frame}

\begin{frame}[t]{Wahr oder Falsch?}
	\FalseQuestion{Für zwei Funktionen $f, g$ gilt immer $f \preceq g$ oder $f \succeq g$.}
	\medskip
	
	\visible<2>{
		Es gibt unvergleichbare Funktionen! Beispiel:
		\begin{align*}
		f(n) &=
		\begin{cases}
		1, & \text{ falls $n$ gerade} \\
		n, & \text{ falls $n$ ungerade} \\
		\end{cases} \\
		g(n) &=
		\begin{cases}
		n, & \text{ falls $n$ gerade} \\
		1, & \text{ falls $n$ ungerade} \\
		\end{cases} \\
		\end{align*}
		Es gilt \textbf{nicht} $g\preceq f$, es gilt \textbf{nicht} $f\preceq
		g$ und es gilt \textbf{erst recht nicht} $f\asymp g$.
	}
\end{frame}
}

\input{../Bloecke/Automaten}

\input{../Bloecke/Akzeptoren}

\input{../Bloecke/Regulaer}

% ---------

\mycomment{
% WiSe 10/11 Aufgabe 6 c 
\begin{frame}{Übung: Akzeptoren}
	Die Sprache $L\subseteq \{\word a,\word b\}^\ast $ sei definiert als die Menge aller Wörter $w$, die folgende Bedingungen erfüllen:
	\begin{align*}
	N_{\word b}(w) &> N_{\word a}(w)\\ 
	\forall v_1,v_2 \in \{\word a,\word b\}^\ast : \qquad w &\neq v_1 \word{bb} v_2 
	\end{align*}
	
	Gebt einen endlichen Akzeptor an, der $L$ erkennt. \\
	
	\bigskip
	\pause
	\begin{block}{Tipps}
		\begin{itemize}[<+->]
			\item Die zweite Bedingung bedeutet: Das Wort darf nirgends zwei \word b hintereinander enthalten.
			\item Was passiert, wenn das Wort mit einem \word a beginnt?\\
			Kann das Wort noch akzeptiert werden?
		\end{itemize}
	\end{block}
\end{frame}

\begin{frame}{Übung: Akzeptoren: Lösung}
	\begin{figure}
		\centering
		\includegraphics[width=0.7\linewidth]{automaten/Loesung2.pdf}
	\end{figure}
\end{frame}
}

\begin{frame}	
	\begin{block}{Was ihr nun wissen solltet}
		\begin{itemize}
			\item Was Automaten sind und wie sie funktionieren
			\item Spezialform: Endlicher Akzeptor!
			\item Ein Regelwerk: Reguläre Ausdrücke
		\end{itemize}
	\end{block}
	
	\begin{block}{Was nächstes Mal kommt}
		\begin{itemize}
			\item Rechtslineare Grammatiken
			\item Turingmaschinen -- mächtiger wird es nicht mehr!
		\end{itemize}
	\end{block}
\end{frame}


\xkcdframevert{1319}{Danke für eure Aufmerksamkeit! \smiley}{2.5}
%\lastframe{0.6}{30}{xkcd/automation_1319.png}{http://www.xkcd.com/1319}
%\xkcdframe{0.5}{30}{xkcd/houston_1438.png}{http://www.xkcd.com/1438}{Oh, hi mom. No, nothing important, just work.}



\slideThanks

\end{document}