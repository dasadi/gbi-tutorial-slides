%beamer

% Comment/uncomment this line to toggle handout mode
\newcommand{\handout}{}

\input{../framework/PraeambelTut.tex}

\morescalingdelimiters

\begin{document}
\starttut{5}

% Zum Aufwärmen: PEBA Tutorenprogramm Aktivierung Code merken (PDF im gleichen Ordner zu finden)

\section{Rückblick}

\begin{frame}{Zu Übungsblatt \#3}
	Schnitt: \quad 12,7 / 20~P

	\begin{itemize}[<+->]
		\item 17 von 23 Tutanden haben etwas abgegeben
		\item Die Musterlösung findet ihr im \ILIAS unter Übungsblätter
		\item Korrekturen gibt es jetzt!
		\item Ihr habt alle pünktlich abgegeben :)
		\item Eigentlich sehr gute Abgaben, viele Punkte bei 3.5 verloren
	\end{itemize}
\end{frame}

\begin{frame}{Zu Übungsblatt \#3}
	Die häufigsten Fehler:
	\begin{itemize}[<+->]
		\item Formulierung: eine Formel \textit{ist} nicht wahr, sondern sie wird \textit{als wahr interpretiert}.
		\item Aussagenlogische Formeln (AL-Formeln) enthalten keinen Wahrheitswert
		\item AL-Formeln enthalten keine Aussagen, wie $x \elem A$
		\item Äquivalenz bei \textbf{AL-Formeln} ist $\equiv$ , bei \textbf{anderen Aussagen} $\gdw$
		\item 3.3b): Das De Morgan'sche Gesetz musste gezeigt werden
		\implitem Das Gesetz selbst darf nicht im Beweis benutzt werden (sonst ist das ein Beweis wie 3.5c))
		\item[] Beweis: $\bleftBr P \boder Q \brightBr \equiv \bnot\bleftBr\bnot \bleftBr P \boder Q \brightBr\brightBr$
		\item[] \textit{Nach Teil a)}:  $\bnot\bleftBr\bnot \bleftBr P \boder Q \brightBr \brightBr \equiv \bnot \bleftBr \bnot P \bund \bnot Q \brightBr$
	\end{itemize}
\end{frame}

\begin{frame}{Zu Übungsblatt \#3}
	Die häufigsten Fehler:
	\begin{itemize}[<+->]
		\item 3.4b): $P=x\elem A$ ist keine korrekte Notation, da die AL-Variable $P$ selbst keinen Wahrheitswert besitzt
		\implitem Modellierung notwendig: z.B. definiere $A \widehat{=} F$ als \textit{Sachverhalt $A$ wird durch AL-Formel $F$ modelliert}
		\item 3.5a) AL-Formel für Aussage 3: \\
		Der Erfolg von Abteilung Q muss den Erfolg von Abteilung P nach sich ziehen
		und darf auf keinen Fall zum Scheitern von Abteilung R führen.
		\implitem $\bleftBr Q \bimp \bleftBr P \bund R \brightBr \brightBr$
		\item 3.5c) Der Beweis ist fehlerhaft: Durch \textbf{Tautologie} $\bleftBr A \bimp A \brightBr$ ist der Erfolg des Beweises
		$\bleftBr A \bimp A \bimp X \brightBr$ nur von der Interpretation von $X$ abhängig.
	\end{itemize}
\end{frame}

 \framePrevEpisode

\begin{frame}{Kahoot!}
	\begin{itemize}
		\item Kahoot! ist ein anonymes Online-Quiz
		\item Ihr bekommt Punkte für schnelles und richtiges raten
		\item Ich schalte das Quiz frei und ihr könnt über \url{https://kahoot.it} beitreten
		\pause
		\item Das Kahoot! könnt ihr euch später nochmal unter diesem Link angucken: \\
			\url{https://create.kahoot.it/share/gbi-woche-5-einstieg/b00f2e0a-0ac4-4df4-ae52-1db141476132}
	\end{itemize}
\end{frame}
 
%\begin{frame}{Rückblick}
%	Auf formale Sprachen können wir \textbf{ähnliche} Operationen anwenden wie auf Wörtern:
%	\begin{itemize}
%		\item $L_1 \cdot L_2 = \{w_1 w_2 \mid w_1 \in L_1 \text{ und } w_2 \in L_2 \}$\\
%		Jeweils ein Wort aus $L_1$ konkateniert mit einem Wort aus $L_2$.
%		\pause
%		\item $L^0 = \{\varepsilon \}, \qquad L^{i+1} = L^i \cdot L$\\
%		Alle Wörter, die aus $i = 0,1,2...$ Wörtern der Sprache zusammengesetzt wurden
%		\pause
%		\item $L^+ = \bigcup \limits_{i=1}^\infty L^i \qquad L^* = L^+ \cup L^0$\\
%		Alle Wörter, die sich aus den Wörtern der Sprache bilden lassen \\ 
%		(ohne/mit zusätzlichem $\eps$ als Würze).
%	\end{itemize}
%	Ein Alphabet selbst ist \textbf{auch} ne formale Sprache, nämlich mit Wörtern der Länge 1.
%\end{frame}

%\begin{frame}[t]{Wahr oder falsch?}
%	\FalseQuestionE{Jede Sprache enthält Wörter.}{ $\emptyset$ ist auch eine gültige Sprache.}
%	\FalseQuestionE{$\word{0}^* = \{\eps, \word{0}, \word{00}, \word{000}, ...\}$}{$\word{0}^*$ gibt es nicht, denn $\word{0} \neq \{\word{0}\}$.}
%	\TrueQuestionE{Es gibt Sprachen $L$, für die gilt $\eps \in L^+$.}{\ZB $L = \set{\eps, \word{aaa}}$.}
%	\FalseQuestionE{$L^+ = L^* \setminus L^0$.}{Gilt nicht, wenn $\varepsilon \in L$.}
%	\TrueQuestionE{$\{\}^* \neq \{\} $.}{ $\{\}^* = \{\varepsilon\}$.}	
%\end{frame}

\input{../Bloecke/FormaleSprachenP2.tex}

\input{../Bloecke/Darstellung}

\mycomment{
	\begin{frame}{Rückblick: Zahlendarstellung}
		\begin{itemize}
			\item Zahlen sind Objekte, die einen festen numerischen Wert haben.\\
			Um eine Zahl aufzuschreiben, benötigen wir aber eine Darstellung. \\
			Die gleiche Zahl kann viele verschiedenen Darstellungen annehmen.
			\item Mit der Auswertungsfunktion $Num_b(\cdot)$ berechnen wir von einer Zahlendarstellung den numerischen Wert.
			\item Mit der Funktion $Repr_b(\cdot)$ können wir eine Zahl in einer beliebigen Darstellung angeben.
		\end{itemize}
		
		Bis jetzt haben wir das alles nur für positive Zahlen gesehen!
	\end{frame}
}

\begin{frame}{Aufgabe: Zahlendarstellung}
	Berechnet die folgenden Zahlenwerte:
	\begin{align*}
		Num_2(\word 1) &= \visible<2->{1} \\
		Num_2(\word{11}) &= \visible<3->{3} \\
		Num_2(\word{111}) &= \visible<4->{7} \\
		Num_2(\word{1111}) &= \visible<5->{15}
	\end{align*}
	
	Gibt es ein allgemeines Muster? \\ \pause[6]
	\delimitershortfall=0pt
	Ja, es gilt $$Num_2(\word 1^\ell) = 2^\ell - 1$$ und allgemein \pause[7] $$Num_b\left({\underbrace{(b-1)}_{\text{\footnotesize $\in Z_b$ }}}^\ell\right) = b^\ell - 1$$
\end{frame}

%\input{../Bloecke/Zweierkomplement.tex}

\begin{frame}{Von einer Zahlendarstellung zur Anderen}
	\vspace{-.2\baselineskip}
	Von Basis $a$ nach Basis $b$ -- eigentlich recht intuitiv:
	$$\fTrans_{b,a} = \fRepr_b \after \fNum_a$$
	Z.~B.
	$$\fTrans_{3,5} = \fRepr_3 \after \fNum_5$$
	
	\begin{block}{Aufgabe}
		Berechne folgende Darstellungen:\\
		\begin{enumerate}[(1)]
			\item $\fRepr_2(42) = \visible<1->{ \word{101010}_2}$ 
			\item $\fTrans_{4,2}(\word{101010}) = \visible<2->{ \word{222}_4}$ 
			\item $\fTrans_{8,10}(\word{42}) = \visible<3->{ \word{52}_8}$ 
			\item $\fTrans_{16,10}(\word{42}) = \visible<4->{ \word{2A}_{16}}$
		\end{enumerate}
	\end{block}

	Was ist bei allen Wörtern gleich? \only<5->{\impl Die Bedeutung!} \\
	Was macht die Rechnungen (2) -- (4) vergleichsweise einfach? \pause[6] \\
	\impl Wir können die Struktur ausnutzen und zeichenweise vorgehen! ($\fTrans_{8,10}(\word{42}) = \fTrans_{8,2}(\word{101} \cdot \word{010}) = \fTrans_{8,2}(\word{101}) \cdot \fTrans_{8,2}(\word{010})$)
\end{frame}

\input{../Bloecke/Codierungen.tex}

%\input{../Bloecke/Huffmann.tex}

\begin{frame}	
	\begin{block}{Was ihr nun wissen solltet}
		\begin{itemize}
			\item Wie man Zahlen anders darstellt
%			\item Wie man das Zweierkomplement bildet
			\item Übersetzungen und Codierungen
%			\item Huffman-Codierung
		\end{itemize}
	\end{block}
	
	\begin{block}{Was nächstes Mal kommt}
		\begin{itemize}
			\item Huffman-Codierung -- möglichst viel Platz sparen
			\item Zweierkomplement -- negative Binärzahlen
			\item Speicher
			\item MIMA -- Die Funktionsweise eines Mikroprozessors
		\end{itemize}
	\end{block}
\end{frame}

\xkcdframevert{153}{Danke für eure Aufmerksamkeit! \smiley}{2.5} % Zweierkomplement
% \lastframe{0.75}{0}{xkcd/tar.png}{https://www.xkcd.com/1168/} % Komprimierung

\slideThanks

\end{document}