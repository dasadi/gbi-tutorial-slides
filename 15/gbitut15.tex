%beamer

% Comment/uncomment this line to toggle handout mode
\newcommand{\handout}{}

\input{../framework/PraeambelTut.tex}

\morescalingdelimiters

\begin{document}
\starttut{15}

\section{Rückblick}

\begin{frame}{Zu Übungsblatt \#12}
	Schnitt: \quad 9,1 / 18~P

	\begin{itemize}[<+->]
		\item 11 von 23 TutandInnen haben etwas abgegeben
		\item Die Musterlösung findet ihr im \ILIAS unter Übungsblätter
		\item Korrekturen gibt es jetzt!
		\item Ihr habt alle pünktlich abgegeben :)
		\item Jeder, der alle Aufgaben bearbeitet hat, hat auch den Übungsschein :)
	\end{itemize}
\end{frame}

\begin{frame}{Zu Übungsblatt \#12}
	Die häufigsten Fehler:
	\begin{itemize}[<+->]
		\item Notation: Bei Graphen/TMs/Automaten Kanten bis an die Knoten zeichnen!
		\item[1)] Nicht jede TM ist ein Akzeptor! Es gibt auch TMs, die nur eine Berechnung durchführen \impl Ausgabe dann Bandzustand nach Terminierung
		\item[2b)] Benutzte Variablen \textbf{immer} definieren!!!
		\item[2c)] Hoar-Tripel lässt sich zwar anwenden, hat aber \textbf{keine Aussagekraft}, da Schleife nicht terminiert
		\item[4b)] Induktion von n+1 nicht aussreichend, da keine direkte Aussage über nächstes n möglich
		\item[] Alternative? 
	\end{itemize}
\end{frame}

\begin{frame}{Aufgabe 12.4b)}
	\includegraphics[width=\textwidth,height=\textheight,keepaspectratio]{A12.4_UB2122.png}
\end{frame}

\begin{frame}{Organisation}
	\begin{itemize}
		\item Übungsschein-Anmeldung noch bis 06.März
		\item Klausur-Anmeldung noch bis 06.März
		\item Nur noch \textbf{2 Wochen} bis zur Klausur \impl Spätestens jetzt mit Lernen anfangen
	\end{itemize}
\end{frame}

\framePrevEpisode

\begin{frame}{Kahoot!}
	\begin{itemize}
		\item Heute noch ein paar Fragen aus dem ganzen Semester!
		\item Ihr bekommt Punkte für schnelles und richtiges raten
		\pause
		\item Ich schalte das Quiz frei und ihr könnt über \url{https://kahoot.it} beitreten
		\item Das Kahoot! könnt ihr euch später nochmal unter diesem Link angucken: \\
			\url{https://create.kahoot.it/share/gbi-ubungsstunde/9733f7de-4557-4cba-ad73-c7cb64578e82}
	\end{itemize}
\end{frame}

\input{../Bloecke/Graphen3Matrix.tex}

\input{../Bloecke/Halbordnungen} %maybe not lets see

\section{Aufgaben}
\begin{frame}{Aufgabe 1: Sprachen}
	\begin{block}{Aufgabe}
		Gegeben seien die folgenden Sprachen:
		\begin{itemize}
			\item[] $L_1 = \left\{\word a^n\word b^m \quad \middle| \quad n,m\in\mathbb{N}_0, n>m\right\}$
			\item[] $L_2 = \left\{\word a^n \word b^m \word c^o \quad \middle| \quad n,o\in\mathbb{N}_0, m\in\mathbb{N}_+\right\}$
		\end{itemize}
		\begin{itemize}
			\item[a)] Geben Sie für $i\in\left\{1,2\right\}$ einen regulären Ausdruck $R_i$ an, sodass gilt: $\lang{R_i}=L_i$. \\
				Falls eine der Sprachen nicht regulär ist, geben Sie eine kontextfreie Grammatik an, die diese Sprache erzeugt.
			\item[b)] Geben Sie für jede reguläre Sprache $L_i$ ($i\in\left\{1,2\right\}$) einen endlichen Akzeptor $A_i$ an, der genau $L_i$ akzeptiert.
		\end{itemize}
	\end{block}
\end{frame}

\begin{frame}{Aufgabe 1: Sprachen}
	\begin{block}{Lösung a)}
		\begin{itemize}
			\item $G_1=(\{S,X\},\{\word{a}, \word{b}\},S, \{S \rightarrow \word{a}S\word{b}|X, X \rightarrow \word{a}X|\word{a}\})$
			\item $R_2=\word{a}\ast\word{b}\word{b}\ast\word{c}\ast$
		\end{itemize}
	\end{block}
	\begin{block}{Lösung b)}
		\begin{center}
			\begin{tikzpicture}[->,>=stealth,shorten >=1pt,auto,node distance=15mm,
			semithick,initial text={}]
			\tikzstyle{every state}=[]
			
			\node[initial,state] (A)                    {$a$};
			\node[state,accepting] (B)  [right of=A]     {$b$};
			\node[state,accepting]		 (M)  [right of=B]		{$c$};
			\node[state]		 (F)  [below of=B]		{$f$};
			
			\path
			(A) edge [loop above]  node {\word a} (A) 
			(A) edge 			  node {\word b} (B) 
			(A) edge 			  node {\word c} (F) 
			(B) edge [loop above]  node {\word b} (B) 
			(B) edge 			  node {\word c} (M)
			(B) edge 			  node {\word a} (F)
			(M) edge [loop above] node {\word c} (M)
			(M) edge 			  node {\word a, \word b} (F)
			(F) edge [loop below] node {\word a, \word b, \word c} (F);
			\end{tikzpicture}
		\end{center}
	\end{block}
\end{frame}

%% Was ihr nun wissen solltet
\section{Schluss}
\def\abbrsize{\footnotesize}
\begin{frame}
	\begin{block}{Und so geht es weiter...}
		\vspace{-.3\baselineskip}
		\begin{itemize}
			\item \textbf{Algo}{\abbrsize rithmen} \textbf{I} \\  -- Mehr zu Algorithmen, Laufzeiten, Datenstrukturen, Graphen
			\item \textbf{T}{\abbrsize echnische} \textbf{I}{\abbrsize nformatik} \\ -- Realisierung von Schaltungen, Prozessoren (MIMA, ...)
			\item \textbf{T}{\abbrsize heoretische} \textbf{G}{\abbrsize rundlagen der} \textbf{I}{\abbrsize nformatik} \\ -- Mehr zu Grammatiken, Komplexität, Entscheidbarkeit, Turingmaschinen
		\end{itemize}
	\end{block}
\end{frame}

%TODO replacement!!! Old xkcd
{\xkcdframevert{2374}{Viel Erfolg  bei euren Klausuren! \smiley}{2.5}}
%\thasse{\lastframe{0.5}{0}{xkcd/proofs_1724.png}{https://www.xkcd.com/1724/}}

\slideThanks
\end{document}