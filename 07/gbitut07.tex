%beamer

% Comment/uncomment this line to toggle handout mode
\newcommand{\handout}{}

\input{../framework/PraeambelTut.tex}

\morescalingdelimiters

\begin{document}
\starttut{7}

% Was hat das mit MIMA zu tun? => Vllt. ans Ende?
% \lastframe{0.40}{25}{xkcd/password_strength.png}{https://www.xkcd.com/936/}

\mycomment{
	\begin{frame}{Zu Blatt \#2}
		Durchschnitt: etwa \thassedaniel{60}{61} \% der Punkte 
		\begin{itemize}		
			\item Vollständige Induktion: Eine Standard-IV lautet \\
			{\enquote{für \textbf{\alert{ein}} $n$ gilt...}}.\\
			Für alle $n$ wollt ihr's ja \textbf{zeigen}!
			\item \textbf{A~2.2}: IV mit „$K=L$“ funktioniert nicht: \\ 
			Das ist ne $\forall$-Aussage wegen der $\bigcup\limits^\infty_{n=0}$-Definition \\
			\impl Besser $K_n = L_n$ zeigen
			\item \textbf{A~2.3}: hat 11 statt 10 Gesamtpunkte (Zenkel hat sich verrechnet...~:P) \\
			\impl Nicht wundern, wenn ihr einen Punkt mehr habt, als ihr solltet
			\item \textbf{A~2.3b}: IA für $n=0 \textbf{ und } 1$, \; IS: $n \?> n+\mathbf{2}$ \quad --- alles andere geht nicht
			\item \textbf{A~2.6}: \emph{Surjektivität} per Induktion geht NICHT!
			\item \textbf{A~2.7}: \emph{Induktive} Definitionen: KEINE Set comprehensions mit Wortlängen-Magie! (Sonst könnt ihr direkt die Def. abschreiben)
		\end{itemize}
		
	\end{frame}
}

\section{Rückblick}

\begin{frame}{Zu Übungsblatt \#4}
	Schnitt: \quad 13,6 / 20~P

	\begin{itemize}[<+->]
		\item 19 von 23 Tutanden haben etwas abgegeben
		\item Die Korrektur und die Musterlösung findet ihr im ILIAS-Aufgaben-Objekt
		\item Nur die Person, die das Übungsblatt abgegeben hat, bekommt die Rückmeldung \impl tauscht euch aus!
		\item Ihr habt alle pünktlich abgegeben :)
		\item Korrektur kam erst diese Woche, sorry dafür!
	\end{itemize}
\end{frame}

\begin{frame}{Zu Übungsblatt \#4}
	Die häufigsten Fehler:
	\begin{itemize}[<+->]
		\item Aufgabe 4.2: Aufgabenstellung \textbf{genau} lesen und \textbf{nur} die erlaubten Zeichen benutzen!
		\item Aufgabe 4.4c): im IS einer vollständigen Induktion müsst ihr die IV verwenden!
	\end{itemize}
\end{frame}

\begin{frame}{Zu Übungsblatt \#5}
	Schnitt: \quad 8,1 / 17~P

	\begin{itemize}[<+->]
		\item 19 von 23 Tutanden haben etwas abgegeben
		\item Die Korrektur und die Musterlösung findet ihr im ILIAS-Aufgaben-Objekt
		\item Nur die Person, die das Übungsblatt abgegeben hat, bekommt die Rückmeldung \impl tauscht euch aus!
		\item Ihr habt alle pünktlich abgegeben :)
	\end{itemize}
\end{frame}

\begin{frame}{Zu Übungsblatt \#5}
	Die häufigsten Fehler:
	\begin{itemize}[<+->]
		\item Viele Probleme mit den Aufgabenstellungen:
		\item Aufgabe 5.1: Zeigen, dass $h_3$ ein Homomorphismus ist.
		\item[] Das macht man, indem man $\forall x,y\in A^* : h_3(xy)=h_3(x)h_3(y)$ zeigt.
		\item Aufgabe 5.2b): Beweise, dass Aussage A eine notwendige und hinreichende Bedingung für Aussage B ist.
		\item[] Das macht man, indem man A \impl B und B \impl A zeigt.
		\item Aufgabe 5.3b): Für \textbf{jedes} ${i\in\mathbb{N}_0} g^i(w)$ und $f(g^i(w))$ angeben.
		\item Aufgabe 5.3d): Im Beweis kleinschrittig vorgehen. Jeder Schritt sollte ohne Händewedeln verständlich sein.
		\item Aufgabe 5.4: Die war einfach schwer :/
	\end{itemize}
\end{frame}

\framePrevEpisode

\begin{frame}{Kahoot!}
	\begin{itemize}[<+->]
		\item Kahoot! ist ein anonymes Online-Quiz
		\item Ihr bekommt Punkte für schnelles und richtiges raten
		\item Ich schalte das Quiz frei und ihr könnt über \url{https://kahoot.it} beitreten
		\item Das Kahoot! könnt ihr euch später nochmal unter diesem Link angucken: \\
			\url{https://create.kahoot.it/share/gbi-woche-7-einstieg/6c223916-02c7-41bc-82b0-ec082b026ca3}
	\end{itemize}
\end{frame}

%\begin{frame}[t]{Wahr oder falsch?}
%	\Socrative
%	\FalseQuestionE{$\memwrite(m,a,v) \in \Val$.}{$\memwrite(m,a,v) \in \Val^\Adr$.}
%	\FalseQuestionE{$\memread(\memwrite(m,a,v), a) = m(a)$ \quad für alle $m,a,v$.}{$\memread(\memwrite(m,a,v), a) = v$.}
%	\FalseQuestionE{Im IR wird die Adresse des aktuellen Befehls gespeichert.}{Das geschieht im IAR – Im IR steht der Befehl selbst.}
%	%\TrueQuestionE{Die Befehlsholphase ist jede Ausführungsrunde identisch.}{}
%	\TrueQuestionE{Ein MIMA-Programm, welches von $-1$ abwärts zählt, kommt \\ irgendwann bei null raus.}{Unterlauf: Nach $-2^{23}$ kommt $2^{23}-1$.}
%\end{frame}

%\begin{frame}[t]{Wahr oder falsch?}
%	\FalseQuestionE{Der Akku führt bei der MIMA Berechnungen aus.}{Das macht die ALU. Im Akku wird das letzte Ergebnis zwischengespeichert.}
%	\FalseQuestionE{LDC $-5$ lädt $-5$ in den Akku.}{LDC funktioniert nicht mit negativen Konstanten!}
%	\FalseQuestionE{Mit der MIMA können wir Zufallszahlen erzeugen.}{Die MIMA arbeitet rein deterministisch und damit ohne jeden Zufall. 
%		\vspace{-.8\baselineskip}
%		\begin{figure}[H]
%			\centering
%			\includegraphics[scale=0.6]{xkcd/random_number} \\
%			%\vspace{-7pt}
%			{\url{http://xkcd.com/221}}
%		\end{figure}
%	}
%\end{frame}

\input{../Bloecke/MIMA.tex}

\begin{frame}{Noch offen: Klammerausdrücke}
	A long, long time ago, in a land far away:\\
	\medskip
	Formale Sprachen angeben durch $\set{}, \*, {}^+, {}^*$...\\
	Was ist mit der \textbf{Sprache aller gültigen Klammerausdrücke}? Können wir die auch so angeben?\\[1em]
	\pause
	\impl Jetzt wissen wir: \textbf{Nein}, das geht nicht! (Siehe VL)\\[1em]
	
	\begin{figure}[H]
		\centering
		\includegraphics[scale=0.7]{xkcd/(.png}
		\vspace{-7pt}
		\caption{ \texttt{\url{https://xkcd.com/859/}} }
	\end{figure}
\end{frame}

\input{../Bloecke/Grammatiken.tex}

% Nicht unbed. nötig. Je nach Zeitbudget.
%\input{../Bloecke/Relationen2.tex}

\begin{frame}	
	\begin{block}{Was ihr nun wissen solltet}
		\begin{itemize}
			%\item Mehr Eigenschaften von Relationen
			\item Was die MIMA ist
			\item Wie man Programme in der MIMA schreibt
			\item Was eine kontextfreie Grammatik ist
			%\item Wie man Sprachen mit Grammatiken beschreiben kann
			%\item Welche Eigenschaften Relationen haben können
		\end{itemize}
	\end{block}
	
	\begin{block}{Was nächstes Mal kommt}
		\begin{itemize}
			\item Wie man Sprachen mit Grammatiken beschreiben kann
			\item Noch mehr Grammatiken
			\item Mehr Eigenschaften von Relationen
			%\item Nachts sind alle Katzen grau -- Prädikatenlogik
			%\item Algorithmen: Kochrezepte der Informatik
		\end{itemize}
	\end{block}
\end{frame}	

\xkcdframe{327}{Danke für eure Aufmerksamkeit! \smiley}{2.5}

\slideThanks

\end{document}