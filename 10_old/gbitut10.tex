%beamer

% Comment/uncomment this line to toggle handout mode
%\newcommand{\handout}{}

\input{../framework/PraeambelTut.tex}

\morescalingdelimiters

\begin{document}
\starttut{10}

\section{Rückblick}

\begin{frame}{Zu Übungsblatt \#7}
	Bisheriger Schnitt: \quad 12.8 / 20~P

	\begin{itemize}[<+->]
		\item 14 von 23 Tutanden haben etwas abgegeben
		\item Die Musterlösung findet ihr im \ILIAS unter Übungsblätter
		\item Korrekturen gibt es jetzt!
		\item Ihr habt alle pünktlich abgegeben :)
	\end{itemize}
\end{frame}

\begin{frame}{Zu Übungsblatt \#7}
	Die häufigsten Fehler:
	\begin{itemize}[<+->]
		\item Versucht, euch kurz und präzise auszudrücken \small{(Zwei Seiten Text für 1,5 Punkte sind einfach zu viel)}
		\implitem spart euch und mir Zeit, die in der Klausur knapp sein wird
		\implitem bei langen Formulierungen wird's oft schwammig
		\item[1)] Induktion muss nicht sein, das ging besser ohne
		\item[2a)] $\varepsilon$ nicht vergessen, das ist ein typischer Randfall
		\item[2c)] auch hier formal argumentieren:
		\item[] Angenommen es existiert $P''$ sodass $L(G'')=\{\word{aa},\word b\}^*$. Dann gilt $S \derives^* \varepsilon$ und wegen $S\rightarrow \word aS \in P$ auch $S \derives \word aS \derives^* \word a$ und damit $a\in L(G'')=\{\word{aa},\word b\}^*$. Widerspruch.
		\item[4d.ii)] Für unendliche Mengen haben wir die Kardinalität nicht definiert! 
	\end{itemize}
\end{frame}

\begin{frame}{Zu Übungsblatt \#8}
	Bisheriger Schnitt: \quad 12.8 / 19~P

	\begin{itemize}[<+->]
		\item 15 von 23 Tutanden haben etwas abgegeben
		\item Die Korrektur und die Musterlösung findet ihr wie immer im ILIAS-Aufgaben-Objekt
	\end{itemize}
\end{frame}

\begin{frame}{Zu Übungsblatt \#8}
	Die häufigsten Fehler:
	\begin{itemize}[<+->]
		\item[1)] Prädikatenlogische Formeln aufstellen: Wenn wir PL-Formeln formal betrachten, müsst ihr auf alle Formalien achten!
		\item Relationen als Mengen interpretieren, z.B. $I(P)=\left\{ x \in \mathbb{N}_0 \middle| x \text{ gerade }\right\}$ oder $I(R)= \left\{(x,y) \in \mathbb{N}_0 \times \mathbb{N}_0 \middle| x = 2y \right\}$
		\item Genau auf die Formulierung der Aufgabe achten, z.B. ``\textbf{entweder} oder''
		\item[2)] Linkstotalität durch einfache Schlussfolgerungen beweisen
		\item[] z.B. Wenn $S\circ R$ linkstotal, dann $\forall x \in M \exists z\in M : x(S\circ R)z$, also nach Def. von $\circ$ insb. $\forall x\in M \exists y,z\in M : xRy \aland ySz$ und daraus folgt $\forall x \in M \exists y \in M : xRy$
	\end{itemize}
\end{frame}

\begin{frame}{Zu Übungsblatt \#8}
	Die häufigsten Fehler:
	\begin{itemize}[<+->]
		\item[3b)] Dass zwei PL-Formeln nicht logisch äquivalent sind, zeigt man mit einem konkreten Gegenbeispiel:
		\item[] eine Interpretation (und ggf. Variablenbelegung), für die eine Formel wahr und die andere falsch ist.
		\item In PL-Formeln darf man formal nichts einsetzen. Für Auswertungen die Auswertungsfunktion $val_{D,I,\beta}$ benutzen!
		\item[4)] Beim Umgang mit PL-Formeln müssen Variablensymbole unterschieden werden: $x$ ist ein Symbol, $\beta(x)$ sein Wert
	\end{itemize}
\end{frame}


%\lastframe{0.42}{35}{xkcd/dfs_761.png}{https://www.xkcd.com/761}

\framePrevEpisode

\begin{frame}{Kahoot!}
	\begin{itemize}[<+->]
		\item Kahoot! ist ein anonymes Online-Quiz
		\item Ihr bekommt Punkte für schnelles und richtiges raten
		\item Ich schalte das Quiz frei und ihr könnt über \url{https://kahoot.it} beitreten
		\item Das Kahoot! könnt ihr euch später nochmal unter diesem Link angucken: \\
			\url{https://create.kahoot.it/share/gbi-woche-10-einstieg/05064fb0-465a-4382-bf7b-3c4d37a9fb86}
	\end{itemize}
\end{frame}


%\begin{frame}[t]{Wahr oder Falsch?}
%	\Socrative
%	\FalseQuestionE{Die Korrektheit eines Algorithmus kann man immer durch Testen beweisen.}{{\textbf{Nein, nein, nein!}} Durch Testen sieht man höchstens, wo Fehler sind, aber nicht, ob es keine gibt.}
%	\FalseQuestionE{Sinnvolle Schleifeninvarianten kann man \enquote{nach Kochrezept} aufstellen.}{Nein, das Aufstellen sinnvoller Schleifeninvarianten erfordert viel Übung und kann insbesondere \textbf{nicht} vom Rechner gemacht werden.}
%	\FalseQuestionE{Ein Algorithmus ist ein (kompilierbares/ausführbares) Programm.}{Nein, der Algorithmus selbst ist nur die Beschreibung der (Rechen-)\\ Vorschriften, nicht die tatsächliche Umsetzung in Programmcode.}
%	\mycomment{
%		\TrueQuestion{$x = y \aland y = z \impl x = z$ ist allgemeingültig}
%		\FalseQuestion{$x = z \impl x = y \aland y = z$ ist allgemeingültig}
%	}
%\end{frame}


%\begin{frame}[t]{Wahr oder falsch?}
%	\FalseQuestionE{HT-I wird auf Iterationen angewendet.}{HT-I bei \kw{if}-Verzweigungen.}
%	\FalseQuestionE{Ein laut Kalkül gültiges Hoare-Tripel garantiert die Korrektheit \\ des Algorithmus.}{Es muss auch gezeigt werden, dass der Algorithmus terminiert.}
%	%\TrueQuestion{Der Hoare-Kalkül hilft einem, die Korrektheit eines Algorithmus zu beweisen.}
%	%\FalseQuestionE{Der Hoare-Kalkül besteht aus den Axiomen HT-A, HT-E, HT-I, \\ HT-W, HT-S.}{Nur HT-A ist Axiom, die anderen sind Schlussregeln.}
%\end{frame}

\input{../Bloecke/Hoare.tex}

\input{../Bloecke/Graphen}

\begin{frame}	
	\begin{block}{Was ihr nun wissen solltet}
		\begin{itemize}
			\item Wie der Hoare-Kalkül funktioniert % TODO maybe
			\item Wie man mit dem Hoare-Kalkül ein Programm beweist.
			\item Grundbegriffe der Graphen
			%\item Zentrale Eigenschaften von Graphen
		\end{itemize}
	\end{block}
	
	\begin{block}{Was nächstes Mal kommt}
		\begin{itemize}
			\item Zentrale Eigenschaften von Graphen
			\item Wie man Graphen darstellen kann
			\item Warum dauert das so lange? -- Laufzeitbetrachtungen
			%\item One to rule them all -- Das Master-Theorem
			%\item Alles nur von Hand? -- Hier kommen die Automaten!
		\end{itemize}
	\end{block}
\end{frame}

\xkcdframe{974}{Danke für eure Aufmerksamkeit! \smiley}{2.5}
%\lastframe{0.65}{0}{xkcd/the_general_problem_974.png}{http://www.xkcd.com/974}
\slideThanks

\end{document}