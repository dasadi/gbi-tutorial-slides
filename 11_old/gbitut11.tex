%beamer

% Comment/uncomment this line to toggle handout mode
\newcommand{\handout}{}

\input{../framework/PraeambelTut.tex}

\morescalingdelimiters

\begin{document}
\starttut{11}

\section{Rückblick}
\begin{frame}{Zu Übungsblatt \#9}
	Bisheriger Schnitt: \quad 12 / 17~P

	\begin{itemize}
		\item viele schöne Abgaben, weiter so!
		\item 16 von 23 Tutanden haben etwas abgegeben
		\item Die Korrektur und die Musterlösung findet ihr wie immer im ILIAS-Aufgaben-Objekt
		\item Wenn ihr Fragen habt, meldet euch bei mir
	\end{itemize}
\end{frame}

\begin{frame}{Zu Übungsblatt \#9}
	Die häufigsten Fehler:
	\begin{itemize}[<+->]
		\item[1)] Man schreibt $D=...$, nicht $I(D)=...$
		\item[1d)] bei Substitutionen keine Klammern verändern
		\item[2b)] für $n \in \mathbb{N}_0$, $f : A \functionto B$, $x \in A$ ist $f^{n+1}(x)=(f^n \circ f)(x) = f^n(f(x))$...
		\item[] ...und in der Regel nicht $f^{n+1}(x)=f^n(x)\cdot f(x)$
		\item[3d)] Schaut euch die Musterlösung an
	\end{itemize}
\end{frame}

\begin{frame}{Klausurvorbereitung}
	1. Materialien anschauen
	\begin{itemize}
		\item Folien der Vorlesung
		\item Skript der Vorlesung
		\item Folien des Tutoriums
	\end{itemize}

	\pause
	2. Zusammenfassung oder Karteikarten schreiben \\

	\pause
	3. Gaaanz viele Aufgaben rechnen
	\begin{itemize}
		\item Übungsblätter nochmal bearbeiten
		\item Übungsblätter aus vergangenen Jahren bearbeiten
		\item \textbf{$\rightarrow$ Altklausuren $\leftarrow$} rechnen
		\item Aufgabenarchiv: \url{http://gbi.ira.uka.de/archiv/}
	\end{itemize}
\end{frame}


% Zeit und Anlass
\mycomment{
	\only<handout:0>{
		\lastframe{0.85}{30}{xkcd/the_bdlpswdks_effect_1531.png}{https://www.xkcd.com/1531}
	}
	\only<beamer:0>{
		\lastframe{0.57}{30}{xkcd/the_bdlpswdks_effect_1531.png}{https://www.xkcd.com/1531}
	}
} 

%\begin{frame}{Schwarzes Brett}
%	
%	Zum Bestehen des Übungsscheins reichen \textbf{121.5}~Punkte aus. \\
%	\bigskip
%	(Kann sein, dass weniger auch reicht, aber wer mind. 121.5~P hat, besteht auf jeden Fall.)
%\end{frame}

\framePrevEpisode

\begin{frame}{Kahoot!}
	\begin{itemize}[<+->]
		\item Kahoot! ist ein anonymes Online-Quiz
		\item Ihr bekommt Punkte für schnelles und richtiges raten
		\item Ich schalte das Quiz frei und ihr könnt über \url{https://kahoot.it} beitreten
		\item Das Kahoot! könnt ihr euch später nochmal unter diesem Link angucken: \\
			\url{https://create.kahoot.it/share/gbi-woche-11-einstieg/7a9bac61-e7d5-455b-9b84-0623e2cf0d7f}
	\end{itemize}
\end{frame}

%\begin{frame}{Rückblick: Graphen}
%	\begin{itemize}[<+->]
%		\item Gerichtete und ungerichtete Graphen
%		\item Knoten, Kanten, Schlingen, Isomorphie
%		\item Pfade, Wege, Zyklen, Kreise
%		\item Bäume
%		%\item Adjazenzlisten und Adjazenzmatrizen
%	\end{itemize}
%\end{frame}

%\begin{frame}[t]{Wahr oder Falsch?}
%	\begin{figure}
%		\begin{tikzpicture}[->,>=stealth,baseline=-5mm]
%		\matrix[matrix of math nodes,nodes={draw,circle,minimum size=5mm,inner sep=2pt},row sep=10mm,column sep=10mm,ampersand replacement=\&]
%		{
%			|(0)| 0 \& |(1)| 1 \& |(2)| 2 \\
%			\& |(3)| 3 \& \\
%		};
%		\draw  (0) -- (1);
%		\draw  (0) -- (3);
%		\draw  (2)  to [bend left] (3);
%		\draw  (2) -- (1);
%		\draw  (3) to [bend left] (2);
%		\draw  (0) to [bend left]  (2);
%		\path (2) edge [loop right] ();
%		\draw (1) -- (3);
%		\end{tikzpicture}
%	\end{figure}
%	
%	\TrueQuestionE{$G$ ist gerichtet.}{}
%	\FalseQuestionE{$(1,3,2)$ ist wiederholungsfreier Weg in $G$.}{Nein, aber ein Pfad!}
%	\TrueQuestionE{$(1)$ ist ein Pfad in $G$.}{Aber $(1,1)$ nicht!}
%	\FalseQuestion{$()$ ist ein Pfad/Weg in $G$.}
%	\FalseQuestionE{$G$ ist ein DAG.}{$(2,3,2)$ ist ein Zyklus in $G$.}
%\end{frame}

\begin{frame}{Rückblick: Graphen}
	\begin{itemize}[<+->]
		\item Ein \textbf{Graph} ist ein Paar $G=(V,E)$ mit einer endlichen, nichtleeren \textbf{Knotenmenge} $V$ und einer \textbf{Kantenmenge} $E$.
		\item \textbf{gerichteter Graph}: $E \subseteq V \times V$
		\item \textbf{ungerichteter Graph}: $E = \left\{ \left\{u,v\right\} \middle| u \text{ ist mit } v \text{ verbunden }\right\}$
		\item Man kann einen Graphen durch das Paar $G=(V,E)$ oder als Zeichnung angeben
		\item Ein \textbf{Teilgraph} ist ein Graph $G'=(V',E')$ mit $V' \subseteq V$, $E' \subseteq E \cap V' \times V'$ (bzw. $E' \subset E \cap 2^{V'}$)
		\item Eine Schlinge ist eine Kante, die von einem Knoten zu sich selbst führt.
	\end{itemize}
\end{frame}

\input{../Bloecke/Graphen}

\input{../Bloecke/Graphen2}

\mycomment{
	\begin{frame}{Wiederholung: Adjazenzmatrix}
		\begin{Definition}
			Die \textbf{Adjazenzmatrix} eines Graphen $(V, E)$ mit $n$ Knoten ist die Matrix $A\in \{0,1\}^n\times\{0,1\}^n$ mit $$ A_{ij} = \begin{cases} 0 & (i,j) \notin E \\ 1 & (i,j) \in E \end{cases} $$ 
		\end{Definition}
	\end{frame}
}

\begin{frame}{Übung: Adjazenzmatrix}
	\begin{columns}
		\column{0.4\linewidth}
		$$ A = 
		\begin{pmatrix} 
		1 & 1 & 0 & 1 & 0 & 0 \\ 
		0 & 0 & 0 & 1 & 0 & 0 \\ 
		0 & 1 & 1 & 1 & 0 & 0 \\ 
		0 & 0 & 1 & 0 & 0 & 0 \\
		1 & 0 & 0 & 0 & 0 & 0 \\
		0 & 0 & 0 & 0 & 0 & 0 \\
		\end{pmatrix} $$
		
		\column{0.4\linewidth}
		\only<7->{
			\begin{tikzpicture}[->,>=stealth,baseline=-5mm]
			\matrix[matrix of math nodes,nodes={draw,circle,minimum size=6mm,inner sep=2pt},row sep=13mm,column sep=10mm,ampersand replacement=\&]
			{
				|(0)| 0 \& |(1)| 1 \& |(2)| 2 \\
				|(4)| 4 \& |(3)| 3 \& |(5)| 5 \\
			};
			\draw  (0) -- (1);
			\draw  (0) -- (3);
			\draw  (4) -- (0);
			\draw  (1) -- (3);
			\draw  (2)  to [bend left] (3);
			\draw  (2) -- (1);
			\draw  (3) to [bend left] (2);
			\path  (2) edge [loop right] ();
			\path  (0) edge [loop left] ();
			\end{tikzpicture}
		}
	\end{columns}
	
	\bigskip
	Was kann man an der Adjazenzmatrix ablesen?
	\begin{itemize}
		\item Gerichtet oder ungerichtet? \\ \pause
		\impl gerichtet (weil $A$ nicht symm.) \pause
		\item Schlingen? \\ \pause
		\impl Auf der Diagonalen von $A$: \quad Knoten 0, 2 \pause
		\item Zusammenhängend? \\ \pause
		\impl Nein (5 ist isoliert). 
	\end{itemize}
\end{frame}

\input{../Bloecke/Graphen3Matrix}

% Hier CS50 Binary Search (Telefonbuchszene) zeigen! (Spulen!!!)
% https://www.youtube.com/watch?v=o4SGkB_8fFs&list=PLhQjrBD2T382VRUw5ZpSxQSFrxMOdFObl

% Live kam das ganze bisher immer sehr gut an, also wer ein Telefonbuch hat...
% Video geht aber auch, es braucht also nicht zwingend ein Telefonbuch...
% Aber wer eines hat (die Post hat welche!), Live wäre das ganze natürlich schon besser...

% QUESTION (DJ): Wieso hier? Mastertheorem ist doch erst später?
% ANSWER   (TH): Geht auch gar nicht mit dem vereinfachten MT aus dem Tut.
%                Es geht hier eher darum, zur Motivation ein Gefühl für "es gibt langsame und schnelle Algorithmen" zu entwickeln 
%                und außerdem binäre Suche zu verstehen, und wie logarithmische Laufzeiten entstehen und warum sie toll sind.

%\input{../Bloecke/GrossO}

\begin{frame}	
	\begin{block}{Was ihr nun wissen solltet}
		\begin{itemize}
			\item Mehr Eigenschaften von Graphen
			\item Wie man Graphen darstellen kann
			\item Adjazenz- und Wegematrizen % und wie man sie (schnell) ausrechnen kann
		\end{itemize}
	\end{block}
	
	\begin{block}{Was nächstes Mal kommt}
		\begin{itemize}
			\item Lizenz zum Schludern: O-Kalkül
			\item Einfache Laufzeitabschätzungen mit dem Master-Theorem
			\item Endliche Automaten: Mealy und Moore
			\item Endliche Akzeptoren
			%TODO nexttut
			%\item Grenzen endlicher Automaten -- Wer kann zählen?
			%\item Ein Regelwerk für einen Ausdruck -- Reguläre Ausdrücke
		\end{itemize}
	\end{block}
\end{frame}

\xkcdframevert{612}{Danke für eure Aufmerksamkeit! \smiley}{2.5}
%\lastframe{0.65}{15}{xkcd/file_transfer_612.png}{http://www.xkcd.com/612}
\slideThanks

\end{document}