%beamer

% Comment/uncomment this line to toggle handout mode
\newcommand{\handout}{}

\input{../framework/PraeambelTut.tex}

\morescalingdelimiters

\begin{document}
\starttut{13}

\section{Rückblick}

\begin{frame}{Zu Übungsblatt \#11}
	Bisheriger Schnitt: \quad 12.7 / 21~P

	\begin{itemize}
		\item 13 von 23 Tutanden haben etwas abgegeben
		\item Die Korrektur und die Musterlösung findet ihr wie immer im ILIAS-Aufgaben-Objekt
		\item Wenn ihr Fragen habt, meldet euch bei mir
	\end{itemize}
\end{frame}

\begin{frame}{Zu Übungsblatt \#11}
	Die häufigsten Fehler:
	\begin{itemize}[<+->]
		\item[1d)] ``Wie viele Knoten dürfen \textit{höchstens} in $F$ sein, damit ...''
		\implitem Begründen, warum für $|F| = 2$ nicht immer ein sicherer Weg existiert \textbf{und} dass für $|F|=1$ immer ein sicherer Weg existiert.
		\item[3)] Aussagen im O-Kalkül begründet man mit den \textit{Rechenregeln aus der Vorlesung} oder anhand der \textit{Definitionen von asymptotischen Wachstum}.
		\item Dabei \textbf{präzise} sein, z.B.: $2^{\sqrt{n}}$ hat nicht die Form ``$a^n$ für ein $a\in \mathbb{R}^+$''.
		\item $f_1 \preceq f_2$ bedeutet nicht unbedingt $f_2 \not\preceq f_1$
		\item \textbf{Auf keinen Fall} $f=\Oh{g}$ schreiben! $\Oh{g}$ ist eine Menge von Funktionen und man schreibt $f\in\Oh{g}$
		\item[*)] ``Laut Tutorium'' ist keine gute Begründung \smiley
	\end{itemize}
\end{frame}

\begin{frame}{Schwarzes Brett}
	\begin{itemize}
		\item Korrektur: Ein zusammenhängender ungerichteter Graph ist genau dann ein Baum, wenn es in ihm keine echten Kreise \textit{und keine Schleifen} gibt.
		\item Anmeldung \textbf{Klausur} $+$ \textbf{Übungsschein} nicht vergessen!
		\item Erinnerung: Klausur am \Klausurtermin
		\item Bonus: Übungsstunde am 22. Februar 2021, 12:00 - 13:30 Uhr
	\end{itemize}
\end{frame}

\framePrevEpisode

\begin{frame}{Kahoot!}
	\begin{itemize}[<+->]
		\item Kahoot! ist ein anonymes Online-Quiz
		\item Ihr bekommt Punkte für schnelles und richtiges raten
		\item Ihr könnt mit eurem Handy oder PC über \url{https://kahoot.it} mitspielen
		\item Das Kahoot! könnt ihr euch später nochmal unter diesem Link angucken: \\
			\url{https://create.kahoot.it/share/gbi-woche-13-einstieg/9cfc29f7-2261-49ce-974b-9ca3b305c81b}
	\end{itemize}
\end{frame}

%\begin{frame}[t]{Wahr oder Falsch?}
%	\FalseQuestionE{Das (komplizierte) Master-Theorem kann man immer anwenden.}{ Nur bei rekursiven Algorithmen, bei denen das Problem in gleich große Teilprobleme aufgeteilt wird.}
%	\FalseQuestionE{Jeder Moore-Automat kann in einen Mealy-Automaten umgewandelt werden, der für jedes Wort die gleiche Ausgabe produziert.}{ Für das leere Wort kann ein Mealy-Automat niemals eine Ausgabe produzieren.}
%	\TrueQuestionE{Endliche Akzeptoren sind Moore-Automaten mit dem Ausgabealphabet $\{\word 0,\word 1\}$.}{}
%	\FalseQuestionE{Mit endlichen Automaten kann jede beliebige Sprache erkannt \\ werden.}{Tatsächlich ist die Menge der akzeptierbaren Sprachen sogar sehr eingeschränkt.}	
%\end{frame}

\input{../Bloecke/Akzeptoren}

\input{../Bloecke/Turing}

\input{../Bloecke/TMKomplexitaet}

\input{../Bloecke/Regulaer}

\input{../Bloecke/RechtslineareGrammatiken}

\begin{frame}{Turingmaschinen: Klausur}
	Noch ein Hinweis zum Schluss:\\
	Bisher kam in {\tiny fast} \textbf{jeder} Klausur eine Aufgabe zu Turingmaschinen dran.\\
	Diese gibt meist relativ viele Punkte.\\
	
	\bigskip
	Es ist hochwahrscheinlich, dass auch dieses Mal wieder eine TM-Aufgabe drankommt.\\
	\textbf{Also übt das!} Hier zählt vor allem Geschwindigkeit (und Präzision).
\end{frame}

\begin{frame}	
	\begin{block}{Was ihr nun wissen solltet}
		\begin{itemize}
			\item Turingmaschinen
			\item Komplexität
			\item Entscheidbarkeit -- Wir können alles. Außer Halteproblem. Und so Zeug.
			\item Reguläre Ausdrücke
			\item Rechtslineare Grammatiken
		\end{itemize}
	\end{block}
	
	%\begin{block}{Was nächstes Mal kommt}
	%	\begin{itemize}
	%		\item Reguläre Ausdrücke
	%		\item Rechtslineare Grammatiken
	%	\end{itemize}
	%\end{block}
\end{frame}

% TODO ?
%input{../Bloecke/StrukturelleInduktion}

%TODO replacement?
{\xkcdframe{1724}{Danke für eure Aufmerksamkeit! \smiley}{2.5}}
%\thasse{\lastframe{0.5}{0}{xkcd/proofs_1724.png}{https://www.xkcd.com/1724/}}

\slideThanks

\end{document}