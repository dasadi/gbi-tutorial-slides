\section{Eigenschaften von Relationen}

\begin{frame}{Eigenschaften von Relationen}
	\begin{block}{Totalität}
		Eine Relation $R \subseteq A \times B$ heißt \textbf{linkstotal}, wenn es für \emph{jedes} $a \in A$ ein zugehöriges $b \in B$ gibt, sodass $$(a,b) \in R.$$ \textbf{Rechtstotal} analog (für jedes $b$ ein $a$).
	\end{block}
	
	\pause
	\begin{block}{Eindeutigkeit}
		Eine Relation $R \subseteq A \times B$ heißt \textbf{linkseindeutig}, wenn für jedes $b \in B$ und $a_1, a_2 \in A$ gilt: $$\text{Wenn } \quad (a_1,b) \in R \text{ und } (a_2,b) \in R, \quad \text{dann} \quad a_1 = a_2.$$ \\
		\textbf{Rechtseindeutig} analog $\left(\text{wenn } (a, b_1) \text{ und } (a, b_2) \in R, \text{dann } b_1 = b_2 \right)$.
	\end{block}
\end{frame}

\subsection{Aufgabe 1}
\begin{frame}{Aufgabe 1 Teil 1} % (WS 2010)
	Es sei $A$ die Menge aller Kinobesucher in einer Vorstellung und $B$ die Menge aller Sitzplätze. Die Relation $R$ ordnet den Kinobesuchern die Sitzplätze zu:
	$$ R \subseteq A \times B$$
	Was bedeutet es im Kino, wenn $R$ linkstotal, linkseindeutig, rechtstotal, rechtseindeutig ist?
	\smallskip
		
	\pause
	\begin{tabular}{@{\hspace{-3pt}}r@{\ \ }l}
		\emph{linkstotal}: & jedem Kinobesucher wird mindestens ein Sitzplatz zugeteilt \\
		\pause
		\emph{linkseindeutig}: & jeder Sitzplatz ist von höchstens einem Kinobesucher belegt \\
		\pause
		\emph{rechtstotal}: & jeder Sitzplatz ist von mindestens einem Kinobesucher belegt \\ 
		\pause
		\emph{rechtseindeutig}: & kein Kinobesucher belegt mehr als einen Sitzplatz
	\end{tabular}
\end{frame}